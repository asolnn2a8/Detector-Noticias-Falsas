% Template:     Template Reporte LaTeX
% Documento:    Núcleo del template
% Versión:      2.0.1 (28/06/2020)
% Codificación: UTF-8
%
% Autor: Pablo Pizarro R.
%        Facultad de Ciencias Físicas y Matemáticas
%        Universidad de Chile
%        pablo@ppizarror.com
%
% Sitio web:    [https://latex.ppizarror.com/reporte]
% Licencia MIT: [https://opensource.org/licenses/MIT]

% CONFIGURACIONES
% Template:     Template Reporte LaTeX
% Documento:    Configuraciones del template
% Versión:      2.0.1 (28/06/2020)
% Codificación: UTF-8
%
% Autor: Pablo Pizarro R.
%        Facultad de Ciencias Físicas y Matemáticas
%        Universidad de Chile
%        pablo@ppizarror.com
%
% Sitio web:    [https://latex.ppizarror.com/reporte]
% Licencia MIT: [https://opensource.org/licenses/MIT]

% CONFIGURACIONES GENERALES
\def\addemptypagetwosides {false}  % Añade pag. en blanco al imprimir a 2 caras
\def\compilertype {pdf2latex}      % Compilador {pdf2latex,xelatex,lualatex}
\def\documentfontsize {11}         % Tamaño de la fuente del documento [pt]
\def\documentinterline {1.025}     % Interlineado del documento defecto [factor]
\def\documentlang {es-CL}          % Define el idioma del documento
\def\fontdocument {lmodern}        % Tipografía base, ver soportadas en manual
\def\fonttypewriter {tmodern}      % Tipografía de \texttt, ver manual
\def\fonturl {same}                % Tipo de fuente url {tt,sf,rm,same}
\def\importtikz {false}            % Utilizar la librería tikz
\def\pointdecimal {true}           % N° decimales con punto en vez de coma
\def\showlinenumbers {false}       % Muestra los números de línea del documento
\def\usespanishbabel {true}        % Español, desactivar para otros idiomas

% ESTILO PORTADA Y HEADER-FOOTER
\def\disablehfrightmark {false}    % Desactiva el rightmark del header-footer
\def\hfstyle {style1}              % Estilo header-footer (16 estilos)
\def\hfwidthcourse {0.35}          % Tamaño máximo del curso en header-footer
\def\hfwidthtitle {0.6}            % Tamaño máximo del título en header-footer
\def\hfwidthwrap {false}           % Activa el tamaño máximo en header-footer
\def\titlefontsize {\Large}        % Tamaño del título principal
\def\titlefontstyle {\normalfont}  % Estilo del título principal
\def\titlelinemargin {1.5em}       % Margen líneas del título principal
\def\titleshowauthor {true}        % Muestra el autor en el título principal
\def\titleshowcourse {false}       % Muestra el curso en el título principal
\def\titleshowdate {true}          % Muestra la fecha en el título principal
\def\titlesupmargin {2em}          % Margen superior del título principal

% CONFIGURACIÓN DE LAS LEYENDAS - CAPTION
\def\captionalignment {justified}  % Posición {centered,justified,left,right}
\def\captionfontsize {small}       % Tamaño de fuente de los caption
\def\captionlabelformat {simple}   % Formato leyenda {empty,simple,parens}
\def\captionlabelsep {colon}       % Sep. {none,colon,period,space,quad,newline}
\def\captionlessmarginimage {0.1}  % Margen sup/inf de fig. si no hay ley. [cm]
\def\captionlrmargin {2.0}         % Márgenes izq/der de la leyenda [cm]
\def\captionlrmarginmc {1.0}       % Mar. izq/der leyenda dentro de columnas [cm]
\def\captionmarginmultimg {0.0}    % Margen izq/der leyendas múltiple img [cm]
\def\captionnumcode {arabic}       % N° código {arabic,alph,Alph,roman,Roman}
\def\captionnumequation {arabic}   % N° ecuaciones {arabic,alph,Alph,roman,Roman}
\def\captionnumfigure {arabic}     % N° figuras {arabic,alph,Alph,roman,Roman}
\def\captionnumsubfigure {alph}    % N° subfiguras {arabic,alph,Alph,roman,Roman}
\def\captionnumsubtable {alph}     % N° subtabla {arabic,alph,Alph,roman,Roman}
\def\captionnumtable {arabic}      % N° tabla {arabic,alph,Alph,roman,Roman}
\def\captiontbmarginfigure {9.35}  % Margen sup/inf de la leyenda en fig. [pt]
\def\captiontbmargintable {7.0}    % Margen sup/inf de la leyenda en tab. [pt]
\def\captiontextbold {false}       % Etiqueta (código,figura,tabla) en negrita
\def\captiontextsubnumbold {false} % N° subfigura/subtabla en negrita
\def\codecaptiontop {true}         % Leyenda arriba del código fuente
\def\figurecaptiontop {false}      % Leyenda arriba de las imágenes
\def\sectioncaptiondelimiter {.}   % Carácter delimitador n° objeto y sección
\def\showsectioncaptioncode {none} % N° sec. código {none,chap,(s/ss/sss/ssss)ec}
\def\showsectioncaptioneqn {none}  % N° sec. ecuación {none,chap,(s/ss/sss/ssss)ec}
\def\showsectioncaptionfig {none}  % N° sec. figuras {none,chap,(s/ss/sss/ssss)ec}
\def\showsectioncaptionmat {none}  % N° matemático {none,chap,(s/ss/sss/ssss)ec}
\def\showsectioncaptiontab {none}  % N° sec. tablas {none,chap,(s/ss/sss/ssss)ec}
\def\subcaptionfsize{footnotesize} % Tamaño de la fuente de los subcaption
\def\subcaptionlabelformat{parens} % Formato leyenda sub. {empty,simple,parens}
\def\subcaptionlabelsep {space}    % Sep. {none,colon,period,space,quad,newline}
\def\tablecaptiontop {true}        % Leyenda arriba de las tablas

% ANEXO, CITAS, REFERENCIAS
\def\apacitebothers {et al.}       % Etiqueta usada en (y otros) con \shortcite
\def\apaciterefnumber {false}      % Lista de referencias con números
\def\apaciterefnumberfinal {]}     % Caracter final números apacite
\def\apaciterefnumberinit {[}      % Caracter inicial números apacite
\def\apaciterefsep {6}             % Separación entre refs. {apacite} [pt]
\def\apaciteshowurl {false}        % Muestra las url en las referencias
\def\apacitestyle {apacite}        % Formato de ref. apacite {apa,ieeetr,etc..}
\def\appendixindepobjnum {true}    % Anexo usa n° objetos independientes
\def\bibtexrefsep {6}              % Separación entre refs. {bibtex} [pt]
\def\bibtexstyle {apa}             % Formato de ref. bibtex {apa,ieeetr,etc...}
\def\bibtextextalign {justify}     % Alineación bibtex {justify,left,right,center}
\def\natbibnumbers {true}          % Forza el uso de números en la bibliografía
\def\natbibrefsep {6}              % Separación entre referencia {natbib} [pt]
\def\natbibrefstyle {ieeetr}       % Formato de ref. natbib {apa,ieeetr,etc...}
\def\natbibsquare {true}           % Usa [] o () en las numeraciones
\def\sectionappendixlastchar {.}   % Carácter entre n° de sec. anexo y título
\def\sectionrefenv {false}         % Las referencias se consideran como sección
\def\stylecitereferences {bibtex}  % Estilo cita/ref. {apacite,bibtex,natbib}
\def\twocolumnreferences {false}   % Referencias en dos columnas

% CONFIGURACIONES DE OBJETOS
\def\columnhspace {-0.4}           % Margen horizontal entre obj. \createcolumn
\def\columnsepwidth {2.1}          % Separación entre columnas [em]
\def\defaultimagefolder {../imgagenes/}     % Carpeta raíz de las imágenes
\def\equationleftalign {false}     % Ecuaciones alineadas a la izquierda
\def\equationrestart {none}        % Reinicio n° {none,chap,(s/ss/sss/ssss)ec}
\def\footnotepagetoprule {false}   % Footnote en pag. tienen separador superior
\def\footnoterestart {none}        % N° footnote {none,chap,page,(s/ss/sss/ssss)ec}
\def\fpremovetopbottomcenter{true} % Elimina espacio vertical al centrar con b!,t!
\def\imagedefaultplacement {H}     % Posición por defecto de las imágenes
\def\marginalignbottom {-0.30}     % Margen inferior entorno align [cm]
\def\marginaligncaptbottom {0.05}  % Margen inferior entorno align caption[cm]
\def\marginaligncapttop {-0.60}    % Margen superior entorno align caption [cm]
\def\marginalignedbottom {-0.30}   % Margen inferior entorno aligned [cm]
\def\marginalignedcaptbottom {0.0} % Margen inferior entorno aligned caption[cm]
\def\marginalignedcapttop {-0.60}  % Margen superior entorno aligned caption[cm]
\def\marginalignedtop {-0.40}      % Margen superior entorno aligned [cm]
\def\marginaligntop {-0.40}        % Margen superior entorno align [cm]
\def\margineqncaptionbottom {0.0}  % Margen inferior caption ecuación [cm]
\def\margineqncaptiontop {-0.65}   % Margen superior caption ecuación [cm]
\def\marginequationbottom {-0.15}  % Margen inferior ecuaciones [cm]
\def\marginequationtop {0.0}       % Margen superior ecuaciones [cm]
\def\marginfloatimages {-13.0}     % Margen sup. fig. insertimageleft/right [pt]
\def\marginfootnote {10.0}         % Margen derecho footnote [pt]
\def\margingatherbottom {-0.20}    % Margen inferior entorno gather [cm]
\def\margingathercaptbottom {0.05} % Margen inferior entorno gather caption [cm]
\def\margingathercapttop {-0.77}   % Margen superior entorno gather [cm]
\def\margingatheredbottom {-0.10}  % Margen inf. entorno gathered [cm]
\def\margingatheredcaptbottom{0.0} % Margen inf. entorno gathered caption [cm]
\def\margingatheredcapttop {-0.77} % Margen superior entorno gathered [cm]
\def\margingatheredtop {-0.40}     % Margen superior entorno gathered [cm]
\def\margingathertop {-0.40}       % Margen superior entorno gather [cm]
\def\marginimagebottom {-0.15}     % Margen inferior figura [cm]
\def\marginimagemultbottom {-0.05} % Margen inferior imágenes múltiples [cm]
\def\marginimagemultright {0.50}   % Margen derecho imágenes múltiples [cm]
\def\marginimagemulttop {-0.30}    % Margen superior imágenes múltiples [cm]
\def\marginimagetop {0.0}          % Margen superior figuras [cm]
\def\numberedequation {true}       % Ecuaciones con \insert... numeradas
\def\sourcecodefontf {\ttfamily}   % Tipo de letra código fuente
\def\sourcecodefonts {\small}      % Tamaño letra código fuente
\def\sourcecodenumbersep {6}       % Separación entre número línea y código [pt]
\def\sourcecodetabsize {3}         % Tamaño tabulación código fuente
\def\tabledefaultplacement {H}     % Posición por defecto de las tablas
\def\tablepaddingh {0.85}          % Espaciado horizontal de celda de las tablas
\def\tablepaddingv {1.05}          % Espaciado vertical de celda de las tablas
\def\tikzdefaultplacement {H}      % Posición por defecto de las figuras tikz

% CONFIGURACIÓN DE LOS TÍTULOS
\def\anumsecaddtocounter {false}   % Insertar títulos anum. aumenta n° de sec
\def\fontsizessstitle{\normalsize} % Tamaño sub-sub-subtítulos
\def\fontsizesubsubtitle {\normalsize} % Tamaño sub-subtítulos
\def\fontsizesubtitle {\large}     % Tamaño subtítulos
\def\fontsizetitle {\Large}        % Tamaño títulos
\def\showdotaftersnum {true}       % Punto al final de n° (s/ss/sss/ssss)ection
\def\stylessstitle {\bfseries}     % Estilo sub-sub-subtítulos
\def\stylesubsubtitle {\bfseries}  % Estilo sub-subtítulos
\def\stylesubtitle {\bfseries}     % Estilo subtítulos
\def\styletitle {\bfseries}        % Estilo títulos

% CONFIGURACIÓN DE LOS COLORES DEL DOCUMENTO
\def\captioncolor {black}          % Color nombre objeto (código,figura,tabla)
\def\captiontextcolor {black}      % Color de la leyenda
\def\colorpage {white}             % Color de la página
\def\highlightcolor {yellow}       % Color del subrayado con \hl
\def\linenumbercolor {gray}        % Color del n° de línea (\showlinenumbers)
\def\linkcolor {black}             % Color de los links del documento
\def\maintextcolor {black}         % Color principal del texto
\def\numcitecolor {black}          % Color del n° de las referencias o citas
\def\showborderonlinks {false}     % Color de un link por un recuadro de color
\def\sourcecodebgcolor {lgray}     % Color de fondo del código fuente
\def\ssstitlecolor {black}         % Color de los sub-sub-subtítulos
\def\subsubtitlecolor {black}      % Color de los sub-subtítulos
\def\subtitlecolor {black}         % Color de los subtítulos
\def\tablelinecolor {black}        % Color de las líneas de las tablas
\def\tablerowfirstcolor {none}     % Primer color de celda de las tablas
\def\tablerowsecondcolor {gray!20} % Segundo color de celda de las tablas
\def\titlecolor {black}            % Color de los títulos
\def\urlcolor {magenta}            % Color de los enlaces web (\href,\url)

% MÁRGENES DE PÁGINA
\def\pagemarginbottom {2.5}       % Margen inferior página [cm]
\def\pagemarginleft {2.}         % Margen izquierdo página [cm]
\def\pagemarginright {2.}        % Margen derecho página [cm]
\def\pagemargintop {2.5}          % Margen superior página [cm]

% OPCIONES DEL PDF COMPILADO
\def\cfgbookmarksopenlevel {1}     % Nivel marcadores en pdf (1:secciones)
\def\cfgpdfbookmarkopen {true}     % Expande marcadores del nivel configurado
\def\cfgpdfcenterwindow {true}     % Centra ventana del lector al abrir el pdf
\def\cfgpdfcopyright {}            % Establece el copyright del documento
\def\cfgpdfdisplaydoctitle {true}  % Muestra título del informe en visor
\def\cfgpdffitwindow {false}       % Ajusta la ventana del lector tamaño pdf
\def\cfgpdfkeywords {}             % Palabras clave del pdf
\def\cfgpdflayout {OneColumn}      % Modo de página {OneColumn,SinglePage}
\def\cfgpdfmenubar {true}          % Muestra el menú del lector
\def\cfgpdfpageview {FitH}         % {Fit,FitH,FitV,FitR,FitB,FitBH,FitBV}
\def\cfgpdfsecnumbookmarks {true}  % Número de la sec. en marcadores del pdf
\def\cfgpdftoolbar {true}          % Muestra barra de herramientas lector pdf
\def\cfgshowbookmarkmenu {false}   % Muestra menú marcadores al abrir el pdf
\def\indexdepth{4}                 % Profundidad de los marcadores
\def\pdfcompilecompression {9}     % Factor de compresión del pdf (0-9)
\def\pdfcompileobjcompression {2}  % Nivel compresión objetos del pdf (0-3)
\def\pdfcompileversion {7}         % Versión mínima del pdf compilado
\def\usepdfmetadata {true}         % Añade metadatos al pdf compilado

% NOMBRE DE OBJETOS
\def\nameabstract {Resumen}        % Nombre del resumen-abstract
\def\nameappendixsection {Anexos}  % Nombre de los anexos
\def\namemathcol {Corolario}       % Nombre de los colorarios
\def\namemathdefn {Definición}     % Nombre de las definiciones
\def\namemathej {Ejemplo}          % Nombre de los ejemplos
\def\namemathlem {Lema}            % Nombre de los lemas
\def\namemathobs {Observación}     % Nombre de las observaciones
\def\namemathprp {Proposición}     % Nombre de las proposiciones
\def\namemaththeorem {Teorema}     % Nombre de los teoremas
\def\namereferences {Referencias}  % Nombre de la sección de referencias
\def\nomchapter {Capítulo}         % Nombre de los capítulos
\def\nomltappendixsection {Anexo}  % Etiqueta sección en anexo/apéndices
\def\nomltwfigure {Figura}         % Etiqueta leyenda de las figuras
\def\nomltwsrc {Código}            % Etiqueta leyenda del código fuente
\def\nomltwtable {Tabla}           % Etiqueta leyenda de las tablas
\def\nomnpageof { de }             % Etiqueta página # de #


% IMPORTACIÓN DE LIBRERÍAS
% -----------------------------------------------------------------------------
% SE GUARDAN VARIABLES ANTES DE CARGAR LIBRERÍAS
% -----------------------------------------------------------------------------
\let\RE\Re
\let\IM\Im

% -----------------------------------------------------------------------------
% PARCHES DE LIBRERÍAS
% -----------------------------------------------------------------------------
\let\counterwithout\relax
\let\counterwithin\relax
\let\underbar\relax
\let\underline\relax

% Si se desactiva el idioma
\def\unaccentedoperators{}
\def\decimalpoint{}
\def\bibname{}

% Parche de sectsty.sty
\makeatletter
\def\underline#1{\relax\ifmmode\@@underline{#1}\else $\@@underline{\hbox{#1}}\m@th$\relax\fi}\def\underbar#1{\underline{\sbox\tw@{#1}\dp\tw@\z@\box\tw@}}
\makeatother

% -----------------------------------------------------------------------------
% LIBRERÍAS DEL NÚCLEO
% -----------------------------------------------------------------------------

% Manejo de condicionales
\usepackage{ifthen}

% Idioma español, borrar esta línea si se desea usar otro idioma
\ifthenelse{\equal{\usespanishbabel}{true}}{
	\usepackage[spanish,es-nosectiondot,es-lcroman,es-noquoting]{babel}}{
}

% Cambia el estilo de los títulos
\usepackage{sectsty}

% Codificación
\ifthenelse{\equal{\compilertype}{pdf2latex}}{
	\usepackage[utf8]{inputenc}}{
}

% Lanza un mensaje de error indicando mala configuración
%	#1	Parámetros opcionales (nostop,noheader)
%	#2	Mensaje de error
% 	#3	Configuración usada
%	#4	Valores esperados
\newcommand{\throwbadconfig}[4][]{
	\ifthenelse{\equal{#1}{noheader}}{
		\errmessage{LaTeX Warning: #4}
	}{
		\ifthenelse{\equal{#1}{noheader-nostop}}{
			\errmessage{LaTeX Warning: #4}
		}{
			\errmessage{LaTeX Warning: #2 (\noexpand #3= #3). Valores esperados: #4}
		}
	}
	\ifthenelse{\equal{#1}{nostop}}{}{
		\ifthenelse{\equal{#1}{noheader-nostop}}{}{
			\stop
		}
	}
}

% Librerías matemáticas
\ifthenelse{\equal{\equationleftalign}{true}}{
	\usepackage[fleqn]{amsmath}
}{
	\usepackage{amsmath}
}

% Tamaño de la fuente del documento
\usepackage{scrextend}
\usepackage{anyfontsize}
\changefontsizes{\documentfontsize pt}

% -----------------------------------------------------------------------------
% LIBRERÍAS INDEPENDIENTES
% -----------------------------------------------------------------------------
\usepackage{amssymb}       % Librerías matemáticas
\usepackage{amsthm}        % Definición de teoremas
\usepackage{array}         % Nuevas características a las tablas
\usepackage{bigstrut}      % Líneas horizontales en tablas
\usepackage{bm}            % Caracteres en negrita en ecuaciones
\usepackage{booktabs}      % Permite manejar elementos visuales en tablas
\usepackage{caption}       % Leyendas
\usepackage{changepage}    % Condicionales para administrar páginas
\usepackage{chngcntr}      % Añade números a las leyendas
\usepackage{color}         % Colores
\usepackage{datetime}      % Fechas
\usepackage{floatpag}      % Maneja números de páginas
\usepackage{floatrow}      % Permite administrar posiciones en los caption
\usepackage{framed}        % Permite creación de recuadros
\usepackage{gensymb}       % Simbología común
\usepackage{graphicx}      % Propiedades extra para los gráficos
\usepackage{lipsum}        % Permite crear párrafos de prueba
\usepackage{listings}      % Permite añadir código fuente
\usepackage{longtable}     % Permite utilizar tablas en varias hojas
\usepackage{mathtools}     % Permite utilizar notaciones matemáticas
\usepackage{multicol}      % Múltiples columnas
\usepackage{needspace}     % Maneja los espacios en página
\usepackage{pdflscape}     % Modo página horizontal de página
\usepackage{pdfpages}      % Permite administrar páginas en pdf
\usepackage{physics}       % Paquete de matemáticas
%\usepackage{ragged2e}      % Redefine centering
\usepackage{rotating}      % Permite rotación de objetos
\usepackage{selinput}      % Compatibilidad con acentos
\usepackage{setspace}      % Cambia el espacio entre líneas
\usepackage{soul}          % Permite subrayar texto
\usepackage{subfig}        % Permite agrupar imágenes
\usepackage{textcomp}      % Simbología común
\usepackage{url}           % Permite añadir enlaces
\usepackage{wasysym}       % Contiene caracteres misceláneos
\usepackage{wrapfig}       % Posición de imágenes
\usepackage{xspace}        % Adminsitra espacios en párrafos y líneas

% En 6.3.7 se desactiva cellspace
% \usepackage{cellspace}     % Tamaños en celdas de tablas

% \usepackage{colortbl}, xcolor al cargar table ya importa colortbl (v6.3.3)
% El paquete colortbl queda obsoleto con array+xcolor si la versión es inferior
% a 1.0c, se recomienda actualizar o ocurren errores asociados a @parbox

% Se prefiere no utilizar matlab-prettifier ya que es muy pesada para lo que
% la librería ofrece
% \usepackage{matlab-prettifier}

% -----------------------------------------------------------------------------
% LIBRERÍAS CON PARÁMETROS
% -----------------------------------------------------------------------------
\usepackage[makeroom]{cancel} % Cancelar términos en fórmulas
\usepackage[inline]{enumitem} % Permite enumerar ítems
\usepackage[subfigure,titles]{tocloft} % Maneja entradas en el índice
\usepackage[figure,table,lstlisting]{totalcount} % Contador de objetos
\usepackage[normalem]{ulem} % Permite tachar y subrayar
\usepackage[dvipsnames,table,usenames]{xcolor} % Paquete de colores avanzado

% -----------------------------------------------------------------------------
% LIBRERÍAS CONDICIONALES
% -----------------------------------------------------------------------------

% Acepta codificación UTF-8 en código fuente
\ifthenelse{\equal{\compilertype}{pdf2latex}}{
	\usepackage{listingsutf8}}{
}

% Regla superior
\ifthenelse{\equal{\footnotepagetoprule}{true}}{
	\usepackage[bottom,hang]{footmisc} % Estilo pie de página
}{
	\usepackage[bottom,norule,hang]{footmisc}
}

% Agrega punto a títulos/subtítulos
\ifthenelse{\equal{\showdotaftersnum}{true}}{
	\usepackage{secdot}
	\sectiondot{subsection}
	\sectiondot{subsubsection}}{
}

% Referencias
% Desde 6.2.8 se debe cargar al final para evitar errores:
%	- Option clash for package hyperref
% 	- name has been referenced but does not exist, replaced by a fixed one
% Desde 6.5.6 se carga después de las referencias (apacite, natbib, bibtex)
\usepackage[pdfencoding=auto,psdextra]{hyperref} % Enlaces, referencias
\ifthenelse{\equal{\stylecitereferences}{natbib}}{ % Formato citas natbib
	\usepackage[nottoc,notlof,notlot]{tocbibind}
	\ifthenelse{\equal{\natbibrefstyle}{apa}}{
		\ifthenelse{\equal{\natbibsquare}{true}}{
			\usepackage[square]{natbib}
		}{
			\usepackage[round]{natbib}
		}
	}{
	\ifthenelse{\equal{\natbibrefstyle}{ieeetr}}{
		\ifthenelse{\equal{\natbibsquare}{true}}{
			\usepackage[square,numbers]{natbib}
		}{
			\usepackage[round,numbers]{natbib}
		}
	}{
	\ifthenelse{\equal{\natbibrefstyle}{unsrt}}{
		\ifthenelse{\equal{\natbibsquare}{true}}{
			\usepackage[square,numbers]{natbib}
		}{
			\usepackage[round,numbers]{natbib}
		}
	}{
	\ifthenelse{\equal{\natbibrefstyle}{abbrvnat}}{
		\ifthenelse{\equal{\natbibsquare}{true}}{
			\usepackage[square,numbers]{natbib}
		}{
			\usepackage[round,numbers]{natbib}
		}
	}{
		\ifthenelse{\equal{\natbibnumbers}{true}}{
			\ifthenelse{\equal{\natbibsquare}{true}}{
				\usepackage[square,numbers]{natbib}
			}{
				\usepackage[round,numbers]{natbib}
			}
		}{
			\ifthenelse{\equal{\natbibsquare}{true}}{
				\usepackage[square]{natbib}
			}{
				\usepackage[round]{natbib}
			}
		}
	}}}}
}{
	\ifthenelse{\equal{\stylecitereferences}{apacite}}{ % Formato citas apacite
		\usepackage[nottoc,notlof,notlot]{tocbibind}
		\usepackage{apacite}
	}{
		\ifthenelse{\equal{\stylecitereferences}{bibtex}}{ % Formato citas bibtex
		}{}
	}
}

% Anexos/Apéndices
\def\showappendixsecindex{false}
\ifthenelse{\equal{\showappendixsecindex}{true}}{
}{
	\usepackage{appendix}
}

% Dimensiones y geometría del documento
\ifthenelse{\equal{\compilertype}{lualatex}}{ % En lualatex sólo se puede cambiar 1 vez el margen
	\usepackage[top=\pagemargintop cm,bottom=\pagemarginbottom cm,margin=\pagemarginleft cm]{geometry}
}{ % pdf2latex, xelatex
	\usepackage{geometry}
}

% -----------------------------------------------------------------------------
% -----------------------------------------------------------------------------
% LIBRERÍAS DEPENDIENTES
% -----------------------------------------------------------------------------
\usepackage{bookmark}      % Administración de marcadores en pdf
\usepackage{fancyhdr}      % Encabezados y pie de páginas
\usepackage{float}         % Administrador de posiciones de objetos
\usepackage{hyperxmp}      % Etiquetas opcionales para el pdf compilado
\usepackage{multirow}      % Agrega nuevas opciones a las tablas
\usepackage{notoccite}     % Desactiva las citas en el índice
\usepackage{titlesec}      % Administración de títulos

% -----------------------------------------------------------------------------
% TIPOGRAFÍA DEL DOCUMENTO
% -----------------------------------------------------------------------------

% Tipografías clásicas
\ifthenelse{\equal{\fontdocument}{lmodern}}{ % Default
	\usepackage{lmodern}
}{
\ifthenelse{\equal{\fontdocument}{arial}}{
	\usepackage{helvet}
	\renewcommand{\familydefault}{\sfdefault}
}{
\ifthenelse{\equal{\fontdocument}{arial2}}{
	\usepackage{arial}
}{
\ifthenelse{\equal{\fontdocument}{times}}{
	\usepackage{mathptmx}
}{
\ifthenelse{\equal{\fontdocument}{helvet}}{
	\renewcommand{\familydefault}{\sfdefault}
	\usepackage[scaled=0.95]{helvet}
	\usepackage[helvet]{sfmath}
}{
\ifthenelse{\equal{\fontdocument}{alegreyasans}}{
	\usepackage[sfdefault]{AlegreyaSans}
	\renewcommand*\oldstylenums[1]{{\AlegreyaSansOsF #1}}
}{
\ifthenelse{\equal{\fontdocument}{opensans}}{
	\usepackage[default,scale=0.95]{opensans}
}{
\ifthenelse{\equal{\fontdocument}{mathpazo}}{
	\usepackage{mathpazo}
}{

% Otros
\ifthenelse{\equal{\fontdocument}{accantis}}{
	\usepackage{accanthis}
}{
\ifthenelse{\equal{\fontdocument}{alegreya}}{
	\usepackage{Alegreya}
	\renewcommand*\oldstylenums[1]{{\AlegreyaOsF #1}}
}{
\ifthenelse{\equal{\fontdocument}{algolrevived}}{
	\usepackage{algolrevived}
}{
\ifthenelse{\equal{\fontdocument}{antiqua}}{
	\usepackage{antiqua}
}{
\ifthenelse{\equal{\fontdocument}{antpolt}}{
	\usepackage{antpolt}
}{
\ifthenelse{\equal{\fontdocument}{antpoltlight}}{
	\usepackage[light]{antpolt}
}{
\ifthenelse{\equal{\fontdocument}{anttor}}{
	\usepackage[math]{anttor}
}{
\ifthenelse{\equal{\fontdocument}{anttorcondensed}}{
	\usepackage[condensed,math]{anttor}
}{
\ifthenelse{\equal{\fontdocument}{anttorlight}}{
	\usepackage[light,math]{anttor}
}{
\ifthenelse{\equal{\fontdocument}{anttorlightcondensed}}{
	\usepackage[light,condensed,math]{anttor}
}{
\ifthenelse{\equal{\fontdocument}{arev}}{
	\usepackage{arev}
}{
\ifthenelse{\equal{\fontdocument}{arimo}}{
	\usepackage[sfdefault]{arimo}
	\renewcommand*\familydefault{\sfdefault}
}{
\ifthenelse{\equal{\fontdocument}{aurical}}{
	\usepackage{aurical}
}{
\ifthenelse{\equal{\fontdocument}{avant}}{
	\usepackage{avant}
}{
\ifthenelse{\equal{\fontdocument}{baskervald}}{
	\usepackage{baskervald}
}{
\ifthenelse{\equal{\fontdocument}{berasans}}{
	\usepackage[scaled]{berasans}
	\renewcommand*\familydefault{\sfdefault}
}{
\ifthenelse{\equal{\fontdocument}{beraserif}}{
	\usepackage{bera}
}{
\ifthenelse{\equal{\fontdocument}{biolinum}}{
	\usepackage{libertine}
	\renewcommand*\familydefault{\sfdefault}
}{
\ifthenelse{\equal{\fontdocument}{cabin}}{
	\usepackage[sfdefault]{cabin}
	\renewcommand*\familydefault{\sfdefault}
}{
\ifthenelse{\equal{\fontdocument}{cabincondensed}}{
	\usepackage[sfdefault,condensed]{cabin}
	\renewcommand*\familydefault{\sfdefault}
}{
\ifthenelse{\equal{\fontdocument}{cantarell}}{
	\usepackage[default]{cantarell}
}{
\ifthenelse{\equal{\fontdocument}{caladea}}{
	\usepackage{caladea}
}{
\ifthenelse{\equal{\fontdocument}{carlito}}{
	\usepackage[sfdefault]{carlito}
	\renewcommand*\familydefault{\sfdefault}
}{
\ifthenelse{\equal{\fontdocument}{chivolight}}{
	\usepackage[familydefault,light]{Chivo}
}{
\ifthenelse{\equal{\fontdocument}{chivoregular}}{
	\usepackage[familydefault,regular]{Chivo}
}{
\ifthenelse{\equal{\fontdocument}{clearsans}}{
	\usepackage[sfdefault]{ClearSans}
	\renewcommand*\familydefault{\sfdefault}
}{
\ifthenelse{\equal{\fontdocument}{comfortaa}}{
	\usepackage[default]{comfortaa}
}{
\ifthenelse{\equal{\fontdocument}{comicneue}}{
	\usepackage[default]{comicneue}
}{
\ifthenelse{\equal{\fontdocument}{comicneueangular}}{
	\usepackage[default,angular]{comicneue}
}{
\ifthenelse{\equal{\fontdocument}{crimson}}{
	\usepackage{crimson}
}{
\ifthenelse{\equal{\fontdocument}{cyklop}}{
	\usepackage{cyklop}
}{
\ifthenelse{\equal{\fontdocument}{dejavusans}}{
	\usepackage{DejaVuSans}
	\renewcommand*\familydefault{\sfdefault}
}{
\ifthenelse{\equal{\fontdocument}{dejavusanscondensed}}{
	\usepackage{DejaVuSansCondensed}
	\renewcommand*\familydefault{\sfdefault}
}{
\ifthenelse{\equal{\fontdocument}{droidsans}}{
	\usepackage[defaultsans]{droidsans}
	\renewcommand*\familydefault{\sfdefault}
}{
\ifthenelse{\equal{\fontdocument}{fetamont}}{ % Falta implementar aún
	\usepackage{fetamont}
	\renewcommand*\familydefault{\sfdefault}
}{
\ifthenelse{\equal{\fontdocument}{firasans}}{
	\usepackage[sfdefault]{FiraSans}
	\renewcommand*\familydefault{\sfdefault}
}{
\ifthenelse{\equal{\fontdocument}{iwona}}{
	\usepackage[math]{iwona}
}{
\ifthenelse{\equal{\fontdocument}{iwonacondensed}}{
	\usepackage[math]{iwona}
}{
\ifthenelse{\equal{\fontdocument}{iwonalight}}{
	\usepackage[light,math]{iwona}
}{
\ifthenelse{\equal{\fontdocument}{iwonalightcondensed}}{
	\usepackage[light,condensed,math]{iwona}
}{
\ifthenelse{\equal{\fontdocument}{kurier}}{
	\usepackage[math]{kurier}
}{
\ifthenelse{\equal{\fontdocument}{kuriercondensed}}{
	\usepackage[condensed,math]{kurier}
}{
\ifthenelse{\equal{\fontdocument}{kurierlight}}{
	\usepackage[light,math]{kurier}
}{
\ifthenelse{\equal{\fontdocument}{kurierlightcondensed}}{
	\usepackage[light,condensed,math]{kurier}
}{
\ifthenelse{\equal{\fontdocument}{lato}}{
	\usepackage[default]{lato}
}{
\ifthenelse{\equal{\fontdocument}{libris}}{
	\usepackage{libris}
	\renewcommand*\familydefault{\sfdefault}
}{
\ifthenelse{\equal{\fontdocument}{lxfonts}}{
	\usepackage{lxfonts}
}{
\ifthenelse{\equal{\fontdocument}{merriweather}}{
	\usepackage[sfdefault]{merriweather}
}{
\ifthenelse{\equal{\fontdocument}{merriweatherlight}}{
	\usepackage[sfdefault,light]{merriweather}
}{
\ifthenelse{\equal{\fontdocument}{mintspirit}}{
	\usepackage[default]{mintspirit}
}{
\ifthenelse{\equal{\fontdocument}{montserratalternatesextralight}}{
	\usepackage[defaultfam,extralight,tabular,lining,alternates]{montserrat}
	\renewcommand*\oldstylenums[1]{{\fontfamily{Montserrat-TOsF}\selectfont #1}}
}{
\ifthenelse{\equal{\fontdocument}{montserratalternatesregular}}{
	\usepackage[defaultfam,tabular,lining,alternates]{montserrat}
	\renewcommand*\oldstylenums[1]{{\fontfamily{Montserrat-TOsF}\selectfont #1}}
}{
\ifthenelse{\equal{\fontdocument}{montserratalternatesthin}}{
	\usepackage[defaultfam,thin,tabular,lining,alternates]{montserrat}
	\renewcommand*\oldstylenums[1]{{\fontfamily{Montserrat-TOsF}\selectfont #1}}
}{
\ifthenelse{\equal{\fontdocument}{montserratextralight}}{
	\usepackage[defaultfam,extralight,tabular,lining]{montserrat}
	\renewcommand*\oldstylenums[1]{{\fontfamily{Montserrat-TOsF}\selectfont #1}}
}{
\ifthenelse{\equal{\fontdocument}{montserratlight}}{
	\usepackage[defaultfam,light,tabular,lining]{montserrat}
	\renewcommand*\oldstylenums[1]{{\fontfamily{Montserrat-TOsF}\selectfont #1}}
}{
\ifthenelse{\equal{\fontdocument}{montserratregular}}{
	\usepackage[defaultfam,tabular,lining]{montserrat}
	\renewcommand*\oldstylenums[1]{{\fontfamily{Montserrat-TOsF}\selectfont #1}}
}{
\ifthenelse{\equal{\fontdocument}{montserratthin}}{
	\usepackage[defaultfam,thin,tabular,lining]{montserrat}
	\renewcommand*\oldstylenums[1]{{\fontfamily{Montserrat-TOsF}\selectfont #1}}
}{
\ifthenelse{\equal{\fontdocument}{nimbussans}}{
	\usepackage{nimbussans}
	\renewcommand*\familydefault{\sfdefault}
}{
\ifthenelse{\equal{\fontdocument}{noto}}{
	\usepackage[sfdefault]{noto}
	\renewcommand*\familydefault{\sfdefault}
}{
\ifthenelse{\equal{\fontdocument}{opensansserif}}{
	\usepackage[default,oldstyle,scale=0.95]{opensans}
}{
\ifthenelse{\equal{\fontdocument}{overlock}}{
	\usepackage[sfdefault]{overlock}
	\renewcommand*\familydefault{\sfdefault}
}{
\ifthenelse{\equal{\fontdocument}{paratype}}{
	\usepackage{paratype}
	\renewcommand*\familydefault{\sfdefault}
}{
\ifthenelse{\equal{\fontdocument}{paratypesanscaption}}{
	\usepackage{PTSansCaption}
	\renewcommand*\familydefault{\sfdefault}
}{
\ifthenelse{\equal{\fontdocument}{paratypesansnarrow}}{
	\usepackage{PTSansNarrow}
	\renewcommand*\familydefault{\sfdefault}
}{
\ifthenelse{\equal{\fontdocument}{quattrocento}}{
	\usepackage[sfdefault]{quattrocento}
}{
\ifthenelse{\equal{\fontdocument}{raleway}}{
	\usepackage[default]{raleway}
}{
\ifthenelse{\equal{\fontdocument}{roboto}}{
	\usepackage[sfdefault]{roboto}
}{
\ifthenelse{\equal{\fontdocument}{robotocondensed}}{
	\usepackage[sfdefault,condensed]{roboto}
}{
\ifthenelse{\equal{\fontdocument}{robotolight}}{
	\usepackage[sfdefault,light]{roboto}
}{
\ifthenelse{\equal{\fontdocument}{robotolightcondensed}}{
	\usepackage[sfdefault,light,condensed]{roboto}
}{
\ifthenelse{\equal{\fontdocument}{robotothin}}{
	\usepackage[sfdefault,thin]{roboto}
}{
\ifthenelse{\equal{\fontdocument}{rosario}}{
	\usepackage[familydefault]{Rosario}
}{
\ifthenelse{\equal{\fontdocument}{sourcesanspro}}{
	\usepackage[default]{sourcesanspro}
}{
\ifthenelse{\equal{\fontdocument}{uarial}}{
	\usepackage{uarial}
	\renewcommand*\familydefault{\sfdefault}
}{
\ifthenelse{\equal{\fontdocument}{ugq}}{
	\renewcommand*\sfdefault{ugq}
	\renewcommand*\familydefault{\sfdefault}
}{
\ifthenelse{\equal{\fontdocument}{universalis}}{
	\usepackage[sfdefault]{universalis}
}{
\ifthenelse{\equal{\fontdocument}{universaliscondensed}}{
	\usepackage[condensed,sfdefault]{universalis}
}{
\ifthenelse{\equal{\fontdocument}{venturis}}{
	\usepackage[lf]{venturis}
	\renewcommand*\familydefault{\sfdefault}
}{
	\throwbadconfig[nostop]{Fuente desconocida}{\fontdocument}{(Fuentes recomendadas) lmodern,arial,arial2,helvet,times,alegreyasans,mathpazo}
	\throwbadconfig[noheader-nostop]{Fuente desconocida}{\fontdocument}{(Fuentes adicionales) accantis,alegreya,algolrevived,antiqua,antpolt,antpoltlight,anttor,anttorcondensed,anttorlight,anttorlightcondensed,arev,arimo,aurical,avant,baskervald,berasans,beraserif,biolinum,cabin,cabincondensed,cantarell,caladea,carlito,chivolight,chivoregular,clearsans,comfortaa,comicneue,comicneueangular,crimson,cyklop,dejavusans,dejavusanscondensed,droidsans,firasans,iwona,iwonacondensed,iwonalight,iwonalightcondensed,kurier}
	\throwbadconfig[noheader-nostop]{Fuente desconocida}{\fontdocument}{kuriercondensed,kurierlight,kurierlightcondensed,lato,libris,lxfonts,merriweather,merriweatherlight,mintspirit,montserratalternatesextralight,montserratalternatesregular,montserratalternatesthin,montserratextralight,montserratlight,montserratregular,montserratthin,nimbussans,noto,opensansserif,overlock,paratype,paratypesanscaption,paratypesansnarrow,quattrocento,raleway,roboto,robotolight,robotolightcondensed,robotothin,rosario,sourcesanspro,uarial,ugq}
	\throwbadconfig[noheader]{Fuente desconocida}{\fontdocument}{universalis,universaliscondensed,venturis}
	}}}}}}}}}}}}}}}}}}}}}}}}}}}}}}}}}}}}}}}}}}}}}}}}}}}}}}}}}}}}}}}}}}}}}}}}}}}}}}}}}}}}}
}

% -----------------------------------------------------------------------------
% TIPOGRAFÍA TYPEWRITER
% -----------------------------------------------------------------------------
\ifthenelse{\equal{\fonttypewriter}{tmodern}}{ % Default
	\renewcommand*\ttdefault{lmvtt}
}{
\ifthenelse{\equal{\fonttypewriter}{anonymouspro}}{
	\usepackage[ttdefault=true]{AnonymousPro}
}{
\ifthenelse{\equal{\fonttypewriter}{ascii}}{
	\usepackage{ascii}
	\let\SI\relax
}{
\ifthenelse{\equal{\fonttypewriter}{beramono}}{
	\usepackage[scaled]{beramono}
}{
\ifthenelse{\equal{\fonttypewriter}{cmpica}}{
	\usepackage{addfont}
	\addfont{OT1}{cmpica}{\pica}
	\addfont{OT1}{cmpicab}{\picab}
	\addfont{OT1}{cmpicati}{\picati}
	\renewcommand*\ttdefault{pica}
}{
\ifthenelse{\equal{\fonttypewriter}{courier}}{
	\usepackage{courier}
}{
\ifthenelse{\equal{\fonttypewriter}{dejavusansmono}}{
	\usepackage[scaled]{DejaVuSansMono}
}{
\ifthenelse{\equal{\fonttypewriter}{firamono}}{
	\usepackage[scale=0.85]{FiraMono}
}{
\ifthenelse{\equal{\fonttypewriter}{gomono}}{
	\usepackage[scale=0.85]{GoMono}
}{
\ifthenelse{\equal{\fonttypewriter}{inconsolata}}{
	\usepackage{inconsolata}
}{
\ifthenelse{\equal{\fonttypewriter}{nimbusmono}}{
	\usepackage{nimbusmono}
}{
\ifthenelse{\equal{\fonttypewriter}{newtxtt}}{
	\usepackage[zerostyle=d]{newtxtt}
}{
\ifthenelse{\equal{\fonttypewriter}{nimbusmono}}{
	\usepackage{nimbusmono}
}{
\ifthenelse{\equal{\fonttypewriter}{nimbusmononarrow}}{
	\usepackage{nimbusmononarrow}
}{
\ifthenelse{\equal{\fonttypewriter}{lcmtt}}{
	\renewcommand*\ttdefault{lcmtt}
}{
\ifthenelse{\equal{\fonttypewriter}{sourcecodepro}}{
	\usepackage[ttdefault=true,scale=0.85]{sourcecodepro}
}{
\ifthenelse{\equal{\fonttypewriter}{texgyrecursor}}{
	\usepackage{tgcursor}
}{
	\throwbadconfig{Fuente desconocida}{\fonttypewriter}{anonymouspro,ascii,beramono,
		cmpica,courier,dejavusansmono,firamono,gomono,inconsolata,kpmonospaced,lcmtt,
		newtxtt,nimbusmono,nimbusmononarrow,texgyrecursor,tmodern}
	}}}}}}}}}}}}}}}}
}

% -----------------------------------------------------------------------------
% FINALES
% -----------------------------------------------------------------------------
\usepackage[T1]{fontenc} % Caracteres acentuados
\ifthenelse{\equal{\showlinenumbers}{true}}{ % Muestra los números de línea
	\usepackage[switch,columnwise,running]{lineno}}{
}
\usepackage{csquotes} % Citas y comillas, se debe usar después de lineno [6.4.2]
\ifthenelse{\equal{\compilertype}{pdf2latex}}{
	\inputencoding{utf8}}{
}

% IMPORTACIÓN DE FUNCIONES Y ENTORNOS
% Definición de variables globales
\def\GLOBALcaptiondefn {EMPTY-VAR}       % Definición del caption
%\def\GLOBALcaptiondefnimages{EMPTY-VAR} % Definición del caption en images
%\def\GLOBALcaptiondefnsrc{EMPTY-VAR}    % Definición del caption en sourcecode
\def\GLOBALchapternumenabled {false}     % Numeración de capítulos empezó
\def\GLOBALenvimageadded {false}         % Indica que una imagen ha sido añadida
\def\GLOBALenvimageinitialized {false}   % Entorno images activo
\def\GLOBALenvmulticol {false}           % Indica que el entorno multicol está activo
\def\GLOBALsectionalph {false}           % Sección con numeración de letras
\def\GLOBALsectionanumenabled {false}    % Sección sin numeración
\def\GLOBALsubsectionanumenabled {false} % Subsección sin numeración
\def\GLOBALtablerowcolorindex {2}        % Índice tabla colores
\def\GLOBALtablerowcolorswitch {false}   % Tabla con colores cambiados

% Contador global de objetos
\newcounter{templateEquations}      % Ecuaciones
\newcounter{templateIndexEquations} % Ecuaciones en el índice
\newcounter{templateFigures}        % Figuras
\newcounter{templatePageCounter}    % Administra números de páginas
\newcounter{templateTables}         % Tablas
\newcounter{templateListings}       % Códigos fuente

% Contador nivel de bookmarks marcadores
\newcounter{templateBookmarksLevelPrev}
\setcounter{templateBookmarksLevelPrev}{\cfgbookmarksopenlevel}
\addtocounter{templateBookmarksLevelPrev}{-1}

% Para la compatibilidad con template-tesis se define el capítulo
\newcounter{chapter}

% Otros
\let\latex\LaTeX

% Lanza un mensaje de error
% 	#1	Función del error
%	#2	Mensaje
\newcommand{\throwerror}[2]{
	\errmessage{LaTeX Error: \noexpand#1 #2 (linea \the\inputlineno)}
	\stop
}

% Lanza un mensaje de advertencia
%	#1	Mensaje
\newcommand{\throwwarning}[1]{
	\errmessage{LaTeX Warning: #1 (linea \the\inputlineno)}
}

% Lanza un mensaje de error indicando mala configuración dentro de begin{document}
%	#1	Mensaje de error
% 	#2	Configuración usada
%	#3	Valores esperados
\newcommand{\throwbadconfigondoc}[3]{
	\errmessage{#1 \noexpand #2=#2. Valores esperados: #3}
	\stop
}

% Comprueba si una variable está definida
%	#1	Variable
\newcommand{\checkvardefined}[1]{
	\ifthenelse{\isundefined{#1}}{
		\errmessage{LaTeX Warning: Variable \noexpand#1 no definida}
		\stop}{
	}
}

% Comprueba si una variable está definida
%	#1	Variable
%	#2	Mensaje
\newcommand{\checkextravarexist}[2]{
	\ifthenelse{\isundefined{#1}}{
		\errmessage{LaTeX Warning: Variable \noexpand#1 no definida}
		\ifx\hfuzz#2\hfuzz
			\errmessage{LaTeX Warning: Defina la variable en el bloque de INFORMACION DEL DOCUMENTO al comienzo del archivo principal del template}
		\else
			\errmessage{LaTeX Warning: #2}
		\fi}{
	}
}

% Lanza un mensaje de error si una variable no ha sido definida
% 	#1	Función del error
%	#2	Variable
%	#3	Mensaje
\newcommand{\emptyvarerr}[3]{
	\ifx\hfuzz#2\hfuzz
		\errmessage{LaTeX Warning: \noexpand#1 #3 (linea \the\inputlineno)}
	\fi
}

% Cambiar el margen de los caption
% 	#1	Margen en centímetros
\newcommand{\setcaptionmargincm}[1]{
	\captionsetup{margin=#1cm}
}

% Cambia márgenes de las páginas [cm]
% 	#1	Margen izquierdo
%	#2	Margen superior
%	#3	Margen derecho
%	#4	Margen inferior
\newcommand{\setpagemargincm}[4]{
	\ifthenelse{\equal{\compilertype}{lualatex}}{
		% Geometry no válido en lualatex
	}{
		\newgeometry{left=#1cm, top=#2cm, right=#3cm, bottom=#4cm}
	}
}

% Define el caption del índice
% 	#1	Título del caption
\newcommand{\setindexcaption}[1]{\def\GLOBALcaptiondefn{#1}}
%\newcommand{\setindexcaptionimages}[1]{\def\GLOBALcaptiondefnimages{#1}}
%\newcommand{\setindexcaptionsourcecode}[1]{\def\GLOBALcaptiondefnsrc{#1}}

% Resetea los caption
\newcommand{\resetindexcaption}{\def\GLOBALcaptiondefn{EMPTY-VAR}}
%\newcommand{\resetindexcaptionimages}{\def\GLOBALcaptiondefnimages{EMPTY-VAR}}
%\newcommand{\resetindexcaptionsourcecode}{\def\GLOBALcaptiondefnsrc{EMPTY-VAR}}

% Cambia los márgenes del documento
%	#1	Margen izquierdo
%	#2	Margen derecho
\newcommand{\changemargin}[2]{
	\emptyvarerr{\changemargin}{#1}{Margen izquierdo no definido}
	\emptyvarerr{\changemargin}{#2}{Margen derecho no definido}
	\list{}{\rightmargin#2\leftmargin#1}\item[]
}
\let\endchangemargin=\endlist

% Imagen de prueba tikz
% Chequea que las funciones sólo puedan usarse en el entorno images
\newcommand{\checkonlyonenvimage}{\ifthenelse{\equal{\GLOBALenvimageinitialized}{true}}{}{\throwwarning{Funciones \noexpand\addimage o \noexpand\addimageboxed no pueden usarse fuera del entorno \noexpand\images}\stop}}

% Chequea que las funciones sólo puedan usarse fuera del entorno images
\newcommand{\checkoutsideenvimage}{
	\ifthenelse{\equal{\GLOBALenvimageinitialized}{true}}{
		\throwwarning{Esta funcion solo puede usarse fuera del entorno \noexpand\images}
		\stop}{
	}
}

% Chequea que las funciones puedan usarse solo en el entorno multicol
\newcommand{\checkinsidemulticol}{
	\ifthenelse{\equal{\GLOBALenvmulticol}{false}}{
		\throwwarning{Esta funcion solo puede usarse dentro de multicols}
		\stop}{
	}
}

% Agrega una carpeta al path de imágenes
%	#1	Carpeta
\makeatletter
\newcommand\addpathimage[1]{\gappto\Ginput@path{{#1}}}
\makeatother

% Verifica que un tamaño de fuente sea correcto
%	#1	Tamaño de fuente
\newcommand{\corecheckfontsize}[1]{
	\ifthenelse{\equal{#1}{normalsize}}{}{
	\ifthenelse{\equal{#1}{small}}{}{
	\ifthenelse{\equal{#1}{large}}{}{
	\ifthenelse{\equal{#1}{Large}}{}{
	\ifthenelse{\equal{#1}{LARGE}}{}{
	\ifthenelse{\equal{#1}{huge}}{}{
	\ifthenelse{\equal{#1}{Huge}}{}{
	\ifthenelse{\equal{#1}{HUGE}}{}{
	\ifthenelse{\equal{#1}{footnotesize}}{}{
	\ifthenelse{\equal{#1}{scriptsize}}{}{
	\ifthenelse{\equal{#1}{tiny}}{}{
		\errmessage{LaTeX Warning: Tamano de fuente incorrecto (\noexpand #1= #1). Valores esperados: tiny,scriptsize,footnotesize,small,normalisize,large,Large,LARGE,huge,Huge,HUGE}
		\stop
		}}}}}}}}}}
	}
}
% Insertar sub-índice, a_b
% 	#1	Elemento inferior (a)
%	#2	Elemento superior (b)
\newcommand{\lpow}[2]{
	\ensuremath{{#1}_{#2}}
}

% Insertar elevado, a^b
% 	#1	Elemento inferior (a)
%	#2	Elemento superior (b)
\newcommand{\pow}[2]{
	\ensuremath{{#1}^{#2}}
}

% Inserta inverso función seno, sin^-1
%	#1	Elemento
\newcommand{\aasin}[1][]{
	\ifx\hfuzz#1\hfuzz
		\ensuremath{\sin^{-1}#1}
	\else
		\ensuremath{{\sin}^{-1}}
	\fi
}

% Inserta inverso función coseno, cos^-1
%	#1	Elemento
\newcommand{\aacos}[1][]{
	\ifx\hfuzz#1\hfuzz
		\ensuremath{\cos^{-1}#1}
	\else
		\ensuremath{\cos^{-1}}
	\fi
}

% Inserta inverso función tangente, tan^-1
%	#1	Elemento
\newcommand{\aatan}[1][]{
	\ifx\hfuzz#1\hfuzz
		\ensuremath{\tan^{-1}#1}
	\else
		\ensuremath{\tan^{-1}}
	\fi
}

% Inserta inverso función cosecante, csc^-1
%	#1	Elemento
\newcommand{\aacsc}[1][]{
	\ifx\hfuzz#1\hfuzz
		\ensuremath{\csc^{-1}#1}
	\else
		\ensuremath{\csc^{-1}}
	\fi
}

% Inserta inverso función secante, sec^-1
%	#1	Elemento
\newcommand{\aasec}[1][]{
	\ifx\hfuzz#1\hfuzz
		\ensuremath{\sec^{-1}#1}
	\else
		\ensuremath{\sec^{-1}}
	\fi
}

% Inserta inverso función cotangente, cot^-1
%	#1	Elemento
\newcommand{\aacot}[1][]{
	\ifx\hfuzz#1\hfuzz
		\ensuremath{\cot^{-1}#1}
	\else
		\ensuremath{\cot^{-1}}
	\fi
}

% Fracción de derivadas parciales af/ax
% 	#1	Función a derivar (f)
%	#2	Variable a derivar (x)
\newcommand{\fracpartial}[2]{
	\ensuremath{\pdv{#1}{#2}}
}

% Fracción de derivadas parciales dobles a^2f/ax^2
% 	#1	Función a derivar (f)
%	#2	Variable a derivar (x)
\newcommand{\fracdpartial}[2]{
	\ensuremath{\pdv[2]{#1}{#2}}
}

% Fracción de derivadas parciales en n, a^nf/ax^n
% 	#1	Función a derivar (f)
%	#2	Variable a derivar (x)
%	#3	Orden (n)
\newcommand{\fracnpartial}[3]{
	\ensuremath{\pdv[#3]{#1}{#2}}
}

% Fracción de derivadas df/dx
% 	#1	Función a derivar (f)
%	#2	Variable a derivar (x)
\newcommand{\fracderivat}[2]{
	\ensuremath{\dv{#1}{#2}}
}

% Fracción de derivadas dobles d^2/dx^2
% 	#1	Función a derivar (f)
%	#2	Variable a derivar (x)
\newcommand{\fracdderivat}[2]{
	\ensuremath{\dv[2]{#1}{#2}}
}

% Fracción de derivadas en n d^nf/dx^n
% 	#1	Función a derivar (f)
%	#2	Variable a derivar (x)
%	#3	Orden de la derivada (n)
\newcommand{\fracnderivat}[3]{
	\ensuremath{\dv[#3]{#1}{#2}}
}

% Llave superior de equivalencia
% 	#1	Elemento a igualar
%	#2	Igualdad
\newcommand{\topequal}[2]{
	\ensuremath{\overbrace{#1}^{\mathclap{#2}}}
}

% Llave inferior de equivalencia
% 	#1	Elemento a igualar
%	#2	Igualdad
\newcommand{\underequal}[2]{
	\ensuremath{\underbrace{#1}_{\mathclap{#2}}}
}

% Rectángulo superior de equivalencia
% 	#1	Elemento a igualar
%	#2	Igualdad
\newcommand{\topsequal}[2]{
	\ensuremath{\overbracket{#1}^{\mathclap{#2}}}
}

% Rectángulo inferior de equivalencia
% 	#1	Elemento a igualar
%	#2	Igualdad
\newcommand{\undersequal}[2]{
	\ensuremath{\underbracket{#1}_{\mathclap{#2}}}
}

% Paréntesis grande
% 	#1	Expresión
\newcommand{\bigp}[1]{
	\ensuremath{\big(#1\big)}
}

% Paréntesis g+grande
% 	#1	Expresión
\newcommand{\biggp}[1]{
	\ensuremath{\bigg(#1\bigg)}
}

% Cajón grande
% 	#1	Expresión
\newcommand{\bigc}[1]{
	\ensuremath{\big[#1\big]}
}

% Cajón g+grande
% 	#1	Expresión
\newcommand{\biggc}[1]{
	\ensuremath{\bigg[#1\bigg]}
}

% Llave grande
% 	#1	Expresión
\newcommand{\bigb}[1]{
	\ensuremath{\big\{#1\big\}}
}

% Llave g+grande
% 	#1	Expresión
\newcommand{\biggb}[1]{
	\ensuremath{\bigg\{#1\bigg\}}
}

% Expresión divergencia
\newcommand{\divexp}{
	\ensuremath{\rm{div}\ }
}

% Expresión automorfismo
\newcommand{\Autexp}{
	\ensuremath{\rm{Aut}}
}

% Expresión diff
\newcommand{\Diffexp}{
	\ensuremath{\rm{Diff}}
}

% Expresión imaginario
\newcommand{\Imexp}{
	\ensuremath{\rm{Im}}
}

% Expresión imaginario en z
\newcommand{\Imzexp}{
	\ensuremath{\rm{Im}(z)}
}

% Expresión real
\newcommand{\Reexp}{
	\ensuremath{\rm{Re}}
}

% Expresión real en z
\newcommand{\Rezexp}{
	\ensuremath{\rm{Re}(z)}
}

% Definición de letras
\newcommand{\A}{\mathcal{A}}
\let\oldC=\C
\renewcommand{\C}{\mathbb{C}}
\newcommand{\D}{\mathbb{D}}
\newcommand{\E}{\mathbb{E}}
\newcommand{\F}{\mathcal{F}}
\let\oldG=\G
\renewcommand{\G}{\mathcal{G}}
\let\oldH=\H
\renewcommand{\H}{\mathcal{H}}
\newcommand{\K}{\mathcal{K}}
\let\oldL=\L
\renewcommand{\L}{\mathcal{L}}
\newcommand{\M}{\mathcal{M}}
\newcommand{\N}{\mathbb{N}}
\let\oldP=\P
\renewcommand{\P}{\mathbb{P}}
\newcommand{\Q}{\mathbb{Q}}
\newcommand{\R}{\mathbb{R}}
\let\oldS=\S
\renewcommand{\S}{\mathcal{S}}
\newcommand{\T}{\mathcal{T}}
\newcommand{\Z}{\mathbb{Z}}

% Barra superior en elemento
%	#1 	Elemento
\newcommand{\overbar}[1]{\mkern 1.5mu\overline{\mkern-1.5mu#1\mkern-1.5mu}\mkern 1.5mu}

% Definición de teoremas y lemas
\makeatletter
	\renewenvironment{proof}[1][\proofname] {\par\pushQED{\qed}\normalfont\topsep6\p@\@plus6\p@\relax\trivlist\item[\hskip\labelsep\scshape\footnotesize#1\@addpunct{.}]\ignorespaces}{\popQED\endtrivlist\@endpefalse}
\makeatother
% Redimensiona una ecuación en textwidth
% 	#1	Tamaño del nuevo objeto (En textwidth)
%	#2	Ecuación a redimensionar
\newcommand{\equationresize}[2]{
	\emptyvarerr{\equationresize}{#1}{Dimension no definida}
	\emptyvarerr{\equationresize}{#2}{Ecuacion a redimensionar no definida}
	\resizebox{#1\textwidth}{!}{$#2$}
}

% Insertar una ecuación
% 	#1	Label (opcional)
%	#2	Ecuación
\newcommand{\insertequation}[2][]{
	\emptyvarerr{\insertequation}{#2}{Ecuacion no definida}
	\ifthenelse{\equal{\numberedequation}{true}}{
		\vspace{\marginequationtop cm}
		\begin{equation}
			\text{#1} #2
		\end{equation}
		\vspace{\marginequationbottom cm}
	}{
		\ifx\hfuzz#1\hfuzz
		\else
			\throwwarning{Label invalido en ecuacion sin numero}
		\fi
		\insertequationanum{#2}
	}
}

% Insertar una ecuación sin número
%	#1	Ecuación
\newcommand{\insertequationanum}[1]{
	\emptyvarerr{\insertequationanum}{#1}{Ecuacion no definida}
	\vspace{\marginequationtop cm}
	\begin{equation*}
		\ensuremath{#1}
	\end{equation*}
	\vspace{\marginequationbottom cm}
}

% Insertar una ecuación alineada a la izquierda
% 	#1	Label (opcional)
%	#2	Ecuación
\newcommand{\insertequationleft}[2][]{
	\emptyvarerr{\insertequationleft}{#2}{Ecuacion no definida}
	\ifthenelse{\equal{\numberedequation}{true}}{
		\vspace{\marginequationtop cm}
		\vspace{-\baselineskip}
		\begin{equation}
			\hfilneg \text{#1} #2 \hspace{10000pt minus 1fil}
		\end{equation}
		\vspace{\marginequationbottom cm}
	}{
		\ifx\hfuzz#1\hfuzz
		\else
			\throwwarning{Label invalido en ecuacion sin numero}
		\fi
		\insertequationleftanum{#2}
	}
}

% Insertar una ecuación sin número alineada a la izquierda
%	#1	Ecuación
\newcommand{\insertequationleftanum}[1]{
	\emptyvarerr{\insertequationleftanum}{#1}{Ecuacion no definida}
	\vspace{\marginequationtop cm}
	\vspace{-\baselineskip}
	\begin{equation*}
		\hfilneg \ensuremath{#1} \hspace{10000pt minus 1fil}
	\end{equation*}
	\vspace{\marginequationbottom cm}
}

% Insertar una ecuación alineada a la derecha
% 	#1	Label (opcional)
%	#2	Ecuación
\newcommand{\insertequationright}[2][]{
	\emptyvarerr{\insertequationright}{#2}{Ecuacion no definida}
	\ifthenelse{\equal{\numberedequation}{true}}{
		\vspace{\marginequationtop cm}
		\vspace{-\baselineskip}
		\begin{equation}
			\hspace{10000pt minus 1fil} \text{#1} #2 \hfilneg
		\end{equation}
		\vspace{\marginequationbottom cm}
	}{
		\ifx\hfuzz#1\hfuzz
		\else
			\throwwarning{Label invalido en ecuacion sin numero}
		\fi
		\insertequationrightanum{#2}
	}
}

% Insertar una ecuación sin número alineada a la derecha
%	#1	Ecuación
\newcommand{\insertequationrightanum}[1]{
	\emptyvarerr{\insertequationrightanum}{#1}{Ecuacion no definida}
	\vspace{\marginequationtop cm}
	\vspace{-\baselineskip}
	\begin{equation*}
		\hspace{10000pt minus 1fil} \ensuremath{#1} \hfilneg
	\end{equation*}
	\vspace{\marginequationbottom cm}
}

% Insertar una ecuación con leyenda
% 	#1	Label (opcional)
%	#2	Ecuación
%	#3	Leyenda
\newcommand{\insertequationcaptioned}[3][]{
	\emptyvarerr{\insertequationcaptioned}{#2}{Ecuacion no definida}
	\ifx\hfuzz#3\hfuzz
		\insertequation[#1]{#2}
	\else
		\ifthenelse{\equal{\numberedequation}{true}}{
			\vspace{\marginequationtop cm}
			\begin{equation}
				\text{#1} #2
			\end{equation}
			\vspace{\margineqncaptiontop cm}
			\begin{changemargin}{\captionlrmargin cm}{\captionlrmargin cm}
				\centering \textcolor{\captiontextcolor}{\begin{\captionfontsize}#3\end{\captionfontsize}}
				\vspace{\margineqncaptionbottom cm}
			\end{changemargin}
			\vspace{\margineqncaptionbottom cm}
		}{
			\ifx\hfuzz#1\hfuzz
			\else
				\throwwarning{Label invalido en ecuacion sin numero}
			\fi
			\insertequationcaptionedanum{#2}{#3}
		}
	\fi
}

% Insertar una ecuación con leyenda sin número
%	#1	Ecuación
%	#2	Leyenda
\newcommand{\insertequationcaptionedanum}[2]{
	\emptyvarerr{\insertequationcaptionedanum}{#1}{Ecuacion no definida}
	\ifx\hfuzz#2\hfuzz
		\insertequationanum{#1}
	\else
		\vspace{\marginequationtop cm}
		\begin{equation*}
			\ensuremath{#1}
		\end{equation*}
		\vspace{\margineqncaptiontop cm}
		\begin{changemargin}{\captionlrmargin cm}{\captionlrmargin cm}
			\centering \textcolor{\captiontextcolor}{\begin{\captionfontsize}#2\end{\captionfontsize}}
			\vspace{\margineqncaptionbottom cm}
		\end{changemargin}
		\vspace{\margineqncaptionbottom cm}
	\fi
}

% Insertar una ecuación con el ambiente gather
%	#1	Ecuación
\newcommand{\insertgather}[1]{
	\emptyvarerr{\insertgather}{#1}{Ecuacion no definida}
	\ifthenelse{\equal{\numberedequation}{true}}{
		\vspace{\margingathertop cm}
		\begin{gather}
			\ensuremath{#1}
		\end{gather}
		\vspace{\margingatherbottom cm}
	}{
		\insertgatheranum{#1}
	}
}

% Insertar una ecuación con el ambiente gather sin número
%	#1	Ecuación
\newcommand{\insertgatheranum}[1]{
	\emptyvarerr{\insertgatheranum}{#1}{Ecuacion no definida}
	\vspace{\margingathertop cm}
	\begin{gather*}
		\ensuremath{#1}
	\end{gather*}
	\vspace{\margingatherbottom cm}
}

% Insertar una ecuación (gather) con leyenda
%	#1	Ecuación
%	#2	Leyenda
\newcommand{\insertgathercaptioned}[2]{
	\emptyvarerr{\insertgathercaptioned}{#1}{Ecuacion no definida}
	\ifx\hfuzz#2\hfuzz
		\insertgather{#1}
	\else
		\ifthenelse{\equal{\numberedequation}{true}}{
			\vspace{\margingathertop cm}
			\begin{gather}
				\ensuremath{#1}
			\end{gather}
			\vspace{\margingathercapttop cm}
			\begin{changemargin}{\captionlrmargin cm}{\captionlrmargin cm}
				\centering \textcolor{\captiontextcolor}{\begin{\captionfontsize}#2\end{\captionfontsize}}
				\vspace{\margingathercaptbottom cm}
			\end{changemargin}
			\vspace{\margingathercaptbottom cm}
		}{
			\insertgathercaptionedanum{#1}{#2}
		}
	\fi
}

% Insertar una ecuación (gather) con leyenda sin número
%	#1	Ecuación
%	#2	Leyenda
\newcommand{\insertgathercaptionedanum}[2]{
	\emptyvarerr{\insertgathercaptionedanum}{#1}{Ecuacion no definida}
	\ifx\hfuzz#2\hfuzz
		\insertgatheranum{#1}
	\else
		\vspace{\margingathertop cm}
		\begin{gather*}
			\ensuremath{#1}
		\end{gather*}
		\vspace{\margingathercapttop cm}
		\begin{changemargin}{\captionlrmargin cm}{\captionlrmargin cm}
			\centering \textcolor{\captiontextcolor}{\begin{\captionfontsize}#2\end{\captionfontsize}}
			\vspace{\margingathercaptbottom cm}
		\end{changemargin}
		\vspace{\margingathercaptbottom cm}
	\fi
}

% Insertar una ecuación con el ambiente gathered
% 	#1	Label (opcional)
%	#2	Ecuación
\newcommand{\insertgathered}[2][]{
	\emptyvarerr{\insertgathered}{#2}{Ecuacion no definida}
	\ifthenelse{\equal{\numberedequation}{true}}{
		\vspace{\marginequationtop cm}
		\begin{equation}
			\begin{gathered}
				\text{#1} \ensuremath{#2}
			\end{gathered}
		\end{equation}
		\vspace{\margingatheredbottom cm}
	}{
		\ifx\hfuzz#1\hfuzz
		\else
			\throwwarning{Label invalido en ecuacion (gathered) sin numero}
		\fi
		\vspace{\margingatheredtop cm} % \insertgatheredanum{#2} tira warning
		\begin{gather*}
			\ensuremath{#2}
		\end{gather*}
		\vspace{\margingatheredbottom cm}
	}
}

% Insertar una ecuación con el ambiente gathered sin número
%	#1	Ecuación
\newcommand{\insertgatheredanum}[1]{
	\emptyvarerr{\insertgatheredanum}{#1}{Ecuacion no definida}
	\vspace{\margingatheredtop cm}
	\begin{gather*}
		\ensuremath{#1}
	\end{gather*}
	\vspace{\margingatheredbottom cm}
}

% Insertar una ecuación (gathered) con leyenda
% 	#1	Label (opcional)
%	#2	Ecuación
%	#3	Leyenda
\newcommand{\insertgatheredcaptioned}[3][]{
	\emptyvarerr{\insertgatheredcaptioned}{#2}{Ecuacion no definida}
	\ifx\hfuzz#3\hfuzz
		\insertgathered[#1]{#2}
	\else
		\ifthenelse{\equal{\numberedequation}{true}}{
			\vspace{\marginequationtop cm}
			\begin{equation}
				\begin{gathered}
					\text{#1} \ensuremath{#2}
				\end{gathered}
			\end{equation}
			\vspace{\margingatheredcapttop cm}
			\begin{changemargin}{\captionlrmargin cm}{\captionlrmargin cm}
				\centering \textcolor{\captiontextcolor}{\begin{\captionfontsize}#3\end{\captionfontsize}}
				\vspace{\margingatheredcaptbottom cm}
			\end{changemargin}
			\vspace{\margingatheredcaptbottom cm}
		}{
			\ifx\hfuzz#1\hfuzz
			\else
				\throwwarning{Label invalido en ecuacion (gathered) sin numero}
			\fi
			\insertgatheredcaptionedanum{#2}{#3}
		}
		\fi
}

% Insertar una ecuación (gathered) con leyenda sin número
%	#1	Ecuación
%	#2	Leyenda
\newcommand{\insertgatheredcaptionedanum}[2]{
	\emptyvarerr{\insertgatheredcaptionedanum}{#1}{Ecuacion no definida}
	\ifx\hfuzz#2\hfuzz
		\insertgatheredanum{#1}
	\else
		\vspace{\margingatheredtop cm}
		\begin{gather*}
			\ensuremath{#1}
		\end{gather*}
		\vspace{\margingatheredcapttop cm}
		\begin{changemargin}{\captionlrmargin cm}{\captionlrmargin cm}
			\centering \textcolor{\captiontextcolor}{\begin{\captionfontsize}#2\end{\captionfontsize}}
			\vspace{\margingatheredcaptbottom cm}
		\end{changemargin}
		\vspace{\margingathercaptbottom cm}
	\fi
}

% Insertar una ecuación con el ambiente align
%	#1	Ecuación
\newcommand{\insertalign}[1]{
	\emptyvarerr{\insertalign}{#1}{Ecuacion no definida}
	\ifthenelse{\equal{\numberedequation}{true}}{
		\vspace{\marginaligntop cm}
		\begin{align}
			\ensuremath{#1}
		\end{align}
		\vspace{\marginalignbottom cm}
	}{
		\insertalignanum{#1}
	}
}

% Insertar una ecuación con el ambiente align sin número
%	#1	Ecuación
\newcommand{\insertalignanum}[1]{
	\emptyvarerr{\insertalignanum}{#1}{Ecuacion no definida}
	\vspace{\marginaligntop cm}
	\begin{align*}
		\ensuremath{#1}
	\end{align*}
	\vspace{\marginalignbottom cm}
}

% Insertar una ecuación (align) con leyenda
%	#1	Ecuación
%	#2	Leyenda
\newcommand{\insertaligncaptioned}[2]{
	\emptyvarerr{\insertaligncaptioned}{#1}{Ecuacion no definida}
	\ifx\hfuzz#2\hfuzz
		\insertalign{#1}
	\else
		\ifthenelse{\equal{\numberedequation}{true}}{
			\vspace{\marginaligntop cm}
			\begin{align}
				\ensuremath{#1}
			\end{align}
			\vspace{\marginaligncapttop cm}
			\begin{changemargin}{\captionlrmargin cm}{\captionlrmargin cm}
				\centering \textcolor{\captiontextcolor}{\begin{\captionfontsize}#2\end{\captionfontsize}}
				\vspace{\marginaligncaptbottom cm}
			\end{changemargin}
			\vspace{\marginaligncaptbottom cm}
		}{
			\insertaligncaptionedanum{#1}{#2}
		}
	\fi
}

% Insertar una ecuación (align) con leyenda sin número
%	#1	Ecuación
%	#2	Leyenda
\newcommand{\insertaligncaptionedanum}[2]{
	\emptyvarerr{\insertaligncaptionedanum}{#1}{Ecuacion no definida}
	\ifx\hfuzz#2\hfuzz
		\insertalignanum{#1}
	\else
		\vspace{\marginaligntop cm}
		\begin{align*}
			\ensuremath{#1}
		\end{align*}
		\vspace{\marginaligncapttop cm}
		\begin{changemargin}{\captionlrmargin cm}{\captionlrmargin cm}
			\centering \textcolor{\captiontextcolor}{\begin{\captionfontsize}#2\end{\captionfontsize}}
			\vspace{\marginaligncaptbottom cm}
		\end{changemargin}
		\vspace{\marginaligncaptbottom cm}
	\fi
}

% Insertar una ecuación con el ambiente aligned
% 	#1	Label (opcional)
%	#2	Ecuación
\newcommand{\insertaligned}[2][]{
	\emptyvarerr{\insertaligned}{#2}{Ecuacion no definida}
	\ifthenelse{\equal{\numberedequation}{true}}{
		\vspace{\marginequationtop cm}
		\begin{equation}
			\begin{aligned}
				\text{#1} \ensuremath{#2}
			\end{aligned}
		\end{equation}
		\vspace{\marginalignedbottom cm}
	}{
		\ifx\hfuzz#1\hfuzz
		\else
			\throwwarning{Label invalido en ecuacion (aligned) sin numero}
		\fi
		\insertalignedanum{#2}
	}
}

% Insertar una ecuación con el ambiente aligned sin número
%	#1	Ecuación
\newcommand{\insertalignedanum}[1]{
	\emptyvarerr{\insertalignedanum}{#1}{Ecuacion no definida}
	\vspace{\marginalignedtop cm}
	\begin{align*}
		\ensuremath{#1}
	\end{align*}
	\vspace{\marginalignedbottom cm}
}

% Insertar una ecuación (aligned) con leyenda
% 	#1	Label (opcional)
%	#2	Ecuación
%	#3	Leyenda
\newcommand{\insertalignedcaptioned}[3][]{
	\emptyvarerr{\insertalignedcaptioned}{#2}{Ecuacion no definida}
	\ifx\hfuzz#3\hfuzz
		\insertaligned[#1]{#2}
	\else
		\ifthenelse{\equal{\numberedequation}{true}}{
			\vspace{\marginequationtop cm}
			\begin{equation}
				\begin{aligned}
					\text{#1} \ensuremath{#2}
				\end{aligned}
			\end{equation}
			\vspace{\marginalignedcapttop cm}
			\begin{changemargin}{\captionlrmargin cm}{\captionlrmargin cm}
				\centering \textcolor{\captiontextcolor}{\begin{\captionfontsize}#3\end{\captionfontsize}}
				\vspace{\marginalignedcaptbottom cm}
			\end{changemargin}
			\vspace{\marginalignedcaptbottom cm}
		}{
			\ifx\hfuzz#1\hfuzz
			\else
				\throwwarning{Label invalido en ecuacion (aligned) sin numero}
			\fi
			\insertalignedcaptionedanum{#2}{#3}
		}
	\fi
}

% Insertar una ecuación (aligned) con leyenda sin número
%	#1	Ecuación
%	#2	Leyenda
\newcommand{\insertalignedcaptionedanum}[2]{
	\emptyvarerr{\insertalignedcaptionedanum}{#1}{Ecuacion no definida}
	\ifx\hfuzz#2\hfuzz
		\insertalignedanum{#1}
	\else
		\vspace{\marginequationtop cm}
		\begin{equation}
			\begin{aligned}
				\ensuremath{#1}
			\end{aligned}
		\end{equation}
		\vspace{\marginalignedcapttop cm}
		\begin{changemargin}{\captionlrmargin cm}{\captionlrmargin cm}
			\centering \textcolor{\captiontextcolor}{\begin{\captionfontsize}#2\end{\captionfontsize}}
			\vspace{\marginalignedcaptbottom cm}
		\end{changemargin}
		\vspace{\marginalignedcaptbottom cm}
	\fi
}
% Añade una imagen a un env 'image' con borde
%	#1	Dirección de la imagen
%	#2	Parámetros de la imagen
%	#3	Leyenda de la imagen (opcional)
\newcommand{\addimage}[3]{\addimageboxed{#1}{#2}{0}{#3}}

% Añade una imagen a un env 'image' con borde
%	#1	Dirección de la imagen
%	#2	Parámetros de la imagen
%	#3	Ancho de la línea (en pt)
%	#4	Leyenda de la imagen (opcional)
\newcommand{\addimageboxed}[4]{\checkonlyonenvimage\begingroup\setlength{\fboxsep}{0 pt}\setlength{\fboxrule}{#3 pt}\ifthenelse{\equal{\GLOBALenvimageadded}{true}}{\hspace{\marginimagemultright cm}\hspace{-0.125cm}}{}\subfloat[#4]{\fbox{\includegraphics[#2]{#1}}}\endgroup\ifthenelse{\equal{\GLOBALenvimageadded}{true}}{}{\def\GLOBALenvimageadded {true}}}

% Añade una imagen a un env 'image' con borde sin leyenda
%	#1	Dirección de la imagen
%	#2	Parámetros de la imagen
\newcommand{\addimageanum}[2]{\addimageboxedanum{#1}{#2}{0}}

% Añade una imagen a un env 'image' con borde sin leyenda
%	#1	Dirección de la imagen
%	#2	Parámetros de la imagen
%	#3	Ancho de la línea (en pt)
\newcommand{\addimageboxedanum}[3]{\checkonlyonenvimage\begingroup\setlength{\fboxsep}{0 pt}\setlength{\fboxrule}{#3 pt}\ifthenelse{\equal{\GLOBALenvimageadded}{true}}{\hspace{\marginimagemultright cm}\hspace{-0.125cm}}{}\subfloat{\fbox{\includegraphics[#2]{#1}}}\endgroup\ifthenelse{\equal{\GLOBALenvimageadded}{true}}{}{\def\GLOBALenvimageadded {true}}}

% Añade un salto de línea en las imágenes
\newcommand{\imagesnewline}{\checkonlyonenvimage\def\GLOBALenvimageadded {false}\vspace{\marginimagemultbottom cm}\linebreak\noindent}

% Agrega un espacio horizontal a las imágenes
% 	#1 Tamaño del espacio
\newcommand{\imageshspace}[1]{\checkonlyonenvimage\def\GLOBALenvimageadded {false}\hspace{#1}}

% Agrega un espacio vertical a las imágenes
% 	#1 Tamaño del espacio
\newcommand{\imagesvspace}[1]{\checkonlyonenvimage\def\GLOBALenvimageadded {false}~ \\ \vspace*{#1}}

% Insertar una imagen
% 	#1	Label (opcional)
%	#2	Dirección de la imagen
%	#3	Parámetros de la imagen
%	#4	Leyenda de la imagen (opcional)
\newcommand{\insertimage}[4][]{%
	\insertimageboxed[#1]{#2}{#3}{0}{#4}
}

% Insertar una imagen con recuadro
% 	#1	Label (opcional)
%	#2	Dirección de la imagen
%	#3	Parámetros de la imagen
%	#4	Ancho de la línea (en pt)
%	#5	Leyenda de la imagen (opcional)
\newcommand{\insertimageboxed}[5][]{%
	\emptyvarerr{\insertimageboxed}{#2}{Direccion de la imagen no definida}
	\emptyvarerr{\insertimageboxed}{#3}{Parametros de la imagen no definidos}
	\emptyvarerr{\insertimageboxed}{#4}{Ancho de la linea no definido}
	\checkoutsideenvimage
	\vspace{\marginimagetop cm}
	\begin{figure}[H]%
		\thisfloatpagestyle{fancy}
		\begingroup
			\setlength{\fboxsep}{0 pt}
			\setlength{\fboxrule}{#4 pt}
			\centering
			\fbox{\includegraphics[#3]{#2}}%
		\endgroup
		\ifx\hfuzz#5\hfuzz
			\vspace{\captionlessmarginimage cm}
		\else
			\hspace{0cm}
			\ifthenelse{\equal{\GLOBALcaptiondefn}{EMPTY-VAR}}{\caption{#5 #1}}{\caption[\GLOBALcaptiondefn]{#5 #1}}
		\fi
	\end{figure}
	\vspace{\marginimagebottom cm}
	\resetindexcaption
}

% Insertar una imagen completa en un entorno multicol
% 	#1	Label (opcional)
%	#2	Dirección de la imagen
%	#3	Parámetros de la imagen
%	#4	Posición, "bottom" o "top"
%	#5	Leyenda de la imagen (opcional)
\newcommand{\insertimagemc}[5][]{%
	\insertimageboxedmc[#1]{#2}{#3}{0}{#4}{#5}
}

% Insertar una imagen completa con recuadro en un entorno multicol
% 	#1	Label (opcional)
%	#2	Dirección de la imagen
%	#3	Parámetros de la imagen
%	#4	Ancho de la línea (en pt)
%	#5	Posición, "bottom" o "top"
%	#6	Leyenda de la imagen (opcional)
\newcommand{\insertimageboxedmc}[6][]{%
	\emptyvarerr{\insertimageboxedmc}{#2}{Direccion de la imagen no definida}
	\emptyvarerr{\insertimageboxedmc}{#3}{Parametros de la imagen no definidos}
	\emptyvarerr{\insertimageboxedmc}{#4}{Ancho de la linea no definido}
	\emptyvarerr{\insertimageboxedmc}{#5}{Posicion de la imagen no definida}
	\checkoutsideenvimage
	\checkinsidemulticol
	\setcaptionmargincm{\captionlrmarginmc}
	\ifthenelse{\equal{#5}{bottom}}{
		\begin{figure*}[b!]
	}{
	\ifthenelse{\equal{#5}{top}}{
		\begin{figure*}[t!]
	}{
		\errmessage{LaTeX Warning: Posicion de imagen invalida, valores esperados: bottom,top}
	}}
		\thisfloatpagestyle{fancy}
		\begingroup
			\setlength{\fboxsep}{0 pt}
			\setlength{\fboxrule}{#4 pt}
			\centering
			\fbox{\includegraphics[#3]{#2}}%
		\endgroup
		\ifx\hfuzz#6\hfuzz
			\vspace{\captionlessmarginimage cm}
		\else
			\hspace{0cm}
			\ifthenelse{\equal{\GLOBALcaptiondefn}{EMPTY-VAR}}{\caption{#6 #1}}{\caption[\GLOBALcaptiondefn]{#6 #1}}
		\fi
	\end{figure*}
	\setcaptionmargincm{\captionlrmargin}
	\resetindexcaption
}

% Insertar una imagen dentro de una tabla
%	#1	Dirección de la imagen
%	#2	Parámetros de la imagen
\newcommand{\inserttableimage}[2]{\inserttableimageboxed{#1}{#2}{0}}

% Insertar una imagen dentro de una tabla con recuadro
%	#1	Dirección de la imagen
%	#2	Parámetros de la imagen
%	#3	Ancho de la línea (en pt)
\newcommand{\inserttableimageboxed}[3]{\emptyvarerr{\inserttableimageboxed}{#1}{Direccion de la imagen no definida}\emptyvarerr{\inserttableimageboxed}{#2}{Parametros de la imagen no definidos}\emptyvarerr{\inserttableimageboxed}{#3}{Ancho de la linea no definido}\checkoutsideenvimage\begingroup\setlength{\fboxsep}{0 pt}\setlength{\fboxrule}{#3 pt}\raisebox{-1\totalheight}{\fbox{\includegraphics[#2]{#1}}}\endgroup\resetindexcaption}

% Insertar una imagen a la izquierda, escalada, ancho fijo
% 	#1	Label (opcional)
%	#2	Dirección de la imagen
%	#3	Ancho de la imagen (en textwidth)
%	#4	Leyenda de la imagen (opcional)
\newcommand{\insertimageleft}[4][]{%
	\insertimageleftboxed[#1]{#2}{#3}{0}{#4}
}

% Insertar una imagen a la izquierda, escalada, ancho fijo
% 	#1	Label (opcional)
%	#2	Dirección de la imagen
%	#3	Ancho de la imagen (en textwidth)
%	#4	Ancho de la línea (en pt)
%	#5	Leyenda de la imagen (opcional)
\newcommand{\insertimageleftboxed}[5][]{%
	\emptyvarerr{\insertimageleftboxed}{#2}{Direccion de la imagen no definida}
	\emptyvarerr{\insertimageleftboxed}{#3}{Ancho de la imagen no definido}
	\emptyvarerr{\insertimageleftboxed}{#4}{Ancho de la linea no definido}
	\checkoutsideenvimage
	~
	\vspace{-\baselineskip}
	\par%
	\begin{wrapfigure}{l}{#3\textwidth}%
		% \thisfloatpagestyle{fancy} no funciona bien
		\setcaptionmargincm{0}
		\ifthenelse{\equal{\figurecaptiontop}{true}}{}{
			\vspace{\marginfloatimages pt}
		}
		\begingroup
			\setlength{\fboxsep}{0 pt}
			\setlength{\fboxrule}{#4 pt}
			\centering
			\fbox{\includegraphics[width=\linewidth]{#2}}%
		\endgroup
		\ifx\hfuzz#5\hfuzz
			\vspace{\captionlessmarginimage cm}
		\else
			\ifthenelse{\equal{\GLOBALcaptiondefn}{EMPTY-VAR}}{\caption{#5 #1}}{\caption[\GLOBALcaptiondefn]{#5 #1}}
		\fi
	\end{wrapfigure}
	\setcaptionmargincm{\captionlrmargin}
	\resetindexcaption
}

% Insertar una imagen a la izquierda, ajustada en un número de líneas, escalada, ancho fijo
% 	#1	Label (opcional)
%	#2	Dirección de la imagen
%	#3	Ancho de la imagen (en textwidth)
%	#4	Altura en líneas de la imagen
%	#5	Leyenda de la imagen (opcional)
\newcommand{\insertimageleftline}[5][]{%
	\insertimageleftlineboxed[#1]{#2}{#3}{0}{#4}{#5}
}

% Insertar una imagen recuadrada a la izquierda, ajustada en un número de líneas, escalada, ancho fijo
% 	#1	Label (opcional)
%	#2	Dirección de la imagen
%	#3	Ancho de la imagen (en textwidth)
%	#4	Ancho de la línea (en pt)
%	#5	Altura en líneas de la imagen
%	#6	Leyenda de la imagen (opcional)
\newcommand{\insertimageleftlineboxed}[6][]{%
	\emptyvarerr{\insertimageleftlineboxed}{#2}{Direccion de la imagen no definida}
	\emptyvarerr{\insertimageleftlineboxed}{#3}{Ancho de la imagen no definido}
	\emptyvarerr{\insertimageleftlineboxed}{#4}{Ancho de la linea no definido}
	\emptyvarerr{\insertimageleftlineboxed}{#5}{Altura en lineas de la imagen flotante izquierda no definida}
	\checkoutsideenvimage
	~
	\vspace{-\baselineskip}
	\par%
	\begin{wrapfigure}[#5]{l}{#3\textwidth}%
		% \thisfloatpagestyle{fancy} no funciona bien
		\setcaptionmargincm{0}
		\ifthenelse{\equal{\figurecaptiontop}{true}}{}{
			\vspace{\marginfloatimages pt}}
		\begingroup
			\setlength{\fboxsep}{0 pt}
			\setlength{\fboxrule}{#4 pt}
			\centering
			\fbox{\includegraphics[width=\linewidth]{#2}}%
		\endgroup
		\ifx\hfuzz#6\hfuzz
			\vspace{\captionlessmarginimage cm}
		\else
			\ifthenelse{\equal{\GLOBALcaptiondefn}{EMPTY-VAR}}{\caption{#6 #1}}{\caption[\GLOBALcaptiondefn]{#6 #1}}
		\fi
	\end{wrapfigure}
	\setcaptionmargincm{\captionlrmargin}
	\resetindexcaption
}

% Insertar una imagen a la derecha, escalada, ancho fijo
% 	#1	Label (opcional)
%	#2	Dirección de la imagen
%	#3	Ancho de la imagen (en textwidth)
%	#4	Leyenda de la imagen (opcional)
\newcommand{\insertimageright}[4][]{%
	\insertimagerightboxed[#1]{#2}{#3}{0}{#4}
}

% Insertar una imagen recuadrada a la derecha, escalada, ancho fijo
% 	#1	Label (opcional)
%	#2	Dirección de la imagen
%	#3	Ancho de la imagen (en textwidth)
%	#4	Ancho de la línea (en pt)
%	#5	Leyenda de la imagen (opcional)
\newcommand{\insertimagerightboxed}[5][]{%
	\emptyvarerr{\insertimagerightboxed}{#2}{Direccion de la imagen no definida}
	\emptyvarerr{\insertimagerightboxed}{#3}{Ancho de la imagen no defindo}
	\emptyvarerr{\insertimagerightboxed}{#4}{Ancho de la linea no definido}
	\checkoutsideenvimage
	~
	\vspace{-\baselineskip}
	\par%
	\begin{wrapfigure}{r}{#3\textwidth}%
		% \thisfloatpagestyle{fancy} no funciona bien
		\setcaptionmargincm{0}
		\ifthenelse{\equal{\figurecaptiontop}{true}}{}{
			\vspace{\marginfloatimages pt}
		}
		\begingroup
			\setlength{\fboxsep}{0 pt}
			\setlength{\fboxrule}{#4 pt}
			\centering
			\fbox{\includegraphics[width=\linewidth]{#2}}%
		\endgroup
		\ifx\hfuzz#5\hfuzz
			\vspace{\captionlessmarginimage cm}
		\else
			\ifthenelse{\equal{\GLOBALcaptiondefn}{EMPTY-VAR}}{\caption{#5 #1}}{\caption[\GLOBALcaptiondefn]{#5 #1}}
		\fi
	\end{wrapfigure}
	\setcaptionmargincm{\captionlrmargin}
	\resetindexcaption
}

% Insertar una imagen a la derecha, ajustada en un número de líneas, escalada, ancho fijo
% 	#1	Label (opcional)
%	#2	Dirección de la imagen
%	#3	Ancho de la imagen (en textwidth)
%	#4	Altura en líneas de la imagen
%	#5	Leyenda de la imagen (opcional)
\newcommand{\insertimagerightline}[5][]{%
	\insertimagerightlineboxed[#1]{#2}{#3}{0}{#4}{#5}
}

% Insertar una imagen recuadrada a la derecha, ajustada en un número de líneas, escalada, ancho fijo
% 	#1	Label (opcional)
%	#2	Dirección de la imagen
%	#3	Ancho de la imagen (en textwidth)
%	#4	Ancho de la línea (en pt)
%	#5	Altura en líneas de la imagen
%	#6	Leyenda de la imagen (opcional)
\newcommand{\insertimagerightlineboxed}[6][]{%
	\emptyvarerr{\insertimagerightlineboxed}{#2}{Direccion de la imagen no definida}
	\emptyvarerr{\insertimagerightlineboxed}{#3}{Ancho de la imagen no defindo}
	\emptyvarerr{\insertimagerightlineboxed}{#4}{Ancho de la linea no definido}
	\emptyvarerr{\insertimagerightlineboxed}{#5}{Altura en lineas de la imagen flotante derecha no definida}
	\checkoutsideenvimage
	~
	\vspace{-\baselineskip}
	\par%
	\begin{wrapfigure}[#5]{r}{#3\textwidth}%
		% \thisfloatpagestyle{fancy} no funciona bien
		\setcaptionmargincm{0}
		\ifthenelse{\equal{\figurecaptiontop}{true}}{}{
			\vspace{\marginfloatimages pt}
		}
		\begingroup
			\setlength{\fboxsep}{0 pt}
			\setlength{\fboxrule}{#4 pt}
			\centering
			\fbox{\includegraphics[width=\linewidth]{#2}}%
		\endgroup
		\ifx\hfuzz#6\hfuzz
			\vspace{\captionlessmarginimage cm}
		\else
			\ifthenelse{\equal{\GLOBALcaptiondefn}{EMPTY-VAR}}{\caption{#6 #1}}{\caption[\GLOBALcaptiondefn]{#6 #1}}
		\fi
	\end{wrapfigure}
	\setcaptionmargincm{\captionlrmargin}
	\resetindexcaption
}

% Insertar una imagen a la izquierda, propiedades variables
% 	#1	Label (opcional)
%	#2	Dirección de la imagen
%	#3	Ancho del objeto
%	#4	Propiedades de la imagen
%	#5	Leyenda de la imagen (opcional)
\newcommand{\insertimageleftp}[5][]{%
	\xspace~\\%
	\vspace{-2\baselineskip}%
	\par%
	\insertimageleftboxedp[#1]{#2}{#3}{#4}{0}{#5}
}

% Insertar una imagen a la izquierda, propiedades variables
% 	#1	Label (opcional)
%	#2	Dirección de la imagen
%	#3	Ancho del objeto
%	#4	Propiedades de la imagen
%	#5	Ancho de la línea (en pt)
%	#6	Leyenda de la imagen (opcional)
\newcommand{\insertimageleftboxedp}[6][]{%
	\emptyvarerr{\insertimageleftboxedp}{#2}{Direccion de la imagen no definida}
	\emptyvarerr{\insertimageleftboxedp}{#3}{Ancho del objeto no definido}
	\emptyvarerr{\insertimageleftboxedp}{#4}{Propiedades de la imagen no defindos}
	\emptyvarerr{\insertimageleftboxedp}{#5}{Ancho de la linea no definido}
	\checkoutsideenvimage
	~
	\vspace{-\baselineskip}
	\par%
	\begin{wrapfigure}{l}{#3}%
		% \thisfloatpagestyle{fancy} no funciona bien
		\setcaptionmargincm{0}
		\ifthenelse{\equal{\figurecaptiontop}{true}}{}{
			\vspace{\marginfloatimages pt}
		}
		\begingroup
			\setlength{\fboxsep}{0 pt}
			\setlength{\fboxrule}{#5 pt}
			\centering
			\fbox{\includegraphics[#4]{#2}}%
		\endgroup
		\ifx\hfuzz#6\hfuzz
			\vspace{\captionlessmarginimage cm}
		\else
			\ifthenelse{\equal{\GLOBALcaptiondefn}{EMPTY-VAR}}{\caption{#6 #1}}{\caption[\GLOBALcaptiondefn]{#6 #1}}
		\fi
	\end{wrapfigure}
	\setcaptionmargincm{\captionlrmargin}
	\resetindexcaption
}

% Insertar una imagen a la izquierda, ajustada en un número de líneas, propiedades variables
% 	#1	Label (opcional)
%	#2	Dirección de la imagen
%	#3	Ancho del objeto
%	#4	Propiedades de la imagen
%	#5	Altura en líneas de la imagen
%	#6	Leyenda de la imagen (opcional)
\newcommand{\insertimageleftlinep}[6][]{%
	\insertimageleftlineboxedp[#1]{#2}{#3}{#4}{0}{#5}{#6}
}

% Insertar una imagen recuadrada a la izquierda, ajustada en un número de líneas, propiedades variables
% 	#1	Label (opcional)
%	#2	Dirección de la imagen
%	#3	Ancho del objeto
%	#4	Propiedades de la imagen
%	#5	Ancho de la línea (en pt)
%	#6	Altura en líneas de la imagen
%	#7	Leyenda de la imagen (opcional)
\newcommand{\insertimageleftlineboxedp}[7][]{
	\emptyvarerr{\insertimageleftlineboxedp}{#2}{Direccion de la imagen no definida}
	\emptyvarerr{\insertimageleftlineboxedp}{#3}{Ancho del objeto no definido}
	\emptyvarerr{\insertimageleftlineboxedp}{#4}{Propiedades de la imagen no definidos}
	\emptyvarerr{\insertimageleftlineboxedp}{#5}{Ancho de la linea no definido}
	\emptyvarerr{\insertimageleftlineboxedp}{#6}{Altura en lineas de la imagen flotante izquierda no definida}
	\checkoutsideenvimage
	~
	\vspace{-\baselineskip}
	\par%
	\begin{wrapfigure}[#6]{l}{#3}%
		% \thisfloatpagestyle{fancy} no funciona bien
		\setcaptionmargincm{0}
		\ifthenelse{\equal{\figurecaptiontop}{true}}{}{
			\vspace{\marginfloatimages pt}
		}
		\begingroup
			\setlength{\fboxsep}{0 pt}
			\setlength{\fboxrule}{#5 pt}
			\centering
			\fbox{\includegraphics[#4]{#2}}%
		\endgroup
		\ifx\hfuzz#7\hfuzz
			\vspace{\captionlessmarginimage cm}
		\else
			\ifthenelse{\equal{\GLOBALcaptiondefn}{EMPTY-VAR}}{\caption{#7 #1}}{\caption[\GLOBALcaptiondefn]{#7 #1}}
		\fi
	\end{wrapfigure}
	\setcaptionmargincm{\captionlrmargin}
	\resetindexcaption
}

% Insertar una imagen a la derecha, propiedades variables
% 	#1	Label (opcional)
%	#2	Dirección de la imagen
%	#3	Ancho del objeto (en cm)
%	#4	Propiedades de la imagen
%	#5	Leyenda de la imagen (opcional)
\newcommand{\insertimagerightp}[5][]{%
	\xspace~\\%
	\vspace{-2\baselineskip}%
	\par%
	\insertimagerightboxedp[#1]{#2}{#3}{#4}{0}{#5}
}

% Insertar una imagen recuadrada a la derecha, propiedades variables
% 	#1	Label (opcional)
%	#2	Dirección de la imagen
%	#3	Ancho del objeto
%	#4	Propiedades de la imagen
%	#5	Ancho de la línea (en pt)
%	#6	Leyenda de la imagen (opcional)
\newcommand{\insertimagerightboxedp}[6][]{
	\emptyvarerr{\insertimagerightboxedp}{#2}{Direccion de la imagen no definida}
	\emptyvarerr{\insertimagerightboxedp}{#3}{Ancho del objeto no definido}
	\emptyvarerr{\insertimagerightboxedp}{#4}{Propiedades de la imagen no definidos}
	\emptyvarerr{\insertimagerightboxedp}{#5}{Ancho de la linea no definido}
	\checkoutsideenvimage
	~
	\vspace{-\baselineskip}
	\par%
	\begin{wrapfigure}{r}{#3}%
		% \thisfloatpagestyle{fancy} no funciona bien
		\setcaptionmargincm{0}
		\ifthenelse{\equal{\figurecaptiontop}{true}}{}{
			\vspace{\marginfloatimages pt}
		}
		\begingroup
			\setlength{\fboxsep}{0 pt}
			\setlength{\fboxrule}{#5 pt}
			\centering
			\fbox{\includegraphics[#4]{#2}}%
		\endgroup
		\ifx\hfuzz#6\hfuzz
			\vspace{\captionlessmarginimage cm}
		\else
			\ifthenelse{\equal{\GLOBALcaptiondefn}{EMPTY-VAR}}{\caption{#6 #1}}{\caption[\GLOBALcaptiondefn]{#6 #1}}
		\fi
	\end{wrapfigure}
	\setcaptionmargincm{\captionlrmargin}
	\resetindexcaption
}

% Insertar una imagen a la derecha, ajustada en un número de líneas, propiedades variables
% 	#1	Label (opcional)
%	#2	Dirección de la imagen
%	#3	Ancho del objeto (en cm)
%	#4	Propiedades de la imagen
%	#5	Altura en líneas de la imagen
%	#6	Leyenda de la imagen (opcional)
\newcommand{\insertimagerightlinep}[6][]{%
	\insertimagerightlineboxedp[#1]{#2}{#3}{#4}{0}{#5}{#6}
}

% Insertar una imagen recuadrada a la derecha, ajustada en un número de líneas, propiedades variables
% 	#1	Label (opcional)
%	#2	Dirección de la imagen
%	#3	Ancho del objeto
%	#4	Propiedades de la imagen
%	#5	Ancho de la línea (en pt)
%	#6	Altura en líneas de la imagen
%	#7	Leyenda de la imagen (opcional)
\newcommand{\insertimagerightlineboxedp}[7][]{%
	\emptyvarerr{\insertimagerightlineboxedp}{#2}{Direccion de la imagen no definida}
	\emptyvarerr{\insertimagerightlineboxedp}{#3}{Ancho del objeto no definido}
	\emptyvarerr{\insertimagerightlineboxedp}{#4}{Propiedades de la imagen no definidos}
	\emptyvarerr{\insertimagerightlineboxedp}{#5}{Ancho de la linea no definido}
	\emptyvarerr{\insertimagerightlineboxedp}{#6}{Altura en lineas de la imagen flotante derecha no definida}
	\checkoutsideenvimage
	~
	\vspace{-\baselineskip}
	\par%
	\begin{wrapfigure}[#6]{r}{#3}%
		% \thisfloatpagestyle{fancy} no funciona bien
		\setcaptionmargincm{0}
		\ifthenelse{\equal{\figurecaptiontop}{true}}{}{
			\vspace{\marginfloatimages pt}
		}
		\begingroup
			\setlength{\fboxsep}{0 pt}
			\setlength{\fboxrule}{#5 pt}
			\centering
			\fbox{\includegraphics[#4]{#2}}%
		\endgroup
		\ifx\hfuzz#7\hfuzz
			\vspace{\captionlessmarginimage cm}
		\else
			\ifthenelse{\equal{\GLOBALcaptiondefn}{EMPTY-VAR}}{\caption{#7 #1}}{\caption[\GLOBALcaptiondefn]{#7 #1}}
		\fi
	\end{wrapfigure}
	\setcaptionmargincm{\captionlrmargin}
	\resetindexcaption
}

% Define la clave resolution al insertar imágenes
\makeatletter
\define@key{Gin}{resolution}{\pdfimageresolution=#1\relax}
\makeatother
% Activa la numeración en las secciones
\def\coreintializetitlenumbering {
	% section
	\ifthenelse{\equal{\GLOBALchapternumenabled}{false}}{
		\ifthenelse{\equal{\GLOBALsectionalph}{true}}{
			\renewcommand{\thesection}{\Alph{section}}
		}{
			\renewcommand{\thesection}{\arabic{section}}
		}
	}{
		\ifthenelse{\equal{\GLOBALsectionalph}{true}}{
			\renewcommand{\thesection}{\thechapter.\Alph{section}}
		}{
			\renewcommand{\thesection}{\thechapter.\arabic{section}}
		}
	}
	% subsection
	\ifthenelse{\equal{\GLOBALsectionanumenabled}{true}}{
		\renewcommand{\thesubsection}{\arabic{subsection}}
	}{
		\renewcommand{\thesubsection}{\thesection.\arabic{subsection}}
	}
	% subsubsection
	\ifthenelse{\equal{\GLOBALsubsectionanumenabled}{true}}{
		\renewcommand{\thesubsubsection}{\arabic{subsubsection}}
	}{
		\renewcommand{\thesubsubsection}{\thesubsection.\arabic{subsubsection}}
	}
	% subsubsubsection
	\ifthenelse{\equal{\GLOBALsubsectionanumenabled}{true}}{
		\ifthenelse{\equal{\showdotaftersnum}{true}}{
			\renewcommand{\thesubsubsubsection}{\arabic{subsubsection}.\arabic{subsubsubsection}.}
		}{
			\renewcommand{\thesubsubsubsection}{\arabic{subsubsection}.\arabic{subsubsubsection}}
		}
	}{
		\ifthenelse{\equal{\showdotaftersnum}{true}}{
			\renewcommand{\thesubsubsubsection}{\thesubsubsection.\arabic{subsubsubsection}.}
		}{
			\renewcommand{\thesubsubsubsection}{\thesubsubsection.\arabic{subsubsubsection}}
		}
	}
}

% Parcha el formato de capítulos
\pretocmd{\chapter}{
	\ifthenelse{\equal{\showsectioncaptioncode}{chap}}{ % Reinicia código fuente
		\addtocounter{templateListings}{\value{lstlisting}}
		\setcounter{lstlisting}{0}
	}{}
	\ifthenelse{\equal{\showsectioncaptioneqn}{chap}}{ % Reinicia ecuaciones
		\addtocounter{templateEquations}{\value{equation}}
		\setcounter{equation}{0}
	}{}
	\ifthenelse{\equal{\equationrestart}{chap}}{ % Reinicia ecuaciones
		\addtocounter{templateEquations}{\value{equation}}
		\setcounter{equation}{0}
	}{}
	\ifthenelse{\equal{\showsectioncaptionfig}{chap}}{ % Reinicia figuras
		\addtocounter{templateFigures}{\value{figure}}
		\setcounter{figure}{0}
	}{}
	\ifthenelse{\equal{\showsectioncaptiontab}{chap}}{ % Reinicia tablas
		\addtocounter{templateTables}{\value{table}}
		\setcounter{table}{0}
	}{}
	\def\GLOBALchapternumenabled{true}
	\coreintializetitlenumbering
}{}{}

% Parcha el formato de secciones al pasar desde una anum, vuelve a activar número
% de la sección
\pretocmd{\section}{
	\ifthenelse{\equal{\showsectioncaptioncode}{sec}}{ % Reinicia código fuente
		\addtocounter{templateListings}{\value{lstlisting}}
		\setcounter{lstlisting}{0}
	}{}
	\ifthenelse{\equal{\showsectioncaptioneqn}{sec}}{ % Reinicia ecuaciones
		\addtocounter{templateEquations}{\value{equation}}
		\setcounter{equation}{0}
	}{}
	\ifthenelse{\equal{\equationrestart}{sec}}{ % Reinicia ecuaciones
		\addtocounter{templateEquations}{\value{equation}}
		\setcounter{equation}{0}
	}{}
	\ifthenelse{\equal{\showsectioncaptionfig}{sec}}{ % Reinicia figuras
		\addtocounter{templateFigures}{\value{figure}}
		\setcounter{figure}{0}
	}{}
	\ifthenelse{\equal{\showsectioncaptiontab}{sec}}{ % Reinicia tablas
		\addtocounter{templateTables}{\value{table}}
		\setcounter{table}{0}
	}{}
	\def\GLOBALsectionanumenabled{false}
	\def\GLOBALsubsectionanumenabled{false}
	\coreintializetitlenumbering
}{}{}

% Comienza nueva subsección, si está dentro de una sectionanum entonces no dibuja el
% número de sección, si no entonces dibuja el número de forma normal
\pretocmd{\subsection}{
	\ifthenelse{\equal{\showsectioncaptioncode}{ssec}}{ % Reinicia código fuente
		\addtocounter{templateListings}{\value{lstlisting}}
		\setcounter{lstlisting}{0}
	}{}
	\ifthenelse{\equal{\showsectioncaptioneqn}{ssec}}{ % Reinicia ecuaciones
		\addtocounter{templateEquations}{\value{equation}}
		\setcounter{equation}{0}
	}{}
	\ifthenelse{\equal{\equationrestart}{ssec}}{ % Reinicia ecuaciones
		\addtocounter{templateEquations}{\value{equation}}
		\setcounter{equation}{0}
	}{}
	\ifthenelse{\equal{\showsectioncaptionfig}{ssec}}{ % Reinicia figuras
		\addtocounter{templateFigures}{\value{figure}}
		\setcounter{figure}{0}
	}{}
	\ifthenelse{\equal{\showsectioncaptiontab}{ssec}}{ % Reinicia tablas
		\addtocounter{templateTables}{\value{table}}
		\setcounter{table}{0}
	}{}
	\def\GLOBALsubsectionanumenabled{false}
	\coreintializetitlenumbering
}{}{}

% Comienza nueva subsubsección, aquí hay varios casos:
%	- si está dentro de una subsección sin número ignora la sección
%	- si no, entonces puede estar dentro de una sección sin número o no, en ese caso
%     debe evaluar ambas posibilidades
\pretocmd{\subsubsection}{
	\ifthenelse{\equal{\showsectioncaptioncode}{sssec}}{ % Reinicia código fuente
		\addtocounter{templateListings}{\value{lstlisting}}
		\setcounter{lstlisting}{0}
	}{}
	\ifthenelse{\equal{\showsectioncaptioneqn}{sssec}}{ % Reinicia ecuaciones
		\addtocounter{templateEquations}{\value{equation}}
		\setcounter{equation}{0}
	}{}
	\ifthenelse{\equal{\equationrestart}{sssec}}{ % Reinicia ecuaciones
		\addtocounter{templateEquations}{\value{equation}}
		\setcounter{equation}{0}
	}{}
	\ifthenelse{\equal{\showsectioncaptionfig}{sssec}}{ % Reinicia figuras
		\addtocounter{templateFigures}{\value{figure}}
		\setcounter{figure}{0}
	}{}
	\ifthenelse{\equal{\showsectioncaptiontab}{sssec}}{ % Reinicia tablas
		\addtocounter{templateTables}{\value{table}}
		\setcounter{table}{0}
	}{}
	\coreintializetitlenumbering
}{}{}

\pretocmd{\subsubsubsection}{
	\ifthenelse{\equal{\showsectioncaptioncode}{ssssec}}{ % Reinicia código fuente
		\addtocounter{templateListings}{\value{lstlisting}}
		\setcounter{lstlisting}{0}
	}{}
	\ifthenelse{\equal{\showsectioncaptioneqn}{ssssec}}{ % Reinicia ecuaciones
		\addtocounter{templateEquations}{\value{equation}}
		\setcounter{equation}{0}
	}{}
	\ifthenelse{\equal{\equationrestart}{ssssec}}{ % Reinicia ecuaciones
		\addtocounter{templateEquations}{\value{equation}}
		\setcounter{equation}{0}
	}{}
	\ifthenelse{\equal{\showsectioncaptionfig}{ssssec}}{ % Reinicia figuras
		\addtocounter{templateFigures}{\value{figure}}
		\setcounter{figure}{0}
	}{}
	\ifthenelse{\equal{\showsectioncaptiontab}{ssssec}}{ % Reinicia tablas
		\addtocounter{templateTables}{\value{table}}
		\setcounter{table}{0}
	}{}
}{}{}

% Insertar un título sin número
% 	#1	Título
\newcommand{\sectionanum}[1]{
	\emptyvarerr{\sectionanum}{#1}{Titulo no definido}
	\phantomsection
	\needspace{3\baselineskip}
	\section*{#1}
	\addcontentsline{toc}{section}{#1}
	\ifthenelse{\equal{\anumsecaddtocounter}{true}}{\stepcounter{section}}{}
	\changeheadertitle{#1}
	\setcounter{subsection}{0}
	\renewcommand{\thesubsection}{\arabic{subsection}}
	\def\GLOBALsectionanumenabled{true}
}

% Insertar un título sin número y sin indexar
% 	#1	Título
\newcommand{\sectionanumnoi}[1]{
	\emptyvarerr{\sectionanumnoi}{#1}{Titulo no definido}
	\phantomsection
	\needspace{3\baselineskip}
	\section*{#1}
	\ifthenelse{\equal{\anumsecaddtocounter}{true}}{\stepcounter{section}}{}
	\changeheadertitle{#1}
	\setcounter{subsection}{0}
	\renewcommand{\thesubsection}{\arabic{subsection}}
	\def\GLOBALsectionanumenabled{true}
}

% Insertar un título sin número sin cambiar el título del header
% 	#1	Título
\newcommand{\sectionanumheadless}[1]{
	\emptyvarerr{\sectionanumnoheadless}{#1}{Titulo no definido}
	\section*{#1}
	\addcontentsline{toc}{section}{#1}
	\ifthenelse{\equal{\anumsecaddtocounter}{true}}{\stepcounter{section}}{}
	\setcounter{subsection}{0}
	\renewcommand{\thesubsection}{\arabic{subsection}}
	\def\GLOBALsectionanumenabled{true}
}

% Insertar un título sin número, sin indexar y sin cambiar el título del header
% 	#1	Título
\newcommand{\sectionanumnoiheadless}[1]{
	\emptyvarerr{\sectionanumnoiheadless}{#1}{Titulo no definido}
	\section*{#1}
	\ifthenelse{\equal{\anumsecaddtocounter}{true}}{\stepcounter{section}}{}
	\setcounter{subsection}{0}
	\renewcommand{\thesubsection}{\arabic{subsection}}
	\def\GLOBALsectionanumenabled{true}
}

% Insertar un subtítulo sin número
% 	#1	Subtítulo
\newcommand{\subsectionanum}[1]{
	\emptyvarerr{\subsectionanum}{#1}{Subtitulo no definido}
	\subsection*{#1}
	\addcontentsline{toc}{subsection}{#1}
	\ifthenelse{\equal{\anumsecaddtocounter}{true}}{\stepcounter{subsection}}{}
	\setcounter{subsubsection}{0}
	\renewcommand{\thesubsubsection}{\arabic{subsubsection}}
	\def\GLOBALsubsectionanumenabled{true}
}

% Insertar un subtítulo sin número y sin indexar
% 	#1	Subtítulo
\newcommand{\subsectionanumnoi}[1]{
	\emptyvarerr{\subsectionanumnoi}{#1}{Subtitulo no definido}
	\subsection*{#1}
	\ifthenelse{\equal{\anumsecaddtocounter}{true}}{\stepcounter{subsection}}{}
	\setcounter{subsubsection}{0}
	\renewcommand{\thesubsubsection}{\arabic{subsubsection}}
	\def\GLOBALsubsectionanumenabled{true}
}

% Insertar un sub-subtítulo sin número
% 	#1	Sub-subtítulo
\newcommand{\subsubsectionanum}[1]{
	\emptyvarerr{\subsubsectionanum}{#1}{Sub-subtitulo no definido}
	\subsubsection*{#1}
	\addcontentsline{toc}{subsubsection}{#1}
	\ifthenelse{\equal{\anumsecaddtocounter}{true}}{\stepcounter{subsubsection}}{}
	\setcounter{subsubsubsection}{0}
	\ifthenelse{\equal{\showdotaftersnum}{true}}{
		\renewcommand{\thesubsubsubsection}{\arabic{subsubsubsection}.}
	}{
		\renewcommand{\thesubsubsubsection}{\arabic{subsubsubsection}}
	}
}

% Insertar un sub-subtítulo sin número y sin indexar
% 	#1	Sub-subtítulo
\newcommand{\subsubsectionanumnoi}[1]{
	\emptyvarerr{\subsubsectionanumnoi}{#1}{Sub-subtitulo no definido}
	\subsubsection*{#1}
	\ifthenelse{\equal{\anumsecaddtocounter}{true}}{\stepcounter{subsubsection}}{}
	\setcounter{subsubsubsection}{0}
	\ifthenelse{\equal{\showdotaftersnum}{true}}{
		\renewcommand{\thesubsubsubsection}{\arabic{subsubsubsection}.}
	}{
		\renewcommand{\thesubsubsubsection}{\arabic{subsubsubsection}}
	}
}

% Insertar un sub-sub-subtítulo sin número
% 	#1	Sub-sub-subtítulo
\newcommand{\subsubsubsectionanum}[1]{
	\emptyvarerr{\subsubsubsectionanum}{#1}{Sub-sub-subtitulo no definido}
	\subsubsubsection*{#1}
	\addcontentsline{toc}{subsubsubsection}{#1}
	\ifthenelse{\equal{\anumsecaddtocounter}{true}}{\stepcounter{subsubsubsection}}{}
}

% Insertar un sub-sub-subtítulo sin número y sin indexar
% 	#1	Sub-sub-subtítulo
\newcommand{\subsubsubsectionanumnoi}[1]{
	\emptyvarerr{\subsubsubsectionanumnoi}{#1}{Sub-sub-subtitulo no definido}
	\subsubsection*{#1}
	\ifthenelse{\equal{\anumsecaddtocounter}{true}}{\stepcounter{subsubsubsection}}{}
}

% Cambia el título del encabezado (header)
%	#1	Título
\newcommand{\changeheadertitle}[1]{
	\emptyvarerr{\changeheadertitle}{#1}{Titulo no definido}
	\markboth{#1}{}
}

% Elimina el título del encabezado (header)
\newcommand{\clearheadertitle}{
	\markboth{}{}
}

% Insertar un título en un índice, sin número de página
%	#1	Margen superior en pt. (opcional)
%	#2	Título
\newcommand{\insertindextitle}[2][]{
	\emptyvarerr{\insertindextitle}{#2}{Titulo no definido}
	\ifx\hfuzz#1\hfuzz
		\addtocontents{toc}{\protect\addvspace{\indextitlemargin pt}}
	\else
		\addtocontents{toc}{\protect\addvspace{#1 pt}}
	\fi
	\addtocontents{toc}{\noindent\hyperref[swpn]{\textbf{#2}}}
}

% Insertar un título en un índice, con número de página
%	#1	Margen superior en pt. (opcional)
%	#2	Título
\newcommand{\insertindextitlepage}[2][]{
	\emptyvarerr{\insertindextitlepage}{#2}{Titulo no definido}
	\ifx\hfuzz#1\hfuzz
		\addtocontents{toc}{\protect\addvspace{\indextitlemargin pt}}
	\else
		\addtocontents{toc}{\protect\addvspace{#1 pt}}
	\fi
	\addcontentsline{toc}{section}{#2}
}

% Crea una sección en el índice y en el header
%	#1	Margen superior en pt. (opcional)
%	#2	Título
\newcommand{\createhiddensection}[2][]{
	\changeheadertitle{#2}
	\insertindextitlepage[#1]{#2}
}

% Crear un capítulo como una sección
%	#1	Título
\newcommand{\newchapter}[1]{
	\emptyvarerr{\newchapter}{#1}{Titulo no definido}
	\newpage
	\stepcounter{section}
	\phantomsection
	\needspace{3\baselineskip}
	\vspace* {3cm}
	\noindent {\huge{\textbf{\nomchapter\ \thesection}}} \\
	\vspace* {0.5cm} \\
	\noindent {\Huge{\textbf{#1}}} \\
	\vspace {0.5cm} \\
	\addcontentsline{toc}{section}{\protect\numberline{\thesection}#1}
	\markboth{#1}{}
}
% Insertar párrafo
\newcommand{\newp}{\hbadness=10000 \vspace{\baselineskip} \par}

% Insertar párrafo
% 	#1	Párrafo
\newcommand{\newpar}[1]{\hbadness=10000 #1 \newp}

% Insertar párrafo sin nueva línea al final
% 	#1	Párrafo
\newcommand{\newparnl}[1]{#1 \par}

% Crea un salto de columna en el entorno multicol
\newcommand{\newcolumn}{\checkinsidemulticol\vfill\null\columnbreak}

% Inserta un lipsum corregido
% \newcommand{\slipsum}[1][]{~\lipsum[#1]}

% Redimensiona un ítem
% 	#1	Tamaño del nuevo objeto (En textwidth)
%	#2	Objeto a redimensionar
\newcommand{\itemresize}[2]{
	\emptyvarerr{\itemresize}{#1}{Tamano del nuevo objeto no definido}
	\emptyvarerr{\itemresize}{#2}{Objeto a redimensionar no definido}
	\resizebox{#1\textwidth}{!}{#2}
}

% Crea una página vacía sin header o footer
\newcommand{\insertemptypage}{
	\newpage
	\setcounter{templatePageCounter}{\thepage}
	\pagenumbering{gobble}
	\null
	\thispagestyle{empty}
	\newpage
	\pagenumbering{arabic}
	\setcounter{page}{\thetemplatePageCounter}
}

% Inserta una página vacía, aunque conserva header, footer y numeración
\newcommand{\insertblankpage}{
	\newpage
	\null
	\newpage
}

% Añade un archivo pdf con el header
%	#1	Parámetros (opcional)
%	#2	Nombre del archivo pdf
\newcommand{\includehfpdf}[2][]{
	\includepdf[pagecommand={\pagestyle{fancy}},#1]{#2}
}

% Añade un archivo pdf con el header
%	#1	Parámetros (opcional)
%	#2	Nombre del archivo pdf
\newcommand{\includefullhfpdf}[2][]{
	\includepdf[pages=-,pagecommand={\pagestyle{fancy}},#1]{#2}
}

% Inserta un texto entre comillas
%	#1 	Texto
\newcommand{\quotes}[1]{\enquote*{#1}}

% Inserta una cita con texto elevado
%	#1	Cita
\newcommand{\scite}[1]{\textsuperscript{\cite{#1}}}

% Inserta un texto con el formato de enlace
% 	#1 	Enlace
\newcommand{\hreftext}[1]{\ifthenelse{\equal{\fonturl}{same}}{#1}{\ifthenelse{\equal{\fonturl}{tt}}{\texttt{#1}}{\ifthenelse{\equal{\fonturl}{rm}}{\textrm{#1}}{\ifthenelse{\equal{\fonturl}{sf}}{\textsf{#1}}{}}}}}

% Parcha href para incluir el estilo
% \let\oldhref=\href
% \renewcommand{\href}[2]{
% 	\oldhref{#1}{\insertHREFtext{#2}}
% }

% Inserta un email con un link cliqueable
%	#1 	Dirección email
\newcommand{\insertemail}[1]{\href{mailto:#1}{\hreftext{#1}}}

% Inserta un teléfono celular
%	#1	Teléfono celular
\newcommand{\insertphone}[1]{\href{tel:#1}{\hreftext{#1}}}

% Reinicia el número de ecuaciones
\newcommand{\restartequation}{\setcounter{equation}{0}}

% Desactiva el margen de las leyendas
\newcommand{\disablecaptionmargin}{\setcaptionmargincm{0}}

% Reinicia el margen de las leyendas
\newcommand{\resetcaptionmargin}{\setcaptionmargincm{\captionlrmargin}}

% Modifica el color de las tablas
%	#1	Posición inicial del inicio de colores
\newcommand{\settablerowcolors}[1]{
	\emptyvarerr{\settablerowcolors}{#1}{Posicion de fila no definida}
	
	% Usa colores normales
	\ifthenelse{\equal{\GLOBALtablerowcolorswitch}{false}}{
		\ifthenelse{\equal{\tablerowfirstcolor}{none}}{
			\ifthenelse{\equal{\tablerowsecondcolor}{none}}{
				\rowcolors{#1}{}{}
			}{
				\rowcolors{#1}{\tablerowsecondcolor}{}
			}
		}{
			\ifthenelse{\equal{\tablerowsecondcolor}{none}}{
				\rowcolors{#1}{}{\tablerowfirstcolor}
			}{
				\rowcolors{#1}{\tablerowsecondcolor}{\tablerowfirstcolor}
			}
		}
	% Usa colores alternados
	}{
		\ifthenelse{\equal{\tablerowfirstcolor}{none}}{
			\ifthenelse{\equal{\tablerowsecondcolor}{none}}{
				\rowcolors{#1}{}{}
			}{
				\rowcolors{#1}{}{\tablerowsecondcolor}
			}
		}{
			\ifthenelse{\equal{\tablerowsecondcolor}{none}}{
				\rowcolors{#1}{\tablerowfirstcolor}{}
			}{
				\rowcolors{#1}{\tablerowfirstcolor}{\tablerowsecondcolor}
			}
		}
	}

	% Actualiza el índice previo
	\def\GLOBALtablerowcolorindex{#1}
}

% Alterna los colores de las tablas a la última ejecución
\newcommand{\settablerowcolorslast}{
	% Usa colores normales
	\ifthenelse{\equal{\GLOBALtablerowcolorswitch}{false}}{
		\ifthenelse{\equal{\tablerowfirstcolor}{none}}{
			\ifthenelse{\equal{\tablerowsecondcolor}{none}}{
				\rowcolors{\GLOBALtablerowcolorindex}{}{}
			}{
				\rowcolors{\GLOBALtablerowcolorindex}{\tablerowsecondcolor}{}
			}
		}{
			\ifthenelse{\equal{\tablerowsecondcolor}{none}}{
				\rowcolors{\GLOBALtablerowcolorindex}{}{\tablerowfirstcolor}
			}{
				\rowcolors{\GLOBALtablerowcolorindex}{\tablerowsecondcolor}{\tablerowfirstcolor}
			}
		}
	% Usa colores alternados
	}{
		\ifthenelse{\equal{\tablerowfirstcolor}{none}}{
			\ifthenelse{\equal{\tablerowsecondcolor}{none}}{
				\rowcolors{\GLOBALtablerowcolorindex}{}{}
			}{
				\rowcolors{\GLOBALtablerowcolorindex}{}{\tablerowsecondcolor}
			}
		}{
			\ifthenelse{\equal{\tablerowsecondcolor}{none}}{
				\rowcolors{\GLOBALtablerowcolorindex}{\tablerowfirstcolor}{}
			}{
				\rowcolors{\GLOBALtablerowcolorindex}{\tablerowfirstcolor}{\tablerowsecondcolor}
			}
		}
	}
}

% Activa el color de las filas de las tablas
%	#1	Posición inicial del inicio de colores
\newcommand{\enabletablerowcolor}[1][]{
	\ifx\hfuzz#1\hfuzz
		\settablerowcolors{2}
	\else
		\settablerowcolors{#1}
	\fi
}

% Desactiva el color de las filas de las tablas
\newcommand{\disabletablerowcolor}{\rowcolors{2}{}{}}

% Alterna los colores de las filas de las tablas
\newcommand{\switchtablerowcolors}{
	\ifthenelse{\equal{\GLOBALtablerowcolorswitch}{false}}{
		\def\GLOBALtablerowcolorswitch{true}
	}{
		\def\GLOBALtablerowcolorswitch{false}
	}
	\settablerowcolorslast
}

% Cambia el tamaño de la página
%	#1	Orientacion de la página, puede ser 0 o 90. Por defecto es cero
%	#2	Ancho de la página (cm)
%	#3	Alto de la página (cm)
\newcommand{\changepagesize}[3][]{
	% \emptyvarerr{\changepagesize}{#1}{Orientacion de la pagina}
	\emptyvarerr{\changepagesize}{#2}{Ancho de la pagina no definida}
	\emptyvarerr{\changepagesize}{#3}{Altura de la pagina no definida}
	\ifthenelse{\equal{\compilertype}{lualatex}}{
		\throwwarning{Funcion no valida en compilador lualatex}
	}{
		\newpage
		\ifthenelse{\equal{#1}{}}{
			\newgeometry{left=\pagemarginleft cm, top=\pagemargintop cm, right=\pagemarginright cm, bottom=\pagemarginbottom cm, paperwidth=#2 cm, paperheight=#3 cm}
		}{
		\ifthenelse{\equal{#1}{0}}{
			\newgeometry{left=\pagemarginleft cm, top=\pagemargintop cm, right=\pagemarginright cm, bottom=\pagemarginbottom cm, paperwidth=#2 cm, paperheight=#3 cm}
		}{
		\ifthenelse{\equal{#1}{90}}{
			\newgeometry{left=\pagemarginleft cm, top=\pagemargintop cm, right=\pagemarginright cm, bottom=\pagemarginbottom cm, paperwidth=#3 cm, paperheight=#2 cm}
		}{
			\throwbadconfig{Orientacion de pagina no valido}{\changepagesize}{0,90}}}
		}
	}
}

% Ofrece diferentes formatos de pagina
% https://www.prepressure.com/library/paper-size
%	#1	Indica la rotación, puede ser 0 o 90
%	#2	Formato de la pagina
\newcommand{\changepagesizeformat}[2][]{
	\emptyvarerr{\changepagesizeformat}{#2}{Formato de pagina no definido}
	\ifthenelse{\equal{#2}{4A0}}{
		\changepagesize[#1]{168.2}{237.8}
	}{
	\ifthenelse{\equal{#2}{2A0}}{
		\changepagesize[#1]{118.9}{168.2}
	}{
	\ifthenelse{\equal{#2}{A0}}{
		\changepagesize[#1]{84.1}{118.9}
	}{
	\ifthenelse{\equal{#2}{A1}}{
		\changepagesize[#1]{59.4}{84.1}
	}{
	\ifthenelse{\equal{#2}{A2}}{
		\changepagesize[#1]{42.0}{84.1}
	}{
	\ifthenelse{\equal{#2}{A3}}{
		\changepagesize[#1]{29.7}{42.0}
	}{
	\ifthenelse{\equal{#2}{A4}}{
		\changepagesize[#1]{21.0}{29.7}
	}{
	\ifthenelse{\equal{#2}{A5}}{
		\changepagesize[#1]{14.8}{21.0}
	}{
	\ifthenelse{\equal{#2}{A6}}{
		\changepagesize[#1]{10.5}{14.8}
	}{
	\ifthenelse{\equal{#2}{letter}}{
		\changepagesize[#1]{21.59}{27.94}
	}{
	\ifthenelse{\equal{#2}{legal}}{
		\changepagesize[#1]{21.59}{35.6}
	}{
	\ifthenelse{\equal{#2}{foolscap}}{
		\changepagesize[#1]{20.3}{33.0}
	}{
	\ifthenelse{\equal{#2}{executive}}{
		\changepagesize[#1]{18.41}{26.67}
	}{
	\ifthenelse{\equal{#2}{ledger}}{
		\changepagesize[#1]{27.94}{43.18}
	}{
	\ifthenelse{\equal{#2}{tabloid}}{
		\changepagesize[#1]{43.18}{27.94}
	}{
	\ifthenelse{\equal{#2}{ANSIC}}{
		\changepagesize[#1]{55.9}{43.2}
	}{
	\ifthenelse{\equal{#2}{ANSID}}{
		\changepagesize[#1]{86.4}{55.9}
	}{
	\ifthenelse{\equal{#2}{ANSIE}}{
		\changepagesize[#1]{111.8}{86.4}
	}{
	\ifthenelse{\equal{#2}{B0}}{
		\changepagesize[#1]{100}{141.4}
	}{
	\ifthenelse{\equal{#2}{B1}}{
		\changepagesize[#1]{70.7}{100}
	}{
	\ifthenelse{\equal{#2}{B2}}{
		\changepagesize[#1]{50}{70.7}
	}{
	\ifthenelse{\equal{#2}{B3}}{
		\changepagesize[#1]{35.3}{50}
	}{
	\ifthenelse{\equal{#2}{B4}}{
		\changepagesize[#1]{25}{35.3}
	}{
	\ifthenelse{\equal{#2}{B5}}{
		\changepagesize[#1]{17.6}{25}
	}{
	\ifthenelse{\equal{#2}{B6}}{
		\changepagesize[#1]{12.5}{17.6}
	}{
		\throwbadconfig{Estilo de pagina no valido}{\changepagesizeformat}{4A0,2A0,A0,A1,A2,A3,A4,A5,A6,letter,legal,foolscap,executive,ledger,tabloid,ANSIC,ANSID,ANSIE,B0,B1,B2,B3,B4,B5,B6}}}}}}}}}}}}}}}}}}}}}}}}}
	}
}
% Crea una sección de referencias solo para bibtex
\newenvironment{references}{
	\ifthenelse{\equal{\stylecitereferences}{bibtex}}{ % Verifica configuraciones
	}{
		\throwerror{\references}{Solo se puede usar entorno references con estilo citas \noexpand\stylecitereferences=bibtex}
	}
	\begingroup
	\ifthenelse{\equal{\sectionrefenv}{true}}{ % Se configura las referencias como una sección
		\section{\namereferences}
	}{
		\sectionanum{\namereferences}
	}
	\renewcommand{\section}[2]{}
	\begin{thebibliography}{} % Inicia la bibliografía
		\ifthenelse{\equal{\bibtextextalign}{justify}}{ % Formato ajuste de línea
		}{
		\ifthenelse{\equal{\bibtextextalign}{left}}{
			\raggedright
		}{
		\ifthenelse{\equal{\bibtextextalign}{right}}{
			\raggedleft
		}{
		\ifthenelse{\equal{\bibtextextalign}{center}}{
			\centering
		}{
			\throwbadconfig{Ajuste de linea referencias bibtex desconocido}{\bibtextextalign}{justified,left,right,center}}}}
		}
	}
	{
	\end{thebibliography}
	\endgroup % Termina el grupo
}

% Crea una sección de anexos
\newenvironment{anexo}{
	\begingroup % Inicia el grupo en nueva página y sección
	\clearpage
	\phantomsection
	\changeheadertitle{\nomltappendixsection} % Cambia el nombre del header
	\def\GLOBALsectionalph{true} % Modifica formato de secciones con sección alph
	\bookmarksetup{
		numbered={true},
		openlevel={\thetemplateBookmarksLevelPrev}
	}
	\appendixtitleon
	\appendixtitletocon
	%\addappheadtotoc
	\bookmarksetupnext{level=part}
	\begin{appendices} % Crea la sección
		\ifthenelse{\equal{\showappendixsecindex}{true}}{}{
			\pdfbookmark{\nameappendixsection}{contents} % Si false
		}
		% \setcounter{secnumdepth}{4}
		% \setcounter{tocdepth}{4}
		\ifthenelse{\equal{\appendixindepobjnum}{true}}{
			\counterwithin{equation}{section}
			\counterwithin{figure}{section}
			\counterwithin{lstlisting}{section}
			\counterwithin{table}{section}}{
		}
	}{
	\end{appendices}
	\def\GLOBALsectionalph{false} % Desactiva formato de secciones con sección alph
	\bookmarksetupnext{level={\thetemplateBookmarksLevelPrev}} % Reestablece índice marcador
	\bookmarksetup{
		numbered={\cfgpdfsecnumbookmarks},
		openlevel={\cfgbookmarksopenlevel}
	}
	\endgroup
}

% Inicia código fuente con parámetros
%	#1	Label (opcional)
%	#2	Estilo de código
%	#3	Parámetros
%	#4	Caption
\newcommand{\coreinitsourcecodep}[4]{
	\emptyvarerr{\coreinitsourcecodep}{#2}{Estilo de codigo no definido}
	\checkvalidsourcecodestyle{#2}
	\ifthenelse{\equal{\showlinenumbers}{true}}{
		\rightlinenumbers}{
	}
	\lstset{
		backgroundcolor=\color{\sourcecodebgcolor}
	}
	\ifthenelse{\equal{\codecaptiontop}{true}}{
		\ifx\hfuzz#4\hfuzz
			\ifx\hfuzz#3\hfuzz
				\lstset{
					style=#2
				}
			\else
				\lstset{
					style=#2,
					#3
				}
			\fi
		\else
			\ifx\hfuzz#3\hfuzz
				\lstset{
					caption={#4 #1},
					captionpos=t,
					style=#2
				}
			\else
				\lstset{
					caption={#4 #1},
					captionpos=t,
					style=#2,
					#3
				}
			\fi
		\fi
	}{
		\ifx\hfuzz#4\hfuzz
			\ifx\hfuzz#3\hfuzz
				\lstset{
					style=#2
				}
			\else
				\lstset{
					style=#2,
					#3
				}
			\fi
		\else
			\ifx\hfuzz#3\hfuzz
				\lstset{
					caption={#4 #1},
					captionpos=b,
					style=#2
				}
			\else
				\lstset{
					caption={#4 #1},
					captionpos=b,
					style=#2,
					#3
				}
			\fi
		\fi	
	}
}

% Inserta código fuente con parámetros
%	#1	Label (opcional)
%	#2	Estilo de código
%	#3	Parámetros
%	#4	Caption
\lstnewenvironment{sourcecodep}[4][]{
	\coreinitsourcecodep{#1}{#2}{#3}{#4}
}{
	\ifthenelse{\equal{\showlinenumbers}{true}}{
		\leftlinenumbers}{
	}
}

% Importa código fuente desde un archivo con parámetros
%	#1	Label (opcional)
%	#2	Estilo de código
%	#3	Parámetros
%	#4	Archivo de código fuente
%	#5	Caption
\newcommand{\importsourcecodep}[5][]{
	\coreinitsourcecodep{#1}{#2}{#3}{#5}
	\inputlisting{#4}
	\ifthenelse{\equal{\showlinenumbers}{true}}{
		\leftlinenumbers}{
	}
}

% Inicia código fuente sin parámetros
%	#1	Label (opcional)
%	#2	Estilo de código
%	#3	Caption
\newcommand{\coreinitsourcecode}[3]{
	\emptyvarerr{\coreinitsourcecode}{#2}{Estilo de codigo no definido}
	\checkvalidsourcecodestyle{#2}
	\ifthenelse{\equal{\showlinenumbers}{true}}{
		\rightlinenumbers}{
	}
	\lstset{
		backgroundcolor=\color{\sourcecodebgcolor}
	}
	\ifthenelse{\equal{\codecaptiontop}{true}}{
		\ifx\hfuzz#3\hfuzz
			\lstset{
				style=#2
			}
		\else
			\lstset{
				style=#2,
				caption={#3 #1},
				captionpos=t
			}
		\fi
	}{
		\ifx\hfuzz#3\hfuzz
			\lstset{
				style=#2
			}
		\else
			\lstset{
				style=#2,
				caption={#3 #1},
				captionpos=b
			}
		\fi
	}
}

% Inserta código fuente sin parámetros
%	#1	Label (opcional)
%	#2	Estilo de código
%	#3	Caption
\lstnewenvironment{sourcecode}[3][]{
	\coreinitsourcecode{#1}{#2}{#3}
}{
	\ifthenelse{\equal{\showlinenumbers}{true}}{
		\leftlinenumbers}{
	}
}

% Importa código fuente desde un archivo sin parámetros
%	#1	Label (opcional)
%	#2	Estilo de código
%	#3	Archivo de código fuente
%	#4	Caption
\newcommand{\importsourcecode}[4][]{
	\coreinitsourcecode{#1}{#2}{#4}
	\lstinputlisting{#3}
	\ifthenelse{\equal{\showlinenumbers}{true}}{
		\leftlinenumbers}{
	}
}

% Itemize en negrita
%	#1	Parámetros opcionales
\newenvironment{itemizebf}[1][]{
	\begin{itemize}[font=\bfseries,#1]
	}{
	\end{itemize}
}

% Enumerate en negrita
%	#1	Parámetros opcionales
\newenvironment{enumeratebf}[1][]{
	\begin{enumerate}[font=\bfseries,#1]
	}{
	\end{enumerate}
}

% Crea una sección de resumen
\newenvironment{resumen}{
	% Tipo de título para abstract
	\sectionfont{\color{\titlecolor} \fontsizetitle \styletitle \selectfont}
	% Inserta un título sin número, sin cabecera y sin aparecer en el índice,
	% para que aparezca en el índice utilizar la función \sectionanumheadless
	\sectionanumnoiheadless{\nameabstract}}{
}

% Crea una sección de imágenes múltiples
%	#1	Label (opcional)
%	#2	Caption
\newenvironment{images}[2][]{
	% Modifica globales
	\def\envimageslabelvar {#1}
	\def\envimagescaptionvar {#2}
	\def\GLOBALenvimageinitialized {true}
	\def\GLOBALenvimageadded {false}
	
	% Configura caption y márgenes
	\vspace{\marginimagetop cm}
	\setcaptionmargincm{\captionmarginmultimg}
	
	% Inicia la figura
	\begin{figure}[H] \centering%
		\thisfloatpagestyle{fancy}%
		\vspace{\marginimagemulttop cm}%
		}{%
		\setcaptionmargincm{\captionlrmargin}%
		\ifthenelse{\equal{\envimagescaptionvar}{}}{ % \ifx\hfuzz no sirve
			\vspace{\captionlessmarginimage cm}%
		}{%
			\caption{\envimagescaptionvar\envimageslabelvar}%
		}%
	\end{figure}%

	% Reestablece caption y márgenes
	\setcaptionmargincm{\captionlrmargin}
	\vspace{\marginimagebottom cm}
	
	% Reestablece globales
	\def\GLOBALenvimageinitialized {false}
}

% Crea una sección de imágenes múltiples completa dentro de un multicol
%	#1	Label (opcional)
%	#2	Posición de la imagen, "bottom" o "top"
%	#3	Caption
\newenvironment{imagesmc}[3][]{
	% Modifica globales
	\def\envimageslabelvar {#1}
	\def\envimagescaptionvar {#3}
	\def\GLOBALenvimageinitialized {true}
	\def\GLOBALenvimageadded {false}
	\checkinsidemulticol
	
	% Configura caption y márgenes
	\setcaptionmargincm{\captionmarginmultimg} % Eso es para los wrapfig
	
	% Inicia la figura
	\ifthenelse{\equal{#2}{bottom}}{%
		\begin{figure*}[b!] \centering%
		}{%
	\ifthenelse{\equal{#2}{top}}{%
		\begin{figure*}[t!] \centering%
	}{%
		\errmessage{LaTeX Warning: Posicion de imagen invalida, valores esperados: bottom,top}
	}}%
		\thisfloatpagestyle{fancy}%
		\vspace{\marginimagemulttop cm}%
	}{%
		\setcaptionmargincm{\captionlrmargin}%
		\ifthenelse{\equal{\envimagescaptionvar}{}}{ % \ifx\hfuzz no sirve
			\vspace{\captionlessmarginimage cm}%
		}{%
			\caption{\envimagescaptionvar\envimageslabelvar}%
		}%
	\end{figure*}%
	
	% Reestablece caption y márgenes
	\setcaptionmargincm{\captionlrmargin}
	
	% Reestablece globales
	\def\GLOBALenvimageinitialized {false}
}

% IMPORTACIÓN DE ESTILOS
\colorlet{numb}{magenta!60!black}
\colorlet{punct}{red!60!black}
\definecolor{delim}{RGB}{20,105,176}
\definecolor{dkcyan}{RGB}{0,123,167}
\definecolor{dkgray}{RGB}{90,90,90}
\definecolor{dkgreen}{RGB}{0,150,0}
\definecolor{dkyellow}{RGB}{179,179,36}
\definecolor{gray}{RGB}{127,127,127}
\definecolor{lbrown}{RGB}{255,252,249}
\definecolor{lgray}{RGB}{240,240,240}
\definecolor{lyellow}{RGB}{255,255,204}
\definecolor{mauve}{RGB}{150,0,210}
\definecolor{ocre}{RGB}{243,102,25}
% -----------------------------------------------------------------------------
% Assembler
% -----------------------------------------------------------------------------
\lstdefinelanguage[x64]{Assembler}[x86masm]{Assembler}{
	morekeywords={
		CDQE,CQO,CMPSQ,CMPXCHG16B,JRCXZ,LODSQ,MOVSXD,POPFQ,PUSHFQ,SCASQ,STOSQ,IRETQ,RDTSCP,SWAPGS,rax,rdx,rcx,rbx,rsi,rdi,rsp,rbp,r8,r8d,r8w,r8b,r9,r9d,r9w,r9b,r10,r10d,r10w,r10b,r11,r11d,r11w,r11b,r12,r12d,r12w,r12b,r13,r13d,r13w,r13b,r14,r14d,r14w,r14b,r15,r15d,r15w,r15b
	}
}
\lstdefinestyle{assemblerx64}{
	language=[x64]Assembler
}
\lstdefinestyle{assemblerx86}{
	language=[x86masm]Assembler
}

% -----------------------------------------------------------------------------
% Bash
% -----------------------------------------------------------------------------
\lstdefinestyle{bash}{
	language=bash,
	breakatwhitespace=false,
	morecomment=[l]{rem},
	morecomment=[s]{::}{::},
	morekeywords={
		call,cp,dig,gcc,git,grep,ls,mv,python,rm,sudo,vim
	},
	sensitive=false
}

% -----------------------------------------------------------------------------
% C
% -----------------------------------------------------------------------------
\lstdefinestyle{c}{
	language=C,
	breakatwhitespace=false,
	keepspaces=true
}

% -----------------------------------------------------------------------------
% C++
% -----------------------------------------------------------------------------
\lstdefinestyle{cpp}{
	language=C++,
	breakatwhitespace=false,
	morecomment=[l][\color{magenta}]{\#}
}

% -----------------------------------------------------------------------------
% C#
% -----------------------------------------------------------------------------
\lstdefinestyle{csharp}{
	language=csh,
	morecomment=[l]{//},
	morecomment=[s]{/*}{*/},
	morekeywords={
		abstract,as,base,bool,break,byte,case,catch,char,checked,class,const,continue,decimal,default,delegate,do,double,else,enum,event,explicit,extern,false,finally,fixed,float,for,foreach,goto,if,implicit,in,int,interface,internal,is,lock,long,namespace,new,null,object,operator,out,override,params,private,protected,public,readonly,ref,return,sbyte,sealed,short,sizeof,stackalloc,static,string,struct,switch,this,throw,true,try,typeof,uint,ulong,unchecked,unsafe,ushort,using,virtual,void,volatile,while
	}
}

% -----------------------------------------------------------------------------
% CSS
% -----------------------------------------------------------------------------
\lstdefinelanguage{CSS}{
	morecomment=[s]{/*}{*/},
	morekeywords={
		-moz-binding,-moz-border-bottom-colors,-moz-border-left-colors,-moz-border-radius,-moz-border-radius-bottomleft,-moz-border-radius-bottomright,-moz-border-radius-topleft,-moz-border-radius-topright,-moz-border-right-colors,-moz-border-top-colors,-moz-opacity,-moz-outline,-moz-outline-color,-moz-outline-style,-moz-outline-width,-moz-user-focus,-moz-user-input,-moz-user-modify,-moz-user-select,-replace,-set-link-source,-use-link-source,accelerator,azimuth,background,background-attachment,background-color,background-image,background-position,background-position-x,background-position-y,background-repeat,behavior,border,border-bottom,border-bottom-color,border-bottom-style,border-bottom-width,border-collapse,border-color,border-left,border-left-color,border-left-style,border-left-width,border-right,border-right-color,border-right-style,border-right-width,border-spacing,border-style,border-top,border-top-color,border-top-style,border-top-width,border-width,bottom,caption-side,clear,clip,color,content,counter-increment,counter-reset,cue,cue-after,cue-before,cursor,direction,display,elevation,empty-cells,filter,float,font,font-family,font-size,font-size-adjust,font-stretch,font-style,font-variant,font-weight,height,ime-mode,include-source,layer-background-color,layer-background-image,layout-flow,layout-grid,layout-grid-char,layout-grid-char-spacing,layout-grid-line,layout-grid-mode,layout-grid-type,left,letter-spacing,line-break,line-height,list-style,list-style-image,list-style-position,list-style-type,margin,margin-bottom,margin-left,margin-right,margin-top,marker-offset,marks,max-height,max-width,min-height,min-width,orphans,outline,outline-color,outline-style,outline-width,overflow,overflow-X,overflow-Y,padding,padding-bottom,padding-left,padding-right,padding-top,page,page-break-after,page-break-before,page-break-inside,pause,pause-after,pause-before,pitch,pitch-range,play-during,position,quotes,richness,right,ruby-align,ruby-overhang,ruby-position,scrollbar-3d-light-color,scrollbar-arrow-color,scrollbar-base-color,scrollbar-dark-shadow-color,scrollbar-face-color,scrollbar-highlight-color,scrollbar-shadow-color,scrollbar-track-color,size,speak,speak-header,speak-numeral,speak-punctuation,speech-rate,stress,table-layout,text-align,text-align-last,text-autospace,text-decoration,text-indent,text-justify,text-kashida-space,text-overflow,text-shadow,text-transform,text-underline-position,top,unicode-bidi,vertical-align,visibility,voice-family,volume,white-space,widows,width,word-break,word-spacing,word-wrap,writing-mode,z-index,zoom
	},
	morestring=[s]{:}{;},
	sensitive=true
}
\lstdefinestyle{css}{
	language=CSS,
	breakatwhitespace=true
}

% -----------------------------------------------------------------------------
% CSV, Archivos separados por coma
% -----------------------------------------------------------------------------
\lstdefinestyle{csv}{
	language={}
}

% -----------------------------------------------------------------------------
% CUDA
% -----------------------------------------------------------------------------
\lstdefinestyle{cuda}{
	language=C++,
	breakatwhitespace=false,
	emph={
		cudaFree,cudaMalloc,__device__,__global__,__host__,__shared__,__syncthreads
	},
	emphstyle=\color{dkcyan}\ttfamily,
	morecomment=[l][\color{magenta}]{\#},
	moredelim=[s][\ttfamily]{<<<}{>>>}
}

% -----------------------------------------------------------------------------
% DOCKER
% -----------------------------------------------------------------------------
\lstdefinelanguage{docker}{
	comment=[l]{\#},
	keywords={
		ADD,CMD,COPY,ENTRYPOINT,ENV,EXPOSE,FROM,LABEL,MAINTAINER,ONBUILD,RUN,STOPSIGNAL,USER,VOLUME,WORKDIR
	},
	morestring=[b]',
	morestring=[b]"
}
\lstdefinestyle{docker}{
	language=docker,
	breakatwhitespace=true
}

% -----------------------------------------------------------------------------
% Fortran-95
% -----------------------------------------------------------------------------
\lstdefinestyle{fortran}{
	language=[95]Fortran,
	breakatwhitespace=false
}

% -----------------------------------------------------------------------------
% GLSL - Shaders
% -----------------------------------------------------------------------------
\lstdefinelanguage{GLSL}{
	alsoletter={\#},
	morekeywords=[1]{
		attribute,bool,break,bvec2,bvec3,bvec4,case,centroid,const,continue,default,discard,do,else,false,flat,float,for,highp,if,in,inout,int,invariant,isampler1D,isampler1DArray,isampler2D,isampler2DArray,isampler2DMS,isampler2DMSArray,isampler2DRect,isampler3D,isamplerBuffer,isamplerCube,ivec2,ivec3,ivec4,layout,lowp,mat2,mat2x2,mat2x3,mat2x4,mat3,mat3x2,mat3x3,mat3x4,mat4,mat4x2,mat4x3,mat4x4,mediump,noperspective,out,precision,return,sampler1D,sampler1DArray,sampler1DArrayShadow,sampler1DShadow,sampler2D,sampler2DArray,sampler2DArrayShadow,sampler2DMS,sampler2DMSArray,sampler2DRect,sampler2DRectShadow,sampler2DShadow,sampler3D,samplerBuffer,samplerCube,samplerCubeShadow,smooth,struct,switch,true,uint,uniform,usampler1D,usampler1DArray,usampler2D,usampler2DArray,usampler2DMS,usampler2DMSArray,usampler2DRect,usampler3D,usamplerBuffer,usamplerCube,uvec2,uvec3,uvec4,varying,vec2,vec3,vec4,void,while
	},
	morekeywords=[2]{
		abs,acos,acosh,all,any,asin,asinh,atan,atan,atanh,ceil,clamp,cos,cosh,cross,degrees,determinant,dFdx,dFdy,distance,dot,EmitVertex,EndPrimitive,equal,exp,exp2,faceforward,floatBitsToInt,floatBitsToUint,floor,fract,fwidth,greaterThan,greaterThanEqual,intBitsToFloat,inverse,inversesqrt,isinf,isnan,length,lessThan,lessThanEqual,log,log2,matrixCompMult,max,min,mix,mod,modf,noise1,noise2,noise3,noise4,normalize,not,notEqual,outerProduct,pow,radians,reflect,refract,round,roundEven,shadow1D,shadow1DLod,shadow1DProj,shadow1DProjLod,shadow2D,shadow2DLod,shadow2DProj,shadow2DProjLod,sign,sin,sinh,smoothstep,sqrt,step,tan,tanh,texelFetch,texelFetchOffset,texture,texture1D,texture1DProj,texture1DProjLod,texture2D,texture2DLod,texture2DProj,texture2DProjLod,texture3D,texture3DLod,texture3DProj,texture3DProjLod,textureCube,textureCubeLod,textureGrad,textureGradOffset,textureLod,textureLodOffset,textureOffset,textureProj,textureProjGrad,textureProjGradOffset,textureProjLod,textureProjLodOffset,textureProjOffset,textureSize,transpose,trunc,uintBitsToFloat
	},
	morekeywords=[3]{
		\#version,core,gl_ClipDistance,gl_ClipDistance,gl_ClipVertex,gl_DepthRange,gl_FragColor,gl_FragCoord,gl_FragData,gl_FragDepth,gl_FrontFacing,gl_InstanceID,gl_Layer,gl_MaxClipDistances,gl_MaxCombinedTextureImageUnits,gl_MaxDrawBuffers,gl_MaxDrawBuffers,gl_MaxFragmentInputComponents,gl_MaxFragmentUniformComponents,gl_MaxGeometryInputComponents,gl_MaxGeometryOutputComponents,gl_MaxGeometryOutputVertices,gl_MaxGeometryOutputVertices,gl_MaxGeometryTextureImageUnits,gl_MaxGeometryTotalOutputComponents,gl_MaxGeometryUniformComponents,gl_MaxGeometryVaryingComponents,gl_MaxTextureImageUnits,gl_MaxVaryingComponents,gl_MaxVaryingFloats,gl_MaxVertexAttribs,gl_MaxVertexOutputComponents,gl_MaxVertexTextureImageUnits,gl_MaxVertexUniformComponents,gl_PerVertex,gl_PointCoord,gl_PointSize,gl_Position,gl_PrimitiveID,gl_VertexID
	},
	morecomment=[l]{//},
	morecomment=[s]{/*}{*/}
}
\lstdefinestyle{glsl}{
	language=GLSL,
	keywordstyle=[3]\color{dkcyan}\ttfamily,
	prebreak=\raisebox{0ex}[0ex][0ex]{\ensuremath{\hookleftarrow}},
	sensitive=true,
	upquote=true
}

% -----------------------------------------------------------------------------
% Haskell
% -----------------------------------------------------------------------------
\lstdefinestyle{haskell}{
	language=haskell,
	morecomment=[l]\%
}

% -----------------------------------------------------------------------------
% HTML5
% -----------------------------------------------------------------------------
\lstdefinelanguage{HTML5}{
	language=html,
	alsoletter={<>=-},
	morecomment=[s]{<!--}{-->},
	ndkeywords={
		% General
		=,
		% HTML attributes
		accept=,accept-charset=,accesskey=,action=,align=,alt=,async=,autocomplete=,autofocus=,autoplay=,autosave=,bgcolor=,border=,buffered=,challenge=,charset=,checked=,cite=,class=,code=,codebase=,color=,cols=,colspan=,content=,contenteditable=,contextmenu=,controls=,coords=,data=,datetime=,default=,defer=,dir=,dirname=,disabled=,download=,draggable=,dropzone=,enctype=,for=,form=,formaction=,headers=,height=,hidden=,high=,href=,hreflang=,http-equiv=,icon=,id=,ismap=,itemprop=,keytype=,kind=,label=,lang=,language=,list=,loop=,low=,manifest=,max=,maxlength=,media=,method=,min=,multiple=,name=,novalidate=,open=,optimum=,pattern=,ping=,placeholder=,poster=,preload=,pubdate=,radiogroup=,readonly=,rel=,required=,reversed=,rows=,rowspan=,sandbox=,scope=,scoped=,seamless=,selected=,shape=,size=,sizes=,span=,spellcheck=,src=,srcdoc=,srclang=,start=,step=,style=,summary=,tabindex=,target=,title=,type=,usemap=,value=,width=,wrap=,
		% CSSproperties
		accelerator:,azimuth:,background:,background-attachment:,background-color:,background-image:,background-position:,background-position-x:,background-position-y:,background-repeat:,behavior:,border:,border-bottom:,border-bottom-color:,border-bottom-style:,border-bottom-width:,border-collapse:,border-color:,border-left:,border-left-color:,border-left-style:,border-left-width:,border-right:,border-right-color:,border-right-style:,border-right-width:,border-spacing:,border-style:,border-top:,border-top-color:,border-top-style:,border-top-width:,border-width:,bottom:,caption-side:,clear:,clip:,color:,content:,counter-increment:,counter-reset:,cue:,cue-after:,cue-before:,cursor:,direction:,display:,elevation:,empty-cells:,filter:,float:,font:,font-family:,font-size:,font-size-adjust:,font-stretch:,font-style:,font-variant:,font-weight:,height:,ime-mode:,include-source:,layer-background-color:,layer-background-image:,layout-flow:,layout-grid:,layout-grid-char:,layout-grid-char-spacing:,layout-grid-line:,layout-grid-mode:,layout-grid-type:,left:,letter-spacing:,line-break:,line-height:,list-style:,list-style-image:,list-style-position:,list-style-type:,margin:,margin-bottom:,margin-left:,margin-right:,margin-top:,marker-offset:,marks:,max-height:,max-width:,min-height:,min-width:,transition-duration:,transition-property:,transition-timing-function:,transform:,-moz-transform:,-moz-binding:,-moz-border-radius:,-moz-border-radius-topleft:,-moz-border-radius-topright:,-moz-border-radius-bottomright:,-moz-border-radius-bottomleft:,-moz-border-top-colors:,-moz-border-right-colors:,-moz-border-bottom-colors:,-moz-border-left-colors:,-moz-opacity:,-moz-outline:,-moz-outline-color:,-moz-outline-style:,-moz-outline-width:,-moz-user-focus:,-moz-user-input:,-moz-user-modify:,-moz-user-select:,orphans:,outline:,outline-color:,outline-style:,outline-width:,overflow:,overflow-X:,overflow-Y:,padding:,padding-bottom:,padding-left:,padding-right:,padding-top:,page:,page-break-after:,page-break-before:,page-break-inside:,pause:,pause-after:,pause-before:,pitch:,pitch-range:,play-during:,position:,quotes:,-replace:,richness:,right:,ruby-align:,ruby-overhang:,ruby-position:,-set-link-source:,size:,speak:,speak-header:,speak-numeral:,speak-punctuation:,speech-rate:,stress:,scrollbar-arrow-color:,scrollbar-base-color:,scrollbar-dark-shadow-color:,scrollbar-face-color:,scrollbar-highlight-color:,scrollbar-shadow-color:,scrollbar-3d-light-color:,scrollbar-track-color:,table-layout:,text-align:,text-align-last:,text-decoration:,text-indent:,text-justify:,text-overflow:,text-shadow:,text-transform:,text-autospace:,text-kashida-space:,text-underline-position:,top:,unicode-bidi:,-use-link-source:,vertical-align:,visibility:,voice-family:,volume:,white-space:,widows:,width:,word-break:,word-spacing:,word-wrap:,writing-mode:,z-index:,zoom:
	},
	otherkeywords={
		<,</,>,</a,<a,</a>,</abbr,<abbr,</abbr>,</address,<address,</address>,</area,<area,</area>,</area,<area,</area>,</article,<article,</article>,</aside,<aside,</aside>,</audio,<audio,</audio>,</audio,<audio,</audio>,</b,<b,</b>,</base,<base,</base>,</bdi,<bdi,</bdi>,</bdo,<bdo,</bdo>,</blockquote,<blockquote,</blockquote>,</body,<body,</body>,</br,<br,</br>,</button,<button,</button>,</canvas,<canvas,</canvas>,</caption,<caption,</caption>,</cite,<cite,</cite>,</code,<code,</code>,</col,<col,</col>,</colgroup,<colgroup,</colgroup>,</data,<data,</data>,</datalist,<datalist,</datalist>,</dd,<dd,</dd>,</del,<del,</del>,</details,<details,</details>,</dfn,<dfn,</dfn>,</div,<div,</div>,</dl,<dl,</dl>,</dt,<dt,</dt>,</em,<em,</em>,</embed,<embed,</embed>,</fieldset,<fieldset,</fieldset>,</figcaption,<figcaption,</figcaption>,</figure,<figure,</figure>,</footer,<footer,</footer>,</form,<form,</form>,</h1,<h1,</h1>,</h2,<h2,</h2>,</h3,<h3,</h3>,</h4,<h4,</h4>,</h5,<h5,</h5>,</h6,<h6,</h6>,</head,<head,</head>,</header,<header,</header>,</hr,<hr,</hr>,</html,<html,</html>,</i,<i,</i>,</iframe,<iframe,</iframe>,</img,<img,</img>,</input,<input,</input>,</ins,<ins,</ins>,</kbd,<kbd,</kbd>,</keygen,<keygen,</keygen>,</label,<label,</label>,</legend,<legend,</legend>,</li,<li,</li>,</link,<link,</link>,</main,<main,</main>,</map,<map,</map>,</mark,<mark,</mark>,</math,<math,</math>,</menu,<menu,</menu>,</menuitem,<menuitem,</menuitem>,</meta,<meta,</meta>,</meter,<meter,</meter>,</nav,<nav,</nav>,</noscript,<noscript,</noscript>,</object,<object,</object>,</ol,<ol,</ol>,</optgroup,<optgroup,</optgroup>,</option,<option,</option>,</output,<output,</output>,</p,<p,</p>,</param,<param,</param>,</pre,<pre,</pre>,</progress,<progress,</progress>,</q,<q,</q>,</rp,<rp,</rp>,</rt,<rt,</rt>,</ruby,<ruby,</ruby>,</s,<s,</s>,</samp,<samp,</samp>,</script,<script,</script>,</section,<section,</section>,</select,<select,</select>,</small,<small,</small>,</source,<source,</source>,</span,<span,</span>,</strong,<strong,</strong>,</style,<style,</style>,</summary,<summary,</summary>,</sup,<sup,</sup>,</svg,<svg,</svg>,</table,<table,</table>,</tbody,<tbody,</tbody>,</td,<td,</td>,</template,<template,</template>,</textarea,<textarea,</textarea>,</tfoot,<tfoot,</tfoot>,</th,<th,</th>,</thead,<thead,</thead>,</time,<time,</time>,</title,<title,</title>,</tr,<tr,</tr>,</track,<track,</track>,</u,<u,</u>,</ul,<ul,</ul>,</var,<var,</var>,</video,<video,</video>,</wbr,<wbr,</wbr>,/>,<!
	},
	sensitive=true,
	tag=[s]
}
\lstdefinestyle{html5}{
	language=HTML5,
	alsodigit={.:;},
	alsolanguage=JavaScript,
	firstnumber=1,
	ndkeywordstyle=\color{dkgreen}\bfseries,
	numberfirstline=true
}

% -----------------------------------------------------------------------------
% INI, Archivos de configuraciones
% -----------------------------------------------------------------------------
\lstdefinestyle{ini}{
	language={},
	commentstyle=\color{gray}\ttfamily,
	keywordstyle={\color{black}\bfseries},
	morecomment=[l]{;},
	morecomment=[l]{\#},
	morecomment=[s][\color{dkgreen}\bfseries]{[}{]},
	morekeywords={},
	otherkeywords={=,:}
}

% -----------------------------------------------------------------------------
% Java
% -----------------------------------------------------------------------------
\lstdefinestyle{java}{
	language=Java,
	breakatwhitespace=true,
	keepspaces=true
}

% -----------------------------------------------------------------------------
% Javascript
% -----------------------------------------------------------------------------
\lstdefinelanguage{JavaScript}{
	comment=[l]{//},
	keepspaces=true,
	keywords={
		break,else,false,for,function,if,in,new,null,return,true,typeof,var,while
	},
	morecomment=[s]{/*}{*/},
	morestring=[b]',
	morestring=[b]",
	morestring=[b]`,
	ndkeywords={
		await,async,case,catch,class,const,default,do,enum,export,extends,finally,from,implements,import,instanceof,let,static,super,switch,then,this,throw,try
	},
	ndkeywordstyle=\color{blue}\bfseries,
	sensitive=false
}
\lstdefinestyle{js}{
	language=JavaScript
}

% -----------------------------------------------------------------------------
% JSON
% -----------------------------------------------------------------------------
\lstdefinestyle{json}{
	literate=*{0}{{{\color{numb}0}}}{1}{1}{{{\color{numb}1}}}{1}{2}
	{{{\color{numb}2}}}{1}{3}{{{\color{numb}3}}}{1}{4}{{{\color{numb}4}}}
	{1}{5}{{{\color{numb}5}}}{1}{6}{{{\color{numb}6}}}{1}{7}{{{\color{numb}7}}}
	{1}{8}{{{\color{numb}8}}}{1}{9}{{{\color{numb}9}}}{1}{:}
	{{{\color{punct}{:}}}}{1}{,}{{{\color{punct}{,}}}}{1}{\{}
	{{{\color{delim}{\{}}}}{1}{\}}{{{\color{delim}{\}}}}}
	{1}{[}{{{\color{delim}{[}}}}{1}{]}{{{\color{delim}{]}}}}{1},
	tabsize=2
}

% -----------------------------------------------------------------------------
% Kotlin
% -----------------------------------------------------------------------------
\lstdefinestyle{kotlin}{
	comment=[l]{//},
	emph={delegate,filter,first,firstOrNull,forEach,lazy,map,mapNotNull,println,
		return@},
	emphstyle={\color{blue}},
	keywords={
		abstract,actual,as,as?,break,by,class,companion,continue,data,do,dynamic,else,enum,expect,false,final,for,fun,get,if,import,in,interface,internal,is,null,object,override,package,private,public,return,set,super,suspend,this,throw,true,try,typealias,val,var,vararg,when,where,while
	},
	morecomment=[s]{/*}{*/},
	morestring=[b]",
	morestring=[s]{"""*}{*"""},
	ndkeywords={
		@Deprecated,@JvmField,@JvmName,@JvmOverloads,@JvmStatic,@JvmSynthetic,Array,Byte,Double,Float,Int,Integer,Iterable,Long,Runnable,Short,String
	},
	ndkeywordstyle=\color{BurntOrange}\bfseries,
	sensitive=true
}

% -----------------------------------------------------------------------------
% LaTeX
% -----------------------------------------------------------------------------
\lstdefinestyle{latex}{
	language=TeX,
	morekeywords={
		aacos,aasin,aatan,acos,addimage,addimageboxed,align,asin,atan,begin,bibitem,bibliography,bigstrut,boldmath,bookmarksetup,boxed,cancelto,caption,changeheadertitle,checkmark,checkvardefined,cite,dd,degree,eqref,equal,frac,fracnpartial,fullcite,hline,href,ifthenelse,imagesnewline,imageshspace,imagesvspace,includehfpdf,includefullhfpdf,insertalign,insertalignanum,insertaligncaptioned,insertaligncaptioned,insertaligncaptionedanum,insertaligned,insertalignedanum,insertalignedcaptioned,insertalignedcaptionedanum,insertemail,insertemptypage,inserteqimage,insertequation,insertequationanum,insertequationcaptioned,insertequationcaptionedanum,insertgather,insertgatheranum,insertgathercaptioned,insertgathercaptionedanum,insertgathered,insertgatheredanum,insertgatheredcaptioned,insertgatheredcaptionedanum,insertimage,insertimageleft,insertimageright,insertindextitle,insertindextitlepage,insertphone,isundefined,itemresize,label,LaTeX,lipsum,lpow,makeatletter,makeatother,newcommand,newcounter,newp,newpage,pow,quotes,ref,renewcommand,section,sectionanum,setcounter,setlength,shortcite,sourcecode,sourcecodep,subsection,subsectionanum,subsubsection,subsubsectionanum,subsubsubsection,subsubsubsection,subsubsubsectionanum,textbf,textit,textregistered,textsuperscript,texttt,throwbadconfig,unboldmath,url,xspace
	}
}

% -----------------------------------------------------------------------------
% Lisp
% -----------------------------------------------------------------------------
\lstdefinestyle{lisp}{
	language=Lisp,
	morekeywords={if}
}

% -----------------------------------------------------------------------------
% Lua
% -----------------------------------------------------------------------------
\lstdefinestyle{lua}{
	language={[5.2]Lua}
}

% -----------------------------------------------------------------------------
% Maple
% -----------------------------------------------------------------------------
\lstdefinelanguage{Maple}{
	morecomment=[l]\#,
	morekeywords={
		and,assuming,break,by,catch,description,do,done,elif,else,end,error,export,fi,finally,for,from,global,if,implies,in,intersect,local,minus,mod,module,next,not,o,option,options,or,proc,quit,read,restart,return,save,stop,subset,then,to,try,union,use,uses,with,while,xor
	},
	morestring=[b]",
	morestring=[d],
	sensitive=true
} 
\lstdefinestyle{maple}{
	language=Maple
}

% -----------------------------------------------------------------------------
% Matlab
% -----------------------------------------------------------------------------
\lstdefinestyle{matlab}{
	language=Matlab,
	deletekeywords={fft},
	keepspaces=true,
	morecomment=[l]\%,
	morecomment=[n]{\%\{\^^M}{\%\}\^^M},
	morekeywords={
		addOptional,box,break,catch,cell,classdef,continue,deal,double,end,factorial,for,gradient,hessian,if,isa,ltitr,matlab2tikz,methods,minor,movegui,normcdf,normpdf,on,ones,parse,persistent,poissrnd,properties,repmat,solve,strcat,subs,syms,try,var,warning,xlim,ylim
	}
}

% -----------------------------------------------------------------------------
% Octave
% -----------------------------------------------------------------------------
\lstdefinestyle{octave}{
	language=Octave,
	keepspaces=true,
	morecomment=[l]\%,
	morecomment=[n]{\%\{\^^M}{\%\}\^^M}
}

% -----------------------------------------------------------------------------
% OpenCL
% -----------------------------------------------------------------------------
\lstdefinestyle{opencl}{
	language=C++,
	breakatwhitespace=false,
	emph={
		bool3,bool4,bool8,bool16,char2,char3,char4,char8,char16,complex,constant,event_t,bool2,float2,float3,float4,float8,float16,global,half2,half3,half4,half8,half16,image2d_t,image3d_t,imaginary,int2,int3,int4,int8,int16,kernel,local,long2,long3,long4,long8,long16,private,quad,quad2,quad3,quad4,quad8,quad16,sampler_t,short2,short3,short4,short8,short16,uchar2,uchar3,uchar4,uchar8,uchar16,uint2,uint3,uint4,uint8,uint16,ulong2,ulong3,ulong4,ulong8,ulong16,ushort2,ushort3,ushort4,ushort8,ushort16,__constant,__global,__kernel,__local,__private
	},
	emphstyle=\color{dkcyan}\ttfamily,
	morecomment=[l][\color{magenta}]{\#}
}

% -----------------------------------------------------------------------------
% OpenSees
% -----------------------------------------------------------------------------
\lstdefinestyle{opensees}{
	language=tcl,
	breakatwhitespace=false,
	emph=[1]{
		-dir,-dof,-ele,-eleRange,-file,-height,-increment,-initial,-iNode,-integration,-iterate,-jNode,-kNode,-mass,-mat,-matConcrete,-matShear,-matSteel,-max,-maxDim,-maxEta,-maxIter,-min,-minEta,-ndf,-ndm,-node,-nodeRange,-numSublevels,-numSubSteps,-region,-rho,-sections,-thick,-time,-tol,-type,-width
	},
	emphstyle=[1]\color{black}\bfseries\em,
	% moredelim=[s][\color{black}\bfseries\em]{-}{\ },
	keepspaces=true,
	morecomment=[l]{\#},
	morekeywords={
		algorithm,analysis,analyze,constraints,deformation,disp,DisplayModel2D,DisplayModel3D,element,equalDOF,fix,fixX,fixY,fixZmodel,geomTransf,initialize,integrator,loadConst,mass,node,numberer,pattern,printA,PySimple1Gen,reaction,recorder,region,section,system,test,uniaxialMaterial,wipe
	},
	ndkeywords={
		9_4_QuadUP,20_8_BrickUP,AC3D8,Aggregator,ArcLength,ASI3D8,AV3D4,AxialSp,AxialSpHD,BandGeneral,BARSLIP,BasicBuilder,bbarBrick,bbarBrickUP,bbarQuad,bbarQuadUP,BeamColumnJoint,BeamContact2D,BeamContact3D,BeamEndContact3D,BFGS,Bilin,BilinearOilDamper,Bond_SP01,BoucWen,Brick20N,brickUP,Broyden,BWBN,Cast,CatenaryCable,CentralDifference,CFSSSWP,CFSWSWP,Concrete01,Concrete01WithSITC,Concrete02,Concrete03,Concrete04,Concrete06,Concrete07,ConcreteCM,ConcreteD,ConfinedConcrete01,constraintsTypeGravity,corotTruss,corotTrussSection,CoupledZeroLength,DeformedShape,dispBeamColumn,dispBeamColumnInt,DisplacementControl,Dodd_Restrepo,ECC01,Elastic,elasticBeamColumn,ElasticBilin,ElasticMultiLinear,ElasticPP,ElasticPPGap,ElasticTimoshenkoBeam,ElasticTubularJoint,elastomericBearingBoucWen,elastomericBearingPlasticity,ElastomericX,EnergyIncr,enhancedQuad,ENT,Explicitdifference,Fatigue,flatSliderBearing,forceBeamColumn,forceBeamColumn,FourNodeTetrahedron,FPBearingPTV,FRPConfinedConcrete,GeneralizedAlpha,Hardening,HDR,HHT,HyperbolicGapMaterial,Hysteretic,ImpactMaterial,InitStrainMaterial,InitStressMaterial,Joint2D,KikuchiAikenHDR,KikuchiAikenLRB,KikuchiBearing,KrylovNewton,LeadRubberX,LimitState,Linear,LoadControl,LoadControl,MinMax,MinUnbalDispNorm,ModElasticBeam2d,ModifiedNewton,ModIMKPeakOriented,ModIMKPinching,MultiLinear,multipleShearSpring,MVLEM,Newmark,NewtonLineSearch,Node,NodeNumbers,nonlinearBeamColumn,NormDispIncr,numberer,Parallel,PathIndependentMaterial,pattern,Pinching4,PinchingLimitStateMaterial,Plain,PyLiq1,PySimple1,quad,quadUP,QzSimple1,RambergOsgoodSteel,rayleigh,RCM,ReinforcingSteel,RJWatsonEqsBearing,SAWS,SecantNewton,SelfCentering,Series,SFI_MVLEM,ShallowFoundationGen,ShellDKGQ,ShellDKGT,ShellMITC4,ShellNL,ShellNLDKGQ,ShellNLDKGT,SimpleContact2D,SimpleContact3D,singleFPBearing,SparseGeneral,SSPbrick,SSPbrickUP,SSPquad,SSPquadUP,Static,stdBrick,Steel01,Steel01,Steel02,Steel4,SteelMPF,SurfaceLoad,TFP,Transient,TRBDF2,tri31,TripleFrictionPendulum,truss,trussSection,twoNodeLink,TzLiq1,TzSimple1,ViewScale,Viscous,ViscousDamper,VS3D4,YamamotoBiaxialHDR,zeroLength,zeroLengthContact,zeroLengthContactNTS2D,zeroLengthImpact3D,zeroLengthImpact3D,zeroLengthInterface2D,zeroLengthND,zeroLengthSection
	},
	ndkeywordstyle=\color{dkcyan}\ttfamily
}

% -----------------------------------------------------------------------------
% Pascal
% -----------------------------------------------------------------------------
\lstdefinestyle{pascal}{
	language=Pascal,
	morecomment=[l]{//},
	sensitive=false
}

% -----------------------------------------------------------------------------
% Perl
% -----------------------------------------------------------------------------
\lstdefinestyle{perl}{
	language=Perl,
	alsoletter={\%},
	breakatwhitespace=false,
	keepspaces=true
}

% -----------------------------------------------------------------------------
% PHP
% -----------------------------------------------------------------------------
\lstdefinestyle{php}{
	language=php,
	emph=[1]{
		php
	},
	emph=[2]{
		if,and,or,else
	},
	emph=[3]{
		abstract,as,const,else,elseif,endfor,endforeach,endif,extends,final,for,foreach,global,if,implements,private,protected,public,static,var
	},
	emphstyle=[1]\color{black},
	emphstyle=[2]\color{blue},
	keywords={
		abstract,and,array,as,break,callable,case,catch,class,clone,const,continue,declare,default,die,do,echo,else,elseif,empty,enddeclare,endfor,endforeach,endif,endswitch,endwhile,eval,exit,extends,final,finally,for,foreach,function,global,goto,if,implements,include,include_once,instanceof,insteadof,interface,isset,list,namespace,new,or,print,private,protected,public,require,require_once,return,static,switch,throw,trait,try,unset,use,var,while,xor,yield,__halt_compiler
	},
	showlines=true,
	upquote=true
}

% -----------------------------------------------------------------------------
% Texto plano
% -----------------------------------------------------------------------------
\lstdefinestyle{plaintext}{
	language={},
	keepspaces=true,
	postbreak={},
	tabsize=4
}

% -----------------------------------------------------------------------------
% Pseudocódigo
% -----------------------------------------------------------------------------
\lstdefinestyle{pseudocode}{
	language={},
	backgroundcolor=\color{white},
	breakatwhitespace=false,
	commentstyle=\color{gray}\upshape,
	frame=tb,
	keepspaces=true,
	keywords={
		and,be,begin,break,datatype,do,elif,else,end,for,foreach,fun,function,if,in,input,let,not,null,or,output,pop,procedure,push,repeat,return,swap,until,while,xor
	},
	keywordstyle=\color{black}\bfseries,
	mathescape=true,
	morecomment=[l]{//},
	morecomment=[l]{\#},
	morecomment=[s]{/*}{*/},
	morecomment=[s]{/**}{*/},
	sensitive=false,
	stringstyle=\color{dkgray}\bfseries\em
}

% -----------------------------------------------------------------------------
% Python
% -----------------------------------------------------------------------------
\lstdefinestyle{python}{
	language=Python,
	breakatwhitespace=false,
	emph={
		AbstractSet,Any,AsyncContextManager,AsyncGenerator,AsyncIterable,AsyncIterator,Awaitable,AwaitableGenerator,BinaryIO,ByteString,Callable,Collection,Container,ContextManager,Coroutine,Dict,ForwardRef,Generator,GenericMeta,Hashable,IO,ItemsView,Iterable,Iterator,KeysView,List,Mapping,MappingView,Match,Meta,MutableMapping,MutableSequence,MutableSet,NamedTuple,Pattern,Reversible,Sequence,Sized,SupportInts,SupportsAbs,SupportsBytes,SupportsComplex,SupportsFloat,SupportsIndex,SupportsRound,TextIO,Tuple,TypeAlias,TYPE_CHECKING,Union,ValuesView,__add__,__and__,__eq__,__floordiv__,__ge__,__gt__,__init__,__le__,__lt__,__main__,__mod__,__mul__,__name__,__ne__,__or__,__pow__,__repr__,__str__,__sub__,__truediv__,__xor__
	},
	emphstyle=\color{dkcyan}\ttfamily,
	keepspaces=true,
	morecomment=[s][\color{BurntOrange}]{@}{\ },
	morekeywords={as,assert,close,listdir,None,self,sorted,split,strip,with}
}

% -----------------------------------------------------------------------------
% R
% -----------------------------------------------------------------------------
\lstdefinestyle{r}{
	language=R,
	alsoletter={.<-},
	alsoother={._$},
	deletekeywords={df,data,frame,length,as,character},
	morecomment=[l]\#,
	morestring=[d]',
	morestring=[d]",
	otherkeywords={!,!=,~,$,*,\&,\%/\%,\%*\%,\%\%,<-,<<-,/}
}

% -----------------------------------------------------------------------------
% Ruby
% -----------------------------------------------------------------------------
\lstdefinestyle{ruby}{
	language=Ruby,
	breakatwhitespace=true,
	morestring=[s][]{\#\{}{\}},
	morestring=*[d]{"},
	sensitive=true
}

% -----------------------------------------------------------------------------
% Scala
% -----------------------------------------------------------------------------
\lstdefinestyle{scala}{
	language=scala,
	breakatwhitespace=true,
	morecomment=[l]{//},
	morecomment=[n]{/*}{*/},
	morekeywords={
		abstract,case,catch,class,def,do,else,extends,false,final,finally,for,if,implicit,import,match,mixin,new,null,object,override,package,private,protected,requires,return,sealed,super,this,throw,trait,true,try,type,val,var,while,with,yield
	},
	morestring=[b]',
	morestring=[b]",
	morestring=[b]""",
	otherkeywords={=>,<-,<\%,<:,>:,\#,@}
}

% -----------------------------------------------------------------------------
% Scheme
% -----------------------------------------------------------------------------
\lstdefinestyle{scheme}{
	language=Lisp,
	morecomment=[l]{;},
	morekeywords={
		and,begin,case,case-lambda,cond,cond-expand,define,delay,delay-force,do,else,force,guard,if,lambda,let,let*,let*-values,let-syntax,let-values,letrec,letrec*,letrec-syntax,make-parameter,make-promise,map,or,parameterize,promise?,quasiquote,quote,set!,syntax-rules,unless,when
	},
	morestring=[b]"
}

% -----------------------------------------------------------------------------
% SQL
% -----------------------------------------------------------------------------
\lstdefinestyle{sql}{
	language=SQL,
	breakatwhitespace=true
}

% -----------------------------------------------------------------------------
% TCL
% -----------------------------------------------------------------------------
\lstdefinestyle{tcl}{
	language=tcl,
	breakatwhitespace=false,
	keepspaces=true,
	morecomment=[l]{\#}
}

% -----------------------------------------------------------------------------
% Visual Basic
% -----------------------------------------------------------------------------
\lstdefinestyle{vbscript}{
	language=[Visual]Basic,
	extendedchars=true
}

% -----------------------------------------------------------------------------
% Verilog
% -----------------------------------------------------------------------------
\lstdefinestyle{verilog}{
	language=Verilog
}

% -----------------------------------------------------------------------------
% VDHL
% -----------------------------------------------------------------------------
\lstdefinelanguage{VHDL}{
	morekeywords=[1]{
		library,use,all,entity,is,port,in,out,end,architecture,of,begin,and,or,Not,downto,ALL
	},
	morekeywords=[2]{
		STD_LOGIC_VECTOR,STD_LOGIC,IEEE,STD_LOGIC_1164,NUMERIC_STD,STD_LOGIC_ARITH,STD_LOGIC_UNSIGNED,std_logic_vector,std_logic
	},
	morecomment=[l]--
}
\lstdefinestyle{vhdl}{
	language=VHDL
}

% -----------------------------------------------------------------------------
% XML
% -----------------------------------------------------------------------------
\lstdefinelanguage{XML}{
	morecomment=[s]{<?}{?>},
	morekeywords={
		encoding,type,version,xmlns
	},
	morestring=[b]",
	morestring=[s]{>}{<}
}
\lstdefinestyle{xml}{
	language=XML,
	tabsize=2
}

% -----------------------------------------------------------------------------
% Configuración de códigos fuente
% -----------------------------------------------------------------------------
\lstset{
	% \mbox{\textcolor{black}{$\hookrightarrow$}\space}
	aboveskip=0.75em,
	basicstyle={\sourcecodefonts\sourcecodefontf\color{\maintextcolor}},
	belowskip=1em,
	breaklines=true,
	columns=fullflexible,
	commentstyle=\color{dkgreen}\upshape,
	extendedchars=true,
	fontadjust=true,
	% fancyvrb=false,
	identifierstyle=\color{black},
	keepspaces=true,
	keywordstyle=\color{blue},
	literate={á}{{\'a}}1 {é}{{\'e}}1 {í}{{\'i}}1 {ó}{{\'o}}1 {ú}{{\'u}}1
		{Á}{{\'A}}1 {É}{{\'E}}1 {Í}{{\'I}}1 {Ó}{{\'O}}1 {Ú}{{\'U}}1 {à}{{\`a}}1
		{è}{{\`e}}1 {ì}{{\`i}}1 {ò}{{\`o}}1 {ù}{{\`u}}1 {À}{{\`A}}1 {È}{{\'E}}1
		{Ì}{{\`I}}1 {Ò}{{\`O}}1 {Ù}{{\`U}}1 {ä}{{\"a}}1 {ë}{{\"e}}1 {ï}{{\"i}}1
		{ö}{{\"o}}1 {ü}{{\"u}}1 {Ä}{{\"A}}1 {Ë}{{\"E}}1 {Ï}{{\"I}}1 {Ö}{{\"O}}1
		{Ü}{{\"U}}1 {â}{{\^a}}1 {ê}{{\^e}}1 {î}{{\^i}}1 {ô}{{\^o}}1 {û}{{\^u}}1
		{Â}{{\^A}}1 {Ê}{{\^E}}1 {Î}{{\^I}}1 {Ô}{{\^O}}1 {Û}{{\^U}}1 {œ}{{\oe}}1
		{Œ}{{\OE}}1 {æ}{{\ae}}1 {Æ}{{\AE}}1 {ß}{{\ss}}1 {ű}{{\H{u}}}1
		{Ű}{{\H{U}}}1 {ő}{{\H{o}}}1 {Ő}{{\H{O}}}1 {ç}{{\c c}}1 {Ç}{{\c C}}1
		{ø}{{\o}}1 {å}{{\r a}}1 {Å}{{\r A}}1 {€}{{\EUR}}1 {£}{{\pounds}}1
		{ñ}{{\~n}}1 {Ñ}{{\~N}}1 {¿}{{?``}}1 {¡}{{!``}}1 {«}{{\guillemotleft}}1
		{»}{{\guillemotright}}1 {°}{{\textdegree}}1 {∢}{{$\sphericalangle$}}1
		{¬}{{$\neg$}}1 {¨}{{\textasciidieresis}}1 {ã}{{\~a}}1 {Ã}{{\~a}}1
		{õ}{{\~o}}1 {Õ}{{\~O}}1 {Ð}{{\DJ}}1 {Ø}{{\O}}1 {Ý}{{\'Y}}1,
	numbers=left,
	numbersep={\sourcecodenumbersep pt},
	numberstyle=\tiny\color{dkgray},
	postbreak=\mbox{$\hookrightarrow$\space},
	showspaces=false,
	showstringspaces=false,
	showtabs=false,
	stepnumber=1,
	stringstyle=\color{mauve},
	tabsize={\sourcecodetabsize}
}

% -----------------------------------------------------------------------------
% Chequeo de estilos, cualquier nuevo estilo añadirlo a esta lista
% -----------------------------------------------------------------------------
\newcommand{\checkvalidsourcecodestyle}[1]{
	\ifthenelse{\equal{#1}{assemblerx64}}{}{
	\ifthenelse{\equal{#1}{assemblerx86}}{}{
	\ifthenelse{\equal{#1}{bash}}{}{
	\ifthenelse{\equal{#1}{c}}{}{
	\ifthenelse{\equal{#1}{cpp}}{}{
	\ifthenelse{\equal{#1}{csharp}}{}{
	\ifthenelse{\equal{#1}{css}}{}{
	\ifthenelse{\equal{#1}{csv}}{}{
	\ifthenelse{\equal{#1}{cuda}}{}{
	\ifthenelse{\equal{#1}{docker}}{}{
	\ifthenelse{\equal{#1}{fortran}}{}{
	\ifthenelse{\equal{#1}{glsl}}{}{
	\ifthenelse{\equal{#1}{haskell}}{}{
	\ifthenelse{\equal{#1}{html5}}{}{
	\ifthenelse{\equal{#1}{ini}}{}{
	\ifthenelse{\equal{#1}{java}}{}{
	\ifthenelse{\equal{#1}{js}}{}{
	\ifthenelse{\equal{#1}{json}}{}{
	\ifthenelse{\equal{#1}{kotlin}}{}{
	\ifthenelse{\equal{#1}{latex}}{}{
	\ifthenelse{\equal{#1}{lisp}}{}{
	\ifthenelse{\equal{#1}{lua}}{}{
	\ifthenelse{\equal{#1}{maple}}{}{
	\ifthenelse{\equal{#1}{matlab}}{}{
	\ifthenelse{\equal{#1}{octave}}{}{
	\ifthenelse{\equal{#1}{opencl}}{}{
	\ifthenelse{\equal{#1}{opensees}}{}{
	\ifthenelse{\equal{#1}{pascal}}{}{
	\ifthenelse{\equal{#1}{perl}}{}{
	\ifthenelse{\equal{#1}{php}}{}{
	\ifthenelse{\equal{#1}{plaintext}}{}{
	\ifthenelse{\equal{#1}{pseudocode}}{}{
	\ifthenelse{\equal{#1}{python}}{}{
	\ifthenelse{\equal{#1}{r}}{}{
	\ifthenelse{\equal{#1}{ruby}}{}{
	\ifthenelse{\equal{#1}{scala}}{}{
	\ifthenelse{\equal{#1}{scheme}}{}{
	\ifthenelse{\equal{#1}{sql}}{}{
	\ifthenelse{\equal{#1}{tcl}}{}{
	\ifthenelse{\equal{#1}{vbscript}}{}{
	\ifthenelse{\equal{#1}{verilog}}{}{
	\ifthenelse{\equal{#1}{vhdl}}{}{
	\ifthenelse{\equal{#1}{xml}}{}{
		\errmessage{LaTeX Warning: Estilo de codigo desconocido. Valores esperados: assemblerx64,assemblerx86,bash,c,cpp,csharp,css,csv,cuda,docker,fortran,glsl,haskell,html5,ini,java,js,json,kotlin,latex,lisp,lua,maple,matlab,octave,opencl,opensees,pascal,perl,php,plaintext,pseudocode,python,r,ruby,scala,scheme,sql,tcl,vbscript,verilog,vhdl,xml}
	}}}}}}}}}}}}}}}}}}}}}}}}}}}}}}}}}}}}}}}}}}}
}
% -----------------------------------------------------------------------------
% Estilo de enumeración en griego
% -----------------------------------------------------------------------------
\RequirePackage{enumitem}
\makeatletter
\def\greek#1{\expandafter\@greek\csname c@#1\endcsname}
\def\Greek#1{\expandafter\@Greek\csname c@#1\endcsname}
\def\@greek#1{
	\ifcase#1
		\or $\alpha$
		\or $\beta$
		\or $\gamma$
		\or $\delta$
		\or $\epsilon$
		\or $\zeta$
		\or $\eta$
		\or $\theta$
		\or $\iota$
		\or $\kappa$
		\or $\lambda$
		\or $\mu$
		\or $\nu$
		\or $\xi$
		\or $o$
		\or $\pi$
		\or $\rho$
		\or $\sigma$
		\or $\tau$
		\or $\upsilon$
		\or $\phi$
		\or $\chi$
		\or $\psi$
		\or $\omega$
	\fi
}
\def\@Greek#1{
	\ifcase#1
		\or $\mathrm{A}$
		\or $\mathrm{B}$
		\or $\Gamma$
		\or $\Delta$
		\or $\mathrm{E}$
		\or $\mathrm{Z}$
		\or $\mathrm{H}$
		\or $\Theta$
		\or $\mathrm{I}$
		\or $\mathrm{K}$
		\or $\Lambda$
		\or $\mathrm{M}$
		\or $\mathrm{N}$
		\or $\Xi$
		\or $\mathrm{O}$
		\or $\Pi$
		\or $\mathrm{P}$
		\or $\Sigma$
		\or $\mathrm{T}$
		\or $\mathrm{Y}$
		\or $\Phi$
		\or $\mathrm{X}$
		\or $\Psi$
		\or $\Omega$
	\fi
}
\makeatother
\AddEnumerateCounter{\greek}{\@greek}{24}
\AddEnumerateCounter{\Greek}{\@Greek}{12}

% CONFIGURACIÓN INICIAL DEL DOCUMENTO
% -----------------------------------------------------------------------------
% Se revisa si las variables no han sido borradas
% -----------------------------------------------------------------------------
\def\titulodelinforme{\titulodelreporte}
\checkvardefined{\autordeldocumento}
\checkvardefined{\departamentouniversidad}
\checkvardefined{\localizacionuniversidad}
\checkvardefined{\nombredelcurso}
\checkvardefined{\nombrefacultad}
\checkvardefined{\nombreuniversidad}
\checkvardefined{\temaatratar}
\checkvardefined{\titulodelinforme}

% -----------------------------------------------------------------------------
% Se añade \xspace a las variables
% -----------------------------------------------------------------------------
\makeatletter
	\g@addto@macro\autordeldocumento\xspace
	\g@addto@macro\codigodelcurso\xspace
	\g@addto@macro\departamentouniversidad\xspace
	\g@addto@macro\localizacionuniversidad\xspace
	\g@addto@macro\nombredelcurso\xspace
	\g@addto@macro\nombrefacultad\xspace
	\g@addto@macro\nombreuniversidad\xspace
	\g@addto@macro\temaatratar\xspace
	\g@addto@macro\titulodelinforme\xspace
\makeatother

% -----------------------------------------------------------------------------
% -----------------------------------------------------------------------------
% Verifica tamaños de fuentes
% -----------------------------------------------------------------------------
\corecheckfontsize{\captionfontsize}
\corecheckfontsize{\subcaptionfsize}

% -----------------------------------------------------------------------------
% Se activan números en menú marcadores del pdf
% -----------------------------------------------------------------------------
\ifthenelse{\equal{\cfgpdfsecnumbookmarks}{true}}{
	\bookmarksetup{numbered}}{
}

% -----------------------------------------------------------------------------
% Se define metadata del pdf
% -----------------------------------------------------------------------------
\ifthenelse{\equal{\cfgshowbookmarkmenu}{true}}{
	\def\cfgpdfpagemode {UseOutlines}
	}{
	\def\cfgpdfpagemode {UseNone}
}
\ifthenelse{\equal{\usepdfmetadata}{true}}{
	\def\pdfmetainfoautor {\autordeldocumento}
	\def\pdfmetainfocodigodelcurso {\codigodelcurso}
	\def\pdfmetainfonombredelcurso {\nombredelcurso}
	\def\pdfmetainfoautor {\autordeldocumento}
	\def\pdfmetainfotema {\temaatratar}
	\def\pdfmetainfotitulo {\titulodelreporte}
	\def\pdfmetainfounidepto {\departamentouniversidad}
	\def\pdfmetainfouninombre {\departamentouniversidad}
	\def\pdfmetainfouniubicacion {\departamentouniversidad}
}{
	\def\pdfmetainfoautor {}
	\def\pdfmetainfocodigodelcurso {}
	\def\pdfmetainfonombredelcurso {}
	\def\pdfmetainfoautor {}
	\def\pdfmetainfotema {}
	\def\pdfmetainfotitulo {}
	\def\pdfmetainfounidepto {}
	\def\pdfmetainfouninombre {}
	\def\pdfmetainfouniubicacion {}
}
\hypersetup{
	bookmarksopen={\cfgpdfbookmarkopen},
	bookmarksopenlevel={\cfgbookmarksopenlevel},
	bookmarkstype={toc},
	pdfauthor={\pdfmetainfoautor},
	pdfcenterwindow={\cfgpdfcenterwindow},
	pdfcopyright={\cfgpdfcopyright},
	pdfcreator={LaTeX},
	pdfdisplaydoctitle={\cfgpdfdisplaydoctitle},
	pdfencoding={unicode},
	pdffitwindow={\cfgpdffitwindow},
	pdfinfo={
		Curso.Codigo={\pdfmetainfocodigodelcurso},
		Curso.Nombre={\pdfmetainfonombredelcurso},
		Documento.Autor={\pdfmetainfoautor},
		Documento.Tema={\pdfmetainfotema},
		Documento.Titulo={\pdfmetainfotitulo},
		Template.Autor.Alias={ppizarror},
		Template.Autor.Email={pablo@ppizarror.com},
		Template.Autor.Nombre={Pablo Pizarro R.},
		Template.Autor.Web={https://ppizarror.com/},
		Template.Codificacion={UTF-8},
		Template.Fecha={28/06/2020},
		Template.Latex.Compilador={pdflatex},
		Template.Licencia.Tipo={MIT},
		Template.Licencia.Web={https://opensource.org/licenses/MIT/},
		Template.Nombre={Template-Reporte},
		Template.Tipo={Normal},
		Template.Version.Dev={2.0.1-REPT},
		Template.Version.Hash={3E30455CA246DFEC02C8F7224C4513DB},
		Template.Version.Release={2.0.1},
		Template.Web.Dev={https://github.com/Template-Latex/Template-Reporte/},
		Template.Web.Manual={https://latex.ppizarror.com/reporte},
		Universidad.Departamento={\pdfmetainfounidepto},
		Universidad.Nombre={\pdfmetainfouninombre},
		Universidad.Ubicacion={\pdfmetainfouniubicacion}
	},
	pdfkeywords={\cfgpdfkeywords},
	pdflang={\documentlang},
	pdfmenubar={\cfgpdfmenubar},
	pdfpagelayout={\cfgpdflayout},
	pdfpagemode={\cfgpdfpagemode},
	pdfproducer={Template-Reporte v2.0.1 | (Pablo Pizarro R.) ppizarror.com},
	pdfremotestartview={Fit},
	pdfstartpage={1},
	pdfstartview={\cfgpdfpageview},
	pdfsubject={\pdfmetainfotema},
	pdftitle={\pdfmetainfotitulo},
	pdftoolbar={\cfgpdftoolbar}
}

% -----------------------------------------------------------------------------
% Establece la carpeta de imágenes por defecto
% -----------------------------------------------------------------------------
\graphicspath{{./\defaultimagefolder}}

% -----------------------------------------------------------------------------
% Elimina el espacio vertical de los flotantes
% -----------------------------------------------------------------------------
\makeatletter
\ifthenelse{\equal{\fpremovetopbottomcenter}{true}}{
	\setlength{\@fptop}{0pt}
	\setlength{\@fpbot}{0pt}
}{}
\makeatother

% -----------------------------------------------------------------------------
% Definición de valores e dimensiones
% -----------------------------------------------------------------------------
\renewcommand{\baselinestretch}{\documentinterline} % Ajuste del entrelineado
\setlength{\headheight}{64 pt} % Tamaño de la cabecera sin fancyhdr
% \setcounter{MaxMatrixCols}{20} % Número máximo de columnas en matrices
\setlength{\footnotemargin}{\marginfootnote pt} % Margen del footnote
\setlength{\columnsep}{\columnsepwidth em} % Separación entre columnas
\ifthenelse{\equal{\showlinenumbers}{true}}{
	\setlength{\linenumbersep}{0.50cm}
	\renewcommand\linenumberfont{\normalfont\tiny\color{\linenumbercolor}}
	}{
}

% -----------------------------------------------------------------------------
% Posición inicial de los objetos
% -----------------------------------------------------------------------------
\floatplacement{figure}{\imagedefaultplacement}
\floatplacement{table}{\tabledefaultplacement}
\floatplacement{tikz}{\tikzdefaultplacement}

% -----------------------------------------------------------------------------
% Configuración de los colores
% -----------------------------------------------------------------------------
\color{\maintextcolor} % Color principal
\arrayrulecolor{\tablelinecolor} % Color de las líneas de las tablas
\sethlcolor{\highlightcolor} % Color del subrayado por defecto
\ifthenelse{\equal{\showborderonlinks}{true}}{
	% Color de links con borde
	\hypersetup{
		citebordercolor=\numcitecolor,
		linkbordercolor=\linkcolor,
		urlbordercolor=\urlcolor
	}
}{
	% Color de links sin borde
	\hypersetup{ % No reorganizar
		hidelinks,
		colorlinks=true,
		citecolor=\numcitecolor,
		linkcolor=\linkcolor,
		urlcolor=\urlcolor
	}
}
\ifthenelse{\equal{\colorpage}{white}}{}{
	\pagecolor{\colorpage}
}

% -----------------------------------------------------------------------------
% Configuración de las leyendas
% -----------------------------------------------------------------------------

% Márgenes de las leyendas por defecto
\setcaptionmargincm{\captionlrmargin}
\ifthenelse{\equal{\captiontextbold}{true}}{ % Texto en negrita en etiquetas
	\renewcommand{\captiontextbold}{bf}}{
	\renewcommand{\captiontextbold}{}
}
\ifthenelse{\equal{\captiontextsubnumbold}{true}}{ % Número en negritas
	\renewcommand{\captiontextsubnumbold}{bf}}{
	\renewcommand{\captiontextsubnumbold}{}
}

% Se configura el texto de los caption
\captionsetup{
	font={\captionfontsize},
	labelfont={color=\captioncolor, \captiontextbold},
	labelformat={\captionlabelformat},
	labelsep={\captionlabelsep},
	textfont={color=\captiontextcolor},
	singlelinecheck=on
}

% Configura texto de los subcaption
\captionsetup*[subfigure]{
	font={\subcaptionfsize},
	labelfont={color=\captioncolor, \captiontextsubnumbold},
	labelformat={\subcaptionlabelformat},
	labelsep={\subcaptionlabelsep},
	textfont={color=\captiontextcolor},
	singlelinecheck=on
}
\captionsetup*[subtable]{
	font={\subcaptionfsize},
	labelfont={color=\captioncolor, \captiontextsubnumbold},
	labelformat={\subcaptionlabelformat},
	labelsep={\subcaptionlabelsep},
	textfont={color=\captiontextcolor},
	singlelinecheck=on
}

% Configuración de márgenes en las figuras
\floatsetup[figure]{
	captionskip=\captiontbmarginfigure pt
}

% Configuración de márgenes en las tablas
\floatsetup[table]{
	captionskip=\captiontbmargintable pt
}

% Caption superior en figuras
\ifthenelse{\equal{\figurecaptiontop}{true}}{
	\floatsetup[figure]{position=above}}{
}

% Caption superior en tablas
\ifthenelse{\equal{\tablecaptiontop}{true}}{
	\floatsetup[table]{position=top}
	}{
	\floatsetup[table]{position=bottom}
}

% Alineado de leyendas
\ifthenelse{\equal{\captionalignment}{justified}}{ % Leyenda justificada
	\captionsetup{
		format=plain,
		justification=justified
	}
}{
\ifthenelse{\equal{\captionalignment}{centered}}{ % Leyenda centrada
	\captionsetup{
		justification=centering
	}
}{
\ifthenelse{\equal{\captionalignment}{left}}{ % Leyenda alineada a la izquierda
	\captionsetup{
		justification=raggedright,
		singlelinecheck=false
	}
}{
\ifthenelse{\equal{\captionalignment}{right}}{ % Leyenda alineada a la derecha
	\captionsetup{
		justification=raggedleft,
		singlelinecheck=false
	}
}{
	\throwbadconfig{Posicion de leyendas desconocida}{\captionalignment}{justified,centered,left,right}}}}
}

% -----------------------------------------------------------------------------
% Configuración de referencias y citas
% -----------------------------------------------------------------------------
\ifthenelse{\equal{\stylecitereferences}{natbib}}{
	\def\twocolumnreferencesmargin{-0.35cm}
	\bibliographystyle{\natbibrefstyle}
	\setlength{\bibsep}{\natbibrefsep pt}
	\newcommand{\shortcite}[1]{\citep{#1}}
	\newcommand{\fullcite}[1]{\citet{#1}}
}{
\ifthenelse{\equal{\stylecitereferences}{apacite}}{
	\def\twocolumnreferencesmargin{-0.39cm}
	\bibliographystyle{\apacitestyle}
	\setlength{\bibitemsep}{\apaciterefsep pt}
	\newcommand{\citep}[1]{\fullcite{#1}}
	\newcommand{\citet}[1]{\shortcite{#1}}
}{
\ifthenelse{\equal{\stylecitereferences}{bibtex}}{
	\def\twocolumnreferencesmargin{-0.35cm}
	\bibliographystyle{\bibtexstyle}
	\newlength{\bibitemsep}
	\setlength{\bibitemsep}{.2\baselineskip plus .05\baselineskip minus .05\baselineskip}
	\newlength{\bibparskip}\setlength{\bibparskip}{0pt}
	\let\oldthebibliography\thebibliography
	\renewcommand\thebibliography[1]{
		\oldthebibliography{#1}
		\setlength{\parskip}{\bibitemsep}
		\setlength{\itemsep}{\bibparskip}
	}
	\setlength{\bibitemsep}{\bibtexrefsep pt}
}{
	\throwbadconfig{Estilo citas desconocido}{\stylecitereferences}{bibtex,apacite,natbib}}}
}
% Crea referencias enumeradas en apacite
\makeatletter
\ifthenelse{\equal{\stylecitereferences}{apacite}}{
	\ifthenelse{\equal{\apaciterefnumber}{true}}{
		\newcounter{apaciteNumberCounter}
		\renewcommand{\theapaciteNumberCounter}{\apaciterefnumberinit\arabic{apaciteNumberCounter}\apaciterefnumberfinal} % Formato de número
		\patchcmd{\@lbibitem}{\item[}{\item[\stepcounter{apaciteNumberCounter}{\hss\llap{\theapaciteNumberCounter}\quad}}{}{}
		\setlength{\bibleftmargin}{2.54em}
		\setlength{\bibindent}{-0.54em}
	}{}
}{}
\makeatother
% Desactiva la URL de apacite
\ifthenelse{\equal{\stylecitereferences}{apacite}}{
	\ifthenelse{\equal{\apaciteshowurl}{false}}{
		\renewenvironment{APACrefURL}[1][]{}{}
		\AtBeginEnvironment{APACrefURL}{\renewcommand{\url}[1]{}}
		\renewcommand{\doiprefix}{doi:~\kern-1pt}
	}{}
}{}
% Referencias en 2 columnas
\makeatletter
\ifthenelse{\equal{\twocolumnreferences}{true}}{
	\renewenvironment{thebibliography}[1]
	{\begin{multicols}{2}[\section*{\refname}\vspace{\twocolumnreferencesmargin}]
		\@mkboth{\MakeUppercase\refname}{\MakeUppercase\refname}
		\list{\@biblabel{\@arabic\c@enumiv}}
		{\settowidth\labelwidth{\@biblabel{#1}}
			\leftmargin\labelwidth
			\advance\leftmargin\labelsep
			\@openbib@code
			\usecounter{enumiv}
			\let\p@enumiv\@empty
			\renewcommand\theenumiv{\@arabic\c@enumiv}}
		\sloppy
		\clubpenalty 4000
		\@clubpenalty \clubpenalty
		\widowpenalty 4000
		\sfcode`\.\@m}
		{\def\@noitemerr
		{\@latex@warning{Ambiente `thebibliography' no definido}}
		\endlist\end{multicols}}}{}
\makeatother

% -----------------------------------------------------------------------------
% Configuración anexo
% -----------------------------------------------------------------------------
\patchcmd{\appendices}{\quad}{\sectionappendixlastchar\quad}{}{}

% -----------------------------------------------------------------------------
% Se añade listings (código fuente) a tocloft
% -----------------------------------------------------------------------------
\begingroup
	\makeatletter
	\let\newcounter\@gobble\let\setcounter\@gobbletwo
	\globaldefs\@ne\let\c@loldepth\@ne
	\newlistof{listings}{lol}{\lstlistlistingname}
	\newlistentry{lstlisting}{lol}{0}
	\makeatother
\endgroup

% -----------------------------------------------------------------------------
% Crea índice de ecuaciones
% -----------------------------------------------------------------------------
\newcommand{\listindexequationsname}{\nomlteqn}
\newlistof{myindexequations}{equ}{\listindexequationsname}
\newcommand{\myindexequations}[1]{
	\addcontentsline{equ}{myindexequations}{\protect\numberline{\theequation}#1}
}
\setcounter{templateIndexEquations}{0}
\DeclareTotalCounter{templateIndexEquations}

% -----------------------------------------------------------------------------
% Reconfiguración de tamaño de páginas
% -----------------------------------------------------------------------------
\makeatletter
	\def\ifGm@preamble#1{\@firstofone}
	\appto\restoregeometry{
		\pdfpagewidth=\paperwidth
		\pdfpageheight=\paperheight}
	\apptocmd\newgeometry{
		\pdfpagewidth=\paperwidth
		\pdfpageheight=\paperheight}{}{}
\makeatother

% -----------------------------------------------------------------------------
% Configuración de hbox y vbox
% -----------------------------------------------------------------------------
\hfuzz=200pt
\vfuzz=200pt
\hbadness=\maxdimen
\vbadness=\maxdimen
% \sloppy Sloppy arruina portadas al exigir "justify", desde 6.4.2 se desactiva

% -----------------------------------------------------------------------------
% Configura las fuentes
% -----------------------------------------------------------------------------
\makeatletter
\def\Hv@scale{.95}
\makeatother

% -----------------------------------------------------------------------------
% Configuraciones de las tablas
% -----------------------------------------------------------------------------

% Reinicia el número de cada fila en todas las tablas
\makeatletter
\preto\tabular{\global\rownum=\z@}
\preto\tabularx{\global\rownum=\z@}
\makeatother

% -----------------------------------------------------------------------------
% Se activa el modo estricto de revisión de números de página
% -----------------------------------------------------------------------------
\strictpagecheck

% -----------------------------------------------------------------------------
% Actualización márgen títulos
% -----------------------------------------------------------------------------
\titlespacing{\section}{0pt}{20pt}{10pt}
\titlespacing{\subsection}{0pt}{15pt}{10pt}

% -----------------------------------------------------------------------------
% Se activa el word-wrap para textos con \texttt{}
% -----------------------------------------------------------------------------
\ttfamily \hyphenchar\the\font=`\-
\makeatletter
\g@addto@macro{\UrlBreaks}{\UrlOrds} % Word-wrap para las \url
\makeatother

% -----------------------------------------------------------------------------
% Se define el tipo de texto de los url
% -----------------------------------------------------------------------------
\urlstyle{\fonturl}

% -----------------------------------------------------------------------------
% -----------------------------------------------------------------------------
% Configuraciones del motor de compilación
% -----------------------------------------------------------------------------
\ifthenelse{\equal{\compilertype}{pdf2latex}}{
	% Nivel de compresión
	\pdfcompresslevel=\pdfcompilecompression
	
	% El óptimo es 2, según
	% http://texdoc.net/texmf-dist/doc/pdftex/manual/pdftex-a.pdf p.20
	\pdfdecimaldigits=2
	
	% Inclusión de PDF
	\pdfinclusionerrorlevel=0
	
	% Versión
	\pdfminorversion=\pdfcompileversion
	
	% Compresión de objetos
	\pdfobjcompresslevel=\pdfcompileobjcompression
}{
\ifthenelse{\equal{\compilertype}{xelatex}}{
}{
\ifthenelse{\equal{\compilertype}{lualatex}}{
}{
	\throwbadconfig{Compilador desconocido}{\compilertype}{pdf2latex,xelatex,lualatex}}}
}

% -----------------------------------------------------------------------------
% Profundidad de las secciones
% -----------------------------------------------------------------------------
\setcounter{secnumdepth}{4} % Límite máximo subsubsubsección

% Crea las subsubsubsecciones
\newcounter{subsubsubsection}[subsubsection]

% Establece el número de las subsubsubsecciones
\ifthenelse{\equal{\showdotaftersnum}{true}}{ % Agrega punto tras el número
	\renewcommand{\thesubsubsubsection}{\thesubsubsection.\arabic{subsubsubsection}.}
	\renewcommand{\theparagraph}{\thesubsubsubsection.\arabic{paragraph}.}
}{
	\renewcommand{\thesubsubsubsection}{\thesubsubsection.\arabic{subsubsubsection}}
	\renewcommand{\theparagraph}{\thesubsubsubsection.\arabic{paragraph}}
}

% -----------------------------------------------------------------------------
% Agrega compatibilidad de subsubsubsecciones al TOC
% -----------------------------------------------------------------------------
\makeatletter
	\def\toclevel@subsubsubsection{4}
	\def\toclevel@paragraph{5}
	\def\toclevel@subparagraph{6}
	\renewcommand\paragraph{\@startsection{paragraph}{5}{\z@}
		{3.25ex \@plus 1ex \@minus .2ex}
		{-1em}
		{\normalfont\normalsize\bfseries}}
	\renewcommand\subparagraph{\@startsection{subparagraph}{6}{\parindent}
		{3.25ex \@plus 1ex \@minus .2ex}
		{-1em}
		{\normalfont\normalsize\bfseries}}
	\ifthenelse{\equal{\showdotaftersnum}{true}}{
		\def\l@subsubsubsection{\@dottedtocline{4}{7.83em}{4.15em}} % Incremento 0.77+3.35 a 3.35
		\def\l@paragraph{\@dottedtocline{5}{11.98em}{4.92em}}
		\def\l@subparagraph{\@dottedtocline{6}{14.65em}{5.69em}}
	}{
		\def\l@subsubsubsection{\@dottedtocline{4}{6.97em}{4em}}
		\def\l@paragraph{\@dottedtocline{5}{10.97em}{5em}}
		\def\l@subparagraph{\@dottedtocline{6}{14em}{6em}}
	}
\makeatother

% -----------------------------------------------------------------------------
% Profundidad del índice y bookmarks pdf
% -----------------------------------------------------------------------------
\setcounter{tocdepth}{\indexdepth}

% -----------------------------------------------------------------------------
% Restauración número footnote
% -----------------------------------------------------------------------------
\ifthenelse{\equal{\footnoterestart}{none}}{
}{
\ifthenelse{\equal{\footnoterestart}{sec}}{
	\counterwithin*{footnote}{section}
}{
\ifthenelse{\equal{\footnoterestart}{ssec}}{
	\counterwithin*{footnote}{subsection}
}{
\ifthenelse{\equal{\footnoterestart}{sssec}}{
	\counterwithin*{footnote}{subsubsection}
}{
\ifthenelse{\equal{\footnoterestart}{ssssec}}{
	\counterwithin*{footnote}{subsubsubsection}
}{
\ifthenelse{\equal{\footnoterestart}{page}}{
	\counterwithin*{footnote}{page}
}{
\ifthenelse{\equal{\footnoterestart}{chap}}{
	\counterwithin*{footnote}{chapter}
}{
	\throwbadconfig{Formato reinicio numero footnote desconocido}{\footnoterestart}{none,chap,page,sec,ssec,sssec,ssssec}}}}}}}
}

% -----------------------------------------------------------------------------
% Restauración número ecuación, NOTA: NO hace nada, sólo se modifica en title.tex
% -----------------------------------------------------------------------------
\ifthenelse{\equal{\equationrestart}{none}}{
}{
\ifthenelse{\equal{\equationrestart}{chap}}{
}{
\ifthenelse{\equal{\equationrestart}{sec}}{
}{
\ifthenelse{\equal{\equationrestart}{ssec}}{
}{
\ifthenelse{\equal{\equationrestart}{sssec}}{
}{
\ifthenelse{\equal{\equationrestart}{ssssec}}{
}{
	\throwbadconfig{Formato reinicio numero ecuacion desconocido}{\equationrestart}{none,chap,sec,ssec,sssec,ssssec}}}}}}
}

% -----------------------------------------------------------------------------
% Configuración elementos matemáticos
% -----------------------------------------------------------------------------
\newtheoremstyle{miestilo}{\baselineskip}{3pt}{\itshape}{}{\bfseries}{}{.5em}{}
\newtheoremstyle{miobs}{\baselineskip}{3pt}{}{}{\bfseries}{}{.5em}{}
\theoremstyle{miestilo}

% Configura números
\ifthenelse{\equal{\showsectioncaptionmat}{none}}{
	\newtheorem{defn}{\namemathdefn}
	\newtheorem{teo}{\namemaththeorem}
	\newtheorem{cor}{\namemathcol}
	\newtheorem{lema}{\namemathlem}
	\newtheorem{prop}{\namemathprp}
}{
\ifthenelse{\equal{\showsectioncaptionmat}{chap}}{
	\newtheorem{defn}{\namemathdefn}[chapter]
	\newtheorem{teo}{\namemaththeorem}[chapter]
	\newtheorem{cor}{\namemathcol}[chapter]
	\newtheorem{lema}{\namemathlem}[chapter]
	\newtheorem{prop}{\namemathprp}[chapter]
}{
\ifthenelse{\equal{\showsectioncaptionmat}{sec}}{
	\newtheorem{defn}{\namemathdefn}[section]
	\newtheorem{teo}{\namemaththeorem}[section]
	\newtheorem{cor}{\namemathcol}[section]
	\newtheorem{lema}{\namemathlem}[section]
	\newtheorem{prop}{\namemathprp}[section]
}{
\ifthenelse{\equal{\showsectioncaptionmat}{ssec}}{
	\newtheorem{defn}{\namemathdefn}[subsection]
	\newtheorem{teo}{\namemaththeorem}[subsection]
	\newtheorem{cor}{\namemathcol}[subsection]
	\newtheorem{lema}{\namemathlem}[subsection]
	\newtheorem{prop}{\namemathprp}[subsection]
}{
\ifthenelse{\equal{\showsectioncaptionmat}{sssec}}{
	\newtheorem{defn}{\namemathdefn}[subsubsection]
	\newtheorem{teo}{\namemaththeorem}[subsubsection]
	\newtheorem{cor}{\namemathcol}[subsubsection]
	\newtheorem{lema}{\namemathlem}[subsubsection]
	\newtheorem{prop}{\namemathprp}[subsubsection]
}{
\ifthenelse{\equal{\showsectioncaptionmat}{ssssec}}{
	\newtheorem{defn}{\namemathdefn}[subsubsubsection]
	\newtheorem{teo}{\namemaththeorem}[subsubsubsection]
	\newtheorem{cor}{\namemathcol}[subsubsubsection]
	\newtheorem{lema}{\namemathlem}[subsubsubsection]
	\newtheorem{prop}{\namemathprp}[subsubsubsection]
}{
	\throwbadconfig{Valor configuracion incorrecto}{\showsectioncaptionmat}{none,chap,sec,ssec,sssec,ssssec}}}}}}
}
\theoremstyle{miobs}
\newtheorem*{ej}{\namemathej}
\newtheorem*{obs}{\namemathobs}

% -----------------------------------------------------------------------------
% Configuraciones del idioma
% -----------------------------------------------------------------------------

% Desactiva carácteres acentuados en operaciones matemáticas
\unaccentedoperators

% -----------------------------------------------------------------------------
% Configura número de objetos en el final del documento
% -----------------------------------------------------------------------------
\AtEndDocument{
	\addtocounter{equation}{\value{templateEquations}}
	\addtocounter{figure}{\value{templateFigures}}
	\addtocounter{lstlisting}{\value{templateListings}}
	\addtocounter{table}{\value{templateTables}}
}

% -----------------------------------------------------------------------------
% Formato de columnas
% -----------------------------------------------------------------------------
\newcolumntype{C}[1]{>{\centering\let\newline\\\arraybackslash\hspace{0pt}}m{#1}}
\newcolumntype{L}[1]{>{\raggedright\let\newline\\\arraybackslash\hspace{0pt}}m{#1}}
\newcolumntype{P}[1]{>{\centering\arraybackslash}p{#1}}
\newcolumntype{R}[1]{>{\raggedleft\let\newline\\\arraybackslash\hspace{0pt}}m{#1}}

% -----------------------------------------------------------------------------
% Parcha el entorno multicols
% -----------------------------------------------------------------------------
\let\SOURCEcaptionlrmargin\captionlrmargin
\BeforeBeginEnvironment{multicols}{\def\captionlrmargin{\captionlrmarginmc}\def\GLOBALenvmulticol{true}\setcaptionmargincm{\captionlrmargin}}
\AfterEndEnvironment{multicols}{\def\captionlrmargin{\SOURCEcaptionlrmargin}\def\GLOBALenvmulticol{false}\setcaptionmargincm{\captionlrmargin}}

% Operaciones especiales Template-Reporte
% -----------------------------------------------------------------------------
\title{\titulodelinforme}
\author{\autordeldocumento}
\date{\fechadelreporte}
\def\predocpageromannumber{false}
\def\predocresetpagenumber{false}
\def\tablaintegrantes{}
\makeatletter
\def\inserttitle{
	\null
	\vskip \titlesupmargin
	\begin{center}
		\let \footnote \thanks
		{\titlefontsize \titlefontstyle \@title \par}
		\ifthenelse{\equal{\titleshowauthor}{true}}{
			\vskip \titlelinemargin
			{\normalsize
				\lineskip 0.5em
				\begin{tabular}[t]{c}
					\@author
				\end{tabular}\par}
		}{}
		\ifthenelse{\equal{\titleshowcourse}{true}}{
			\vskip 0.5em
			{\normalsize \codigodelcurso\ - \nombredelcurso}
		}{}
		\ifthenelse{\equal{\titleshowdate}{true}}{
			\vskip 1em
			{\normalsize \@date}
		}{}
	\end{center}
	\par
	\vskip \titlelinemargin}
\makeatother

% CONFIGURACIÓN DE PÁGINA Y ENCABEZADOS
\newcommand{\templatePagecfg}{

	% -----------------------------------------------------------------------------
	% Numeración de páginas
	% -----------------------------------------------------------------------------
	\newpage
	\ifthenelse{\equal{\predocpageromannumber}{true}}{ % Si se usan números romanos en el predocumento
		\ifthenelse{\equal{\predocpageromanupper}{true}}{
			\pagenumbering{Roman}
		}{
			\pagenumbering{roman}
		}}{
		\pagenumbering{arabic}
	}
	\setcounter{page}{1}
	\setcounter{footnote}{0}

	% -----------------------------------------------------------------------------
	% Márgenes de páginas y tablas
	% -----------------------------------------------------------------------------
	\setpagemargincm{\pagemarginleft}{\pagemargintop}{\pagemarginright}{\pagemarginbottom}
	\def\arraystretch {\tablepaddingv} % Se ajusta el padding vertical de las tablas
	\setlength{\tabcolsep}{\tablepaddingh em} % Se ajusta el padding horizontal de las tablas

	% -----------------------------------------------------------------------------
	% Se define el punto decimal
	% -----------------------------------------------------------------------------
	\ifthenelse{\equal{\pointdecimal}{true}}{
		\decimalpoint}{
	}

	% -----------------------------------------------------------------------------
	% Definición de nombres de objetos
	% -----------------------------------------------------------------------------
	\renewcommand{\appendixname}{\nomltappendixsection} % Nombre del anexo (título)
	\renewcommand{\appendixpagename}{\nameappendixsection} % Nombre del anexo en índice
	\renewcommand{\appendixtocname}{\nameappendixsection} % Nombre del anexo en índice
	\renewcommand{\contentsname}{\nomltcont} % Nombre del índice
	\renewcommand{\figurename}{\nomltwfigure} % Nombre de la leyenda de las fig.
	\renewcommand{\listfigurename}{\nomltfigure} % Nombre del índice de figuras
	\renewcommand{\listtablename}{\nomlttable} % Nombre del índice de tablas
	\renewcommand{\lstlistingname}{\nomltwsrc} % Nombre leyenda del código fuente
	\renewcommand{\lstlistlistingname}{\nomltsrc} % Nombre índice código fuente
	\renewcommand{\refname}{\namereferences} % Nombre de las referencias (bibtex)
	\renewcommand{\bibname}{\namereferences} % Nombre de las referencias (natbib)
	\renewcommand{\tablename}{\nomltwtable} % Nombre de la leyenda de tablas

	% -----------------------------------------------------------------------------
	% Estilo de títulos
	% -----------------------------------------------------------------------------
	\sectionfont{\color{\titlecolor} \fontsizetitle \styletitle \selectfont}
	\subsectionfont{\color{\subtitlecolor} \fontsizesubtitle \stylesubtitle \selectfont}
	\subsubsectionfont{\color{\subsubtitlecolor} \fontsizesubsubtitle \stylesubsubtitle \selectfont}
	\titleformat{\subsubsubsection}{\color{\ssstitlecolor} \normalfont \fontsizessstitle \stylessstitle}{\thesubsubsubsection}{1em}{}
	\titlespacing*{\subsubsubsection}{0pt}{3.25ex plus 1ex minus .2ex}{1.5ex plus .2ex}
	
	% -----------------------------------------------------------------------------
	% Estilo citas
	% -----------------------------------------------------------------------------
	\ifthenelse{\equal{\stylecitereferences}{apacite}}{
		\renewcommand{\BOthers}[1]{\apacitebothers\hbox{}}
	}{}

	% -----------------------------------------------------------------------------
	% Se crean los header-footer
	% -----------------------------------------------------------------------------
	\fancyheadoffset{0pt} % Desactiva el offset de los header-footer
	\def\hfheaderimagesizeA {1.2} % Tamaño de las imágenes del encabezado estilo 3/13
	\ifthenelse{\equal{\hfstyle}{style1}}{
		\pagestyle{fancy}
		\newcommand{\COREstyledefinition}{
			\fancyhf{}
			\ifthenelse{\equal{\disablehfrightmark}{false}}{
				\fancyhead[L]{\nouppercase{\rightmark}}
			}{}
			\fancyhead[R]{\small \thepage}
			\ifthenelse{\equal{\hfwidthwrap}{true}}{
				\fancyfoot[L]{
					\begin{minipage}[t]{\hfwidthtitle\linewidth}
						\begin{flushleft}
							\small \textit{\titulodelinforme}
						\end{flushleft}
					\end{minipage}
				}
				\fancyfoot[R]{
					\begin{minipage}[t]{\hfwidthcourse\linewidth}
						\begin{flushright}
							\small \textit{\codigodelcurso \nombredelcurso}
						\end{flushright}
					\end{minipage}
				}
			}{
				\fancyfoot[L]{\small \textit{\titulodelinforme}}
				\fancyfoot[R]{\small \textit{\codigodelcurso \nombredelcurso}}
			}
			\renewcommand{\headrulewidth}{1pt}
			\renewcommand{\footrulewidth}{1pt}
		}
		\renewcommand{\sectionmark}[1]{\markboth{##1}{}}
		\COREstyledefinition
	}{
	\ifthenelse{\equal{\hfstyle}{style1-i}}{ % impar izquierdo
		\pagestyle{fancy}
		\newcommand{\COREstyledefinition}{
			\fancyhf{}
			\ifthenelse{\equal{\disablehfrightmark}{false}}{
				\fancyhead[LE,RO]{\nouppercase{\rightmark}}
			}{}
			\fancyhead[RE,LO]{\small \thepage}
			\ifthenelse{\equal{\hfwidthwrap}{true}}{
				\fancyfoot[L]{
					\begin{minipage}[t]{\hfwidthtitle\linewidth}
						\begin{flushleft}
							\small \textit{\titulodelinforme}
						\end{flushleft}
					\end{minipage}
				}
				\fancyfoot[R]{
					\begin{minipage}[t]{\hfwidthcourse\linewidth}
						\begin{flushright}
							\small \textit{\codigodelcurso \nombredelcurso}
						\end{flushright}
					\end{minipage}
				}
			}{
				\fancyfoot[L]{\small \textit{\titulodelinforme}}
				\fancyfoot[R]{\small \textit{\codigodelcurso \nombredelcurso}}
			}
			\renewcommand{\headrulewidth}{0.5pt}
			\renewcommand{\footrulewidth}{0.5pt}
		}
		\renewcommand{\sectionmark}[1]{\markboth{##1}{}}
		\COREstyledefinition
	}{
	\ifthenelse{\equal{\hfstyle}{style1-d}}{ % impar derecho
		\pagestyle{fancy}
		\newcommand{\COREstyledefinition}{
			\fancyhf{}
			\ifthenelse{\equal{\disablehfrightmark}{false}}{
				\fancyhead[LO,RE]{\nouppercase{\rightmark}}
			}{}
			\fancyhead[RO,LE]{\small \thepage}
			\ifthenelse{\equal{\hfwidthwrap}{true}}{
				\fancyfoot[L]{
					\begin{minipage}[t]{\hfwidthtitle\linewidth}
						\begin{flushleft}
							\small \textit{\titulodelinforme}
						\end{flushleft}
					\end{minipage}
				}
				\fancyfoot[R]{
					\begin{minipage}[t]{\hfwidthcourse\linewidth}
						\begin{flushright}
							\small \textit{\codigodelcurso \nombredelcurso}
						\end{flushright}
					\end{minipage}
				}
			}{
				\fancyfoot[L]{\small \textit{\titulodelinforme}}
				\fancyfoot[R]{\small \textit{\codigodelcurso \nombredelcurso}}
			}
			\renewcommand{\headrulewidth}{0.5pt}
			\renewcommand{\footrulewidth}{0.5pt}
		}
		\renewcommand{\sectionmark}[1]{\markboth{##1}{}}
		\COREstyledefinition
	}{
	\ifthenelse{\equal{\hfstyle}{style2}}{
		\pagestyle{fancy}
		\newcommand{\COREstyledefinition}{
			\fancyhf{}
			\ifthenelse{\equal{\disablehfrightmark}{false}}{
				\fancyhead[L]{\nouppercase{\rightmark}}
			}{}
			\fancyhead[R]{\small \thepage}
			\ifthenelse{\equal{\hfwidthwrap}{true}}{
				\fancyfoot[L]{
					\begin{minipage}[t]{\hfwidthtitle\linewidth}
						\begin{flushleft}
							\small \textit{\titulodelinforme}
						\end{flushleft}
					\end{minipage}
				}
				\fancyfoot[R]{
					\begin{minipage}[t]{\hfwidthcourse\linewidth}
						\begin{flushright}
							\small \textit{\codigodelcurso \nombredelcurso}
						\end{flushright}
					\end{minipage}
				}
			}{
				\fancyfoot[L]{\small \textit{\titulodelinforme}}
				\fancyfoot[R]{\small \textit{\codigodelcurso \nombredelcurso}}
			}
			\renewcommand{\headrulewidth}{0.5pt}
			\renewcommand{\footrulewidth}{0pt}
		}
		\renewcommand{\sectionmark}[1]{\markboth{##1}{}}
		\COREstyledefinition
	}{
	\ifthenelse{\equal{\hfstyle}{style2-i}}{ % impar izquierdo
		\pagestyle{fancy}
		\newcommand{\COREstyledefinition}{
			\fancyhf{}
			\ifthenelse{\equal{\disablehfrightmark}{false}}{
				\fancyhead[LE,RO]{\nouppercase{\rightmark}}
			}{}
			\fancyhead[RE,LO]{\small \thepage}
			\ifthenelse{\equal{\hfwidthwrap}{true}}{
				\fancyfoot[L]{
					\begin{minipage}[t]{\hfwidthtitle\linewidth}
						\begin{flushleft}
							\small \textit{\titulodelinforme}
						\end{flushleft}
					\end{minipage}
				}
				\fancyfoot[R]{
					\begin{minipage}[t]{\hfwidthcourse\linewidth}
						\begin{flushright}
							\small \textit{\codigodelcurso \nombredelcurso}
						\end{flushright}
					\end{minipage}
				}
			}{
				\fancyfoot[L]{\small \textit{\titulodelinforme}}
				\fancyfoot[R]{\small \textit{\codigodelcurso \nombredelcurso}}
			}
			\renewcommand{\headrulewidth}{0.5pt}
			\renewcommand{\footrulewidth}{0pt}
		}
		\renewcommand{\sectionmark}[1]{\markboth{##1}{}}
		\COREstyledefinition
	}{
	\ifthenelse{\equal{\hfstyle}{style1-d}}{ % impar derecho
		\pagestyle{fancy}
		\newcommand{\COREstyledefinition}{
			\fancyhf{}
			\ifthenelse{\equal{\disablehfrightmark}{false}}{
				\fancyhead[LO,RE]{\nouppercase{\rightmark}}
			}{}
			\fancyhead[RO,LE]{\small \thepage}
			\ifthenelse{\equal{\hfwidthwrap}{true}}{
				\fancyfoot[L]{
					\begin{minipage}[t]{\hfwidthtitle\linewidth}
						\begin{flushleft}
							\small \textit{\titulodelinforme}
						\end{flushleft}
					\end{minipage}
				}
				\fancyfoot[R]{
					\begin{minipage}[t]{\hfwidthcourse\linewidth}
						\begin{flushright}
							\small \textit{\codigodelcurso \nombredelcurso}
						\end{flushright}
					\end{minipage}
				}
			}{
				\fancyfoot[L]{\small \textit{\titulodelinforme}}
				\fancyfoot[R]{\small \textit{\codigodelcurso \nombredelcurso}}
			}
			\renewcommand{\headrulewidth}{0.5pt}
			\renewcommand{\footrulewidth}{0pt}
		}
		\renewcommand{\sectionmark}[1]{\markboth{##1}{}}
		\COREstyledefinition
	}{
	\ifthenelse{\equal{\hfstyle}{style3}}{
		\pagestyle{fancy}
		\newcommand{\COREstyledefinition}{
			\fancyhf{}
			\ifthenelse{\equal{\hfwidthwrap}{true}}{
				\fancyhead[L]{
					\begin{minipage}[t]{\hfwidthtitle\linewidth}
						\begin{flushleft}
							\small \textit{\codigodelcurso \nombredelcurso}
						\end{flushleft}
					\end{minipage}
				}
			}{
				\fancyhead[L]{\small \textit{\codigodelcurso \nombredelcurso}}
			}
			\fancyhead[R]{
				\includegraphics[width=\hfheaderimagesizeA cm]{\imagendepartamento}
				\vspace{-0.15cm}
			}
			\fancyfoot[C]{\thepage}
			\renewcommand{\headrulewidth}{0.5pt}
			\renewcommand{\footrulewidth}{0pt}
		}
		\COREstyledefinition
	}{
	\ifthenelse{\equal{\hfstyle}{style4}}{
		\pagestyle{fancy}
		\newcommand{\COREstyledefinition}{
			\fancyhf{}
			\ifthenelse{\equal{\disablehfrightmark}{false}}{
				\fancyhead[L]{\nouppercase{\rightmark}}
			}{}
			\fancyhead[R]{}
			\fancyfoot[C]{\small \thepage}
			\renewcommand{\headrulewidth}{0.5pt}
			\renewcommand{\footrulewidth}{0pt}
		}
		\renewcommand{\sectionmark}[1]{\markboth{##1}{}}
		\COREstyledefinition
	}{
	\ifthenelse{\equal{\hfstyle}{style5}}{
		\pagestyle{fancy}
		\newcommand{\COREstyledefinition}{
			\fancyhf{}
			\ifthenelse{\equal{\hfwidthwrap}{true}}{
				\fancyhead[L]{
					\begin{minipage}[t]{\hfwidthcourse\linewidth}
						\begin{flushleft}
							\codigodelcurso \nombredelcurso
						\end{flushleft}
					\end{minipage}
				}
				\ifthenelse{\equal{\disablehfrightmark}{false}}{
					\fancyhead[R]{
						\begin{minipage}[t]{\hfwidthtitle\linewidth}
							\begin{flushright}
								\nouppercase{\rightmark}
							\end{flushright}
						\end{minipage}
					}
				}{}
			}{
				\fancyhead[L]{\codigodelcurso \nombredelcurso}
				\ifthenelse{\equal{\disablehfrightmark}{false}}{
					\fancyhead[R]{\nouppercase{\rightmark}}
				}{}
			}
			\fancyfoot[L]{\departamentouniversidad, \nombreuniversidad}
			\fancyfoot[R]{\small \thepage}
			\renewcommand{\headrulewidth}{0pt}
			\renewcommand{\footrulewidth}{0pt}
		}
		\renewcommand{\sectionmark}[1]{\markboth{##1}{}}
		\COREstyledefinition
	}{
	\ifthenelse{\equal{\hfstyle}{style5-d}}{ % impar derecho
		\pagestyle{fancy}
		\newcommand{\COREstyledefinition}{
			\fancyhf{}
			\ifthenelse{\equal{\hfwidthwrap}{true}}{
				\fancyhead[L]{
					\begin{minipage}[t]{\hfwidthcourse\linewidth}
						\begin{flushleft}
							\codigodelcurso \nombredelcurso
						\end{flushleft}
					\end{minipage}
				}
				\ifthenelse{\equal{\disablehfrightmark}{false}}{
					\fancyhead[R]{
						\begin{minipage}[t]{\hfwidthtitle\linewidth}
							\begin{flushright}
								\nouppercase{\rightmark}
							\end{flushright}
						\end{minipage}
					}
				}{}
			}{
				\fancyhead[L]{\codigodelcurso \nombredelcurso}
				\ifthenelse{\equal{\disablehfrightmark}{false}}{
					\fancyhead[R]{\nouppercase{\rightmark}}
				}{}
			}
			\fancyfoot[LO,RE]{\departamentouniversidad, \nombreuniversidad}
			\fancyfoot[RO,LE]{\small \thepage}
			\renewcommand{\headrulewidth}{0pt}
			\renewcommand{\footrulewidth}{0pt}
		}
		\renewcommand{\sectionmark}[1]{\markboth{##1}{}}
		\COREstyledefinition
	}{
	\ifthenelse{\equal{\hfstyle}{style5-i}}{ % impar izquierdo
		\pagestyle{fancy}
		\newcommand{\COREstyledefinition}{
			\fancyhf{}
			\ifthenelse{\equal{\hfwidthwrap}{true}}{
				\fancyhead[L]{
					\begin{minipage}[t]{\hfwidthcourse\linewidth}
						\begin{flushleft}
							\codigodelcurso \nombredelcurso
						\end{flushleft}
					\end{minipage}
				}
				\ifthenelse{\equal{\disablehfrightmark}{false}}{
					\fancyhead[R]{
						\begin{minipage}[t]{\hfwidthtitle\linewidth}
							\begin{flushright}
								\nouppercase{\rightmark}
							\end{flushright}
						\end{minipage}
					}
				}{}
			}{
				\fancyhead[L]{\codigodelcurso \nombredelcurso}
				\ifthenelse{\equal{\disablehfrightmark}{false}}{
					\fancyhead[R]{\nouppercase{\rightmark}}
				}{}
			}
			\fancyfoot[LE,RO]{\departamentouniversidad, \nombreuniversidad}
			\fancyfoot[RE,LO]{\small \thepage}
			\renewcommand{\headrulewidth}{0pt}
			\renewcommand{\footrulewidth}{0pt}
		}
		\renewcommand{\sectionmark}[1]{\markboth{##1}{}}
		\COREstyledefinition
	}{
	\ifthenelse{\equal{\hfstyle}{style6}}{
		\pagestyle{fancy}
		\newcommand{\COREstyledefinition}{
			\fancyhf{}
			\fancyfoot[L]{\departamentouniversidad}
			\fancyfoot[C]{\thepage}
			\fancyfoot[R]{\nombreuniversidad}
			\renewcommand{\headrulewidth}{0pt}
			\renewcommand{\footrulewidth}{0pt}
		}
		\setlength{\headheight}{49pt}
		\COREstyledefinition
	}{
	\ifthenelse{\equal{\hfstyle}{style7}}{
		\pagestyle{fancy}
		\newcommand{\COREstyledefinition}{
			\fancyhf{}
			\fancyfoot[C]{\thepage}
			\renewcommand{\headrulewidth}{0pt}
			\renewcommand{\footrulewidth}{0pt}
		}
		\setlength{\headheight}{49pt}
		\COREstyledefinition
	}{
	\ifthenelse{\equal{\hfstyle}{style8}}{
		\pagestyle{fancy}
		\newcommand{\COREstyledefinition}{
			\fancyhf{}
			\fancyfoot[R]{\thepage}
			\renewcommand{\headrulewidth}{0pt}
			\renewcommand{\footrulewidth}{0pt}
		}
		\setlength{\headheight}{49pt}
		\COREstyledefinition
	}{
	\ifthenelse{\equal{\hfstyle}{style8-d}}{ % impar derecho
		\pagestyle{fancy}
		\newcommand{\COREstyledefinition}{
			\fancyhf{}
			\fancyfoot[RO,LE]{\thepage}
			\renewcommand{\headrulewidth}{0pt}
			\renewcommand{\footrulewidth}{0pt}
		}
		\setlength{\headheight}{49pt}
		\COREstyledefinition
	}{
	\ifthenelse{\equal{\hfstyle}{style8-i}}{ % impar izquierdo
		\pagestyle{fancy}
		\newcommand{\COREstyledefinition}{
			\fancyhf{}
			\fancyfoot[RE,LO]{\thepage}
			\renewcommand{\headrulewidth}{0pt}
			\renewcommand{\footrulewidth}{0pt}
		}
		\setlength{\headheight}{49pt}
		\COREstyledefinition
	}{
	\ifthenelse{\equal{\hfstyle}{style9}}{
		\pagestyle{fancy}
		\newcommand{\COREstyledefinition}{
			\fancyhf{}
			\ifthenelse{\equal{\disablehfrightmark}{false}}{
				\fancyhead[L]{\nouppercase{\rightmark}}
			}{}
			\fancyhead[R]{}
			\fancyfoot[L]{\small \textit{\titulodelinforme}}
			\fancyfoot[R]{\small \thepage}
			\renewcommand{\headrulewidth}{0.5pt}
			\renewcommand{\footrulewidth}{0.5pt}
		}
		\renewcommand{\sectionmark}[1]{\markboth{##1}{}}
		\COREstyledefinition
	}{
	\ifthenelse{\equal{\hfstyle}{style9-d}}{ % impar derecho
		\pagestyle{fancy}
		\newcommand{\COREstyledefinition}{
			\fancyhf{}
			\ifthenelse{\equal{\disablehfrightmark}{false}}{
				\fancyhead[L]{\nouppercase{\rightmark}}
			}{}
			\fancyhead[R]{}
			\fancyfoot[RE,LO]{\small \textit{\titulodelinforme}}
			\fancyfoot[RO,LE]{\small \thepage}
			\renewcommand{\headrulewidth}{0.5pt}
			\renewcommand{\footrulewidth}{0.5pt}
		}
		\renewcommand{\sectionmark}[1]{\markboth{##1}{}}
		\COREstyledefinition
	}{
	\ifthenelse{\equal{\hfstyle}{style9-i}}{ % impar izquierdo
		\pagestyle{fancy}
		\newcommand{\COREstyledefinition}{
			\fancyhf{}
			\ifthenelse{\equal{\disablehfrightmark}{false}}{
				\fancyhead[L]{\nouppercase{\rightmark}}
			}{}
			\fancyhead[R]{}
			\fancyfoot[RO,LE]{\small \textit{\titulodelinforme}}
			\fancyfoot[RE,LO]{\small \thepage}
			\renewcommand{\headrulewidth}{0.5pt}
			\renewcommand{\footrulewidth}{0.5pt}
		}
		\renewcommand{\sectionmark}[1]{\markboth{##1}{}}
		\COREstyledefinition
	}{
	\ifthenelse{\equal{\hfstyle}{style10}}{
		\pagestyle{fancy}
		\newcommand{\COREstyledefinition}{
			\fancyhf{}
			\ifthenelse{\equal{\hfwidthwrap}{true}}{
				\ifthenelse{\equal{\disablehfrightmark}{false}}{
					\fancyhead[L]{
						\begin{minipage}[t]{\hfwidthtitle\linewidth}
							\begin{flushleft}
								\nouppercase{\rightmark}
							\end{flushleft}
						\end{minipage}
					}
				}{}
				\fancyhead[R]{
					\begin{minipage}[t]{\hfwidthcourse\linewidth}
						\begin{flushright}
							\small \textit{\titulodelinforme}
						\end{flushright}
					\end{minipage}
				}
			}{
				\ifthenelse{\equal{\disablehfrightmark}{false}}{
					\fancyhead[L]{\nouppercase{\rightmark}}
				}{}
				\fancyhead[R]{\small \textit{\titulodelinforme}}
			}
			\fancyfoot[L]{}
			\fancyfoot[R]{\small \thepage}
			\renewcommand{\headrulewidth}{0.5pt}
			\renewcommand{\footrulewidth}{0.5pt}
		}
		\renewcommand{\sectionmark}[1]{\markboth{##1}{}}
		\COREstyledefinition
	}{
	\ifthenelse{\equal{\hfstyle}{style10-i}}{ % impar izquierdo
		\pagestyle{fancy}
		\newcommand{\COREstyledefinition}{
			\fancyhf{}
			\ifthenelse{\equal{\hfwidthwrap}{true}}{
				\ifthenelse{\equal{\disablehfrightmark}{false}}{
					\fancyhead[L]{
						\begin{minipage}[t]{\hfwidthtitle\linewidth}
							\begin{flushleft}
								\nouppercase{\rightmark}
							\end{flushleft}
						\end{minipage}
					}
				}{}
				\fancyhead[R]{
					\begin{minipage}[t]{\hfwidthcourse\linewidth}
						\begin{flushright}
							\small \textit{\titulodelinforme}
						\end{flushright}
					\end{minipage}
				}
			}{
				\ifthenelse{\equal{\disablehfrightmark}{false}}{
					\fancyhead[L]{\nouppercase{\rightmark}}
				}{}
				\fancyhead[R]{\small \textit{\titulodelinforme}}
			}
			\fancyfoot[L]{}
			\fancyfoot[RE,LO]{\small \thepage}
			\renewcommand{\headrulewidth}{0.5pt}
			\renewcommand{\footrulewidth}{0.5pt}
		}
		\renewcommand{\sectionmark}[1]{\markboth{##1}{}}
		\COREstyledefinition
	}{
	\ifthenelse{\equal{\hfstyle}{style10-d}}{ % impar derecho
		\pagestyle{fancy}
		\newcommand{\COREstyledefinition}{
			\fancyhf{}
			\ifthenelse{\equal{\hfwidthwrap}{true}}{
				\ifthenelse{\equal{\disablehfrightmark}{false}}{
					\fancyhead[L]{
						\begin{minipage}[t]{\hfwidthtitle\linewidth}
							\begin{flushleft}
								\nouppercase{\rightmark}
							\end{flushleft}
						\end{minipage}
					}
				}{}
				\fancyhead[R]{
					\begin{minipage}[t]{\hfwidthcourse\linewidth}
						\begin{flushright}
							\small \textit{\titulodelinforme}
						\end{flushright}
					\end{minipage}
				}
			}{
				\ifthenelse{\equal{\disablehfrightmark}{false}}{
					\fancyhead[L]{\nouppercase{\rightmark}}
				}{}
				\fancyhead[R]{\small \textit{\titulodelinforme}}
			}
			\fancyfoot[L]{}
			\fancyfoot[LE,RO]{\small \thepage}
			\renewcommand{\headrulewidth}{0.5pt}
			\renewcommand{\footrulewidth}{0.5pt}
		}
		\renewcommand{\sectionmark}[1]{\markboth{##1}{}}
		\COREstyledefinition
	}{
	\ifthenelse{\equal{\hfstyle}{style11}}{ % Similar a 1
		\pagestyle{fancy}
		\newcommand{\COREstyledefinition}{
			\fancyhf{}
			\ifthenelse{\equal{\disablehfrightmark}{false}}{
				\fancyhead[L]{\nouppercase{\rightmark}}
			}{}
			\fancyhead[R]{\small \thepage \nomnpageof \pageref{LastPage}}
			\ifthenelse{\equal{\hfwidthwrap}{true}}{
				\fancyfoot[L]{
					\begin{minipage}[t]{\hfwidthtitle\linewidth}
						\begin{flushleft}
							\small \textit{\titulodelinforme}
						\end{flushleft}
					\end{minipage}
				}
				\fancyfoot[R]{
					\begin{minipage}[t]{\hfwidthcourse\linewidth}
						\begin{flushright}
							\small \textit{\codigodelcurso \nombredelcurso}
						\end{flushright}
					\end{minipage}
				}
			}{
				\fancyfoot[L]{\small \textit{\titulodelinforme}}
				\fancyfoot[R]{\small \textit{\codigodelcurso \nombredelcurso}}
			}
			\renewcommand{\headrulewidth}{0.5pt}
			\renewcommand{\footrulewidth}{0.5pt}
		}
		\renewcommand{\sectionmark}[1]{\markboth{##1}{}}
		\COREstyledefinition
	}{
	\ifthenelse{\equal{\hfstyle}{style12}}{ % Similar a 6
		\pagestyle{fancy}
		\newcommand{\COREstyledefinition}{
			\fancyhf{}
			\fancyfoot[L]{\departamentouniversidad}
			\fancyfoot[C]{\thepage \nomnpageof \pageref{LastPage}}
			\fancyfoot[R]{\nombreuniversidad}
			\renewcommand{\headrulewidth}{0pt}
			\renewcommand{\footrulewidth}{0pt}
		}
		\setlength{\headheight}{49pt}
		\COREstyledefinition
	}{
	\ifthenelse{\equal{\hfstyle}{style13}}{ % Similar a 3
		\pagestyle{fancy}
		\newcommand{\COREstyledefinition}{
			\fancyhf{}
			\ifthenelse{\equal{\hfwidthwrap}{true}}{
				\fancyhead[L]{
					\begin{minipage}[t]{\hfwidthtitle\linewidth}
						\begin{flushleft}
							\small \textit{\codigodelcurso \nombredelcurso}
						\end{flushleft}
					\end{minipage}
				}
			}{
				\fancyhead[L]{\small \textit{\codigodelcurso \nombredelcurso}}
			}
			\fancyhead[R]{
				\includegraphics[width=\hfheaderimagesizeA cm]{\imagendepartamento}
				\vspace{-0.15cm}
			}
			\fancyfoot[C]{\thepage \nomnpageof \pageref{LastPage}}
			\renewcommand{\headrulewidth}{0.5pt}
			\renewcommand{\footrulewidth}{0pt}
		}
		\COREstyledefinition
	}{
	\ifthenelse{\equal{\hfstyle}{style14}}{ % Similar a 4
		\pagestyle{fancy}
		\newcommand{\COREstyledefinition}{
			\fancyhf{}
			\ifthenelse{\equal{\disablehfrightmark}{false}}{
				\fancyhead[L]{\nouppercase{\rightmark}}
			}{}
			\fancyhead[R]{}
			\fancyfoot[C]{\small \thepage \nomnpageof \pageref{LastPage}}
			\renewcommand{\headrulewidth}{0.5pt}
			\renewcommand{\footrulewidth}{0pt}
		}
		\renewcommand{\sectionmark}[1]{\markboth{##1}{}}
		\COREstyledefinition
	}{
	\ifthenelse{\equal{\hfstyle}{style15}}{ % Similar a 1
		\pagestyle{fancy}
		\newcommand{\COREstyledefinition}{
			\fancyhf{}
			\ifthenelse{\equal{\disablehfrightmark}{false}}{
				\fancyhead[L]{\nouppercase{\rightmark}}
			}{}
			\fancyhead[R]{}
			\fancyfoot[L]{
				\small \codigodelcurso \nombredelcurso
			}
			\fancyfoot[R]{
				\small \thepage
			}
			\renewcommand{\headrulewidth}{0.5pt}
			\renewcommand{\footrulewidth}{0.5pt}
		}
		\renewcommand{\sectionmark}[1]{\markboth{##1}{}}
		\COREstyledefinition
	}{
	\ifthenelse{\equal{\hfstyle}{style16}}{
		\pagestyle{fancy}
		\newcommand{\COREstyledefinition}{
			\fancyhf{}
			\renewcommand{\headrulewidth}{0pt}
			\renewcommand{\footrulewidth}{0pt}
		}
		\renewcommand{\sectionmark}[1]{\markboth{##1}{}}
		\COREstyledefinition
	}{
		\throwbadconfigondoc{Estilo de header-footer incorrecto}{\hfstyle}{style1 .. style16}}}}}}}}}}}}}}}}}}}}}}}}}}}}
	}
	\fancypagestyle{plain}{
		\fancyheadoffset{0pt}
		\COREstyledefinition
	}

	% -----------------------------------------------------------------------------
	% Muestra los números de línea
	% -----------------------------------------------------------------------------
	\ifthenelse{\equal{\showlinenumbers}{true}}{
		\linenumbers}{
	}

}

% CONFIGURACIONES FINALES
\newcommand{\templateFinalcfg}{
	
	% -----------------------------------------------------------------------------
	% Se reestablecen headers y footers
	% -----------------------------------------------------------------------------
	\markboth{}{}
	\newpage

	% Actualiza headers
	\ifthenelse{\equal{\disablehfrightmark}{false}}{
		
		% Define funciones generales
		\def\COREhfstyledefA { % 1, 2, 4, 9, 11, 14, 15
			\fancypagestyle{plain}{\fancyhead[L]{\nouppercase{\leftmark}}}
			\fancyhead[L]{\nouppercase{\leftmark}}
		}
		\def\COREhfstyledefB { % 5
			\fancypagestyle{plain}{
				\ifthenelse{\equal{\hfwidthwrap}{true}}{
					\fancyhead[R]{
						\begin{minipage}[t]{\hfwidthtitle\linewidth}
							\begin{flushright}
								\nouppercase{\leftmark}
							\end{flushright}
						\end{minipage}
					}
				}{
					\fancyhead[R]{\nouppercase{\leftmark}}
				}
			}
			\ifthenelse{\equal{\hfwidthwrap}{true}}{
				\fancyhead[R]{
					\begin{minipage}[t]{\hfwidthtitle\linewidth}
						\begin{flushright}
							\nouppercase{\leftmark}
						\end{flushright}
					\end{minipage}
				}
			}{
				\fancyhead[R]{\nouppercase{\leftmark}}
			}
		}
		\def\COREhfstyledefC { % 10
			\fancypagestyle{plain}{
				\ifthenelse{\equal{\hfwidthwrap}{true}}{
					\fancyhead[L]{
						\begin{minipage}[t]{\hfwidthtitle\linewidth}
							\begin{flushleft}
								\nouppercase{\leftmark}
							\end{flushleft}
						\end{minipage}
					}
				}{
					\fancyhead[L]{\nouppercase{\leftmark}}
				}
			}
			\ifthenelse{\equal{\hfwidthwrap}{true}}{
				\fancyhead[L]{
					\begin{minipage}[t]{\hfwidthtitle\linewidth}
						\begin{flushleft}
							\nouppercase{\leftmark}
						\end{flushleft}
					\end{minipage}
				}
			}{
				\fancyhead[L]{\nouppercase{\leftmark}}
			}
		}
		
		% Actualiza los header-footer
		\ifthenelse{\equal{\hfstyle}{style1}}{
			\COREhfstyledefA
		}{
		\ifthenelse{\equal{\hfstyle}{style1-i}}{ % impar izquierdo
			\fancypagestyle{plain}{\fancyhead[LE,RO]{\nouppercase{\leftmark}}}
			\fancyhead[LE,RO]{\nouppercase{\leftmark}}
		}{
		\ifthenelse{\equal{\hfstyle}{style1-d}}{ % impar derecho
			\fancypagestyle{plain}{\fancyhead[LO,RE]{\nouppercase{\leftmark}}}
			\fancyhead[LO,RE]{\nouppercase{\leftmark}}
		}{
		\ifthenelse{\equal{\hfstyle}{style2}}{
			\COREhfstyledefA
		}{
		\ifthenelse{\equal{\hfstyle}{style2-i}}{ % impar izquierdo
			\fancypagestyle{plain}{\fancyhead[LE,RO]{\nouppercase{\leftmark}}}
			\fancyhead[LE,RO]{\nouppercase{\leftmark}}
		}{
		\ifthenelse{\equal{\hfstyle}{style2-d}}{ % impar derecho
			\fancypagestyle{plain}{\fancyhead[LO,RE]{\nouppercase{\leftmark}}}
			\fancyhead[LO,RE]{\nouppercase{\leftmark}}
		}{
		\ifthenelse{\equal{\hfstyle}{style4}}{
			\COREhfstyledefA
		}{
		\ifthenelse{\equal{\hfstyle}{style5}}{
			\COREhfstyledefB
		}{
		\ifthenelse{\equal{\hfstyle}{style5-d}}{ % impar derecho
			\COREhfstyledefB
		}{
		\ifthenelse{\equal{\hfstyle}{style5-i}}{ % impar izquierdo
			\COREhfstyledefB
		}{
		\ifthenelse{\equal{\hfstyle}{style9}}{
			\COREhfstyledefA
		}{
		\ifthenelse{\equal{\hfstyle}{style9-d}}{ % impar derecho
			\COREhfstyledefA
		}{
		\ifthenelse{\equal{\hfstyle}{style9-i}}{ % impar izquierdo
			\COREhfstyledefA
		}{
		\ifthenelse{\equal{\hfstyle}{style10}}{
			\COREhfstyledefC
		}{
		\ifthenelse{\equal{\hfstyle}{style10-d}}{ % impar derecho
			\COREhfstyledefC
		}{
		\ifthenelse{\equal{\hfstyle}{style10-i}}{ % impar izquierdo
			\COREhfstyledefC
		}{
		\ifthenelse{\equal{\hfstyle}{style11}}{ % Similar a 1
			\COREhfstyledefA
		}{
		\ifthenelse{\equal{\hfstyle}{style14}}{ % Similar a 4
			\COREhfstyledefA
		}{
		\ifthenelse{\equal{\hfstyle}{style15}}{ % Similar a 1
			\COREhfstyledefA
		}{
			% No se encontró el header-footer, no hace nada
		}}}}}}}}}}}}}}}}}}}
	}{
	}

	% -----------------------------------------------------------------------------
	% Estilo de títulos - reestablece estilos por el índice
	% -----------------------------------------------------------------------------
	\sectionfont{\color{\titlecolor} \fontsizetitle \styletitle \selectfont}
	\subsectionfont{\color{\subtitlecolor} \fontsizesubtitle \stylesubtitle \selectfont}
	\subsubsectionfont{\color{\subsubtitlecolor} \fontsizesubsubtitle \stylesubsubtitle \selectfont}
	\titleformat{\subsubsubsection}{\color{\ssstitlecolor} \normalfont \fontsizessstitle \stylessstitle}{\thesubsubsubsection}{1em}{}
	\titlespacing*{\subsubsubsection}{0pt}{3.25ex plus 1ex minus .2ex}{1.5ex plus .2ex}

	% -----------------------------------------------------------------------------
	% Crea funciones para numerar objetos
	% -----------------------------------------------------------------------------

	% Numeración de la sección en los objetos CÓDIGO FUENTE
	\ifthenelse{\equal{\showsectioncaptioncode}{none}}{
		\def\sectionobjectnumcode {}
	}{
	\ifthenelse{\equal{\showsectioncaptioncode}{sec}}{
		\def\sectionobjectnumcode {\thesection\sectioncaptiondelimiter}
	}{
	\ifthenelse{\equal{\showsectioncaptioncode}{ssec}}{
		\def\sectionobjectnumcode {\thesubsection\sectioncaptiondelimiter}
	}{
	\ifthenelse{\equal{\showsectioncaptioncode}{sssec}}{
		\def\sectionobjectnumcode {\thesubsubsection\sectioncaptiondelimiter}
	}{
	\ifthenelse{\equal{\showsectioncaptioncode}{ssssec}}{
		\ifthenelse{\equal{\showdotaftersnum}{true}}{
			\def\sectionobjectnumcode {\thesubsubsubsection}
		}{
			\def\sectionobjectnumcode {\thesubsubsubsection\sectioncaptiondelimiter}
		}
	}{
	\ifthenelse{\equal{\showsectioncaptioncode}{chap}}{
		\def\sectionobjectnumcode {\thechapter\sectioncaptiondelimiter}
	}{
		\throwbadconfig{Valor configuracion incorrecto}{\showsectioncaptioncode}{none,chap,sec,ssec,sssec,ssssec}}}}}}
	}

	% Numeración de la sección en los objetos ECUACIONES
	\ifthenelse{\equal{\showsectioncaptioneqn}{none}}{
		\def\sectionobjectnumeqn {}
	}{
	\ifthenelse{\equal{\showsectioncaptioneqn}{sec}}{
		\def\sectionobjectnumeqn {\thesection\sectioncaptiondelimiter}
	}{
	\ifthenelse{\equal{\showsectioncaptioneqn}{ssec}}{
		\def\sectionobjectnumeqn {\thesubsection\sectioncaptiondelimiter}
	}{
	\ifthenelse{\equal{\showsectioncaptioneqn}{sssec}}{
		\def\sectionobjectnumeqn {\thesubsubsection\sectioncaptiondelimiter}
	}{
	\ifthenelse{\equal{\showsectioncaptioneqn}{ssssec}}{
		\ifthenelse{\equal{\showdotaftersnum}{true}}{
			\def\sectionobjectnumeqn {\thesubsubsubsection}
		}{
			\def\sectionobjectnumeqn {\thesubsubsubsection\sectioncaptiondelimiter}
		}
	}{
	\ifthenelse{\equal{\showsectioncaptioneqn}{chap}}{
		\def\sectionobjectnumeqn {\thechapter\sectioncaptiondelimiter}
	}{
		\throwbadconfig{Valor configuracion incorrecto}{\showsectioncaptioneqn}{none,chap,sec,ssec,sssec,ssssec}}}}}}
	}

	% Numeración de la sección en los objetos FIGURAS
	\ifthenelse{\equal{\showsectioncaptionfig}{none}}{
		\def\sectionobjectnumfig {}
	}{
	\ifthenelse{\equal{\showsectioncaptionfig}{sec}}{
		\def\sectionobjectnumfig {\thesection\sectioncaptiondelimiter}
	}{
	\ifthenelse{\equal{\showsectioncaptionfig}{ssec}}{
		\def\sectionobjectnumfig {\thesubsection\sectioncaptiondelimiter}
	}{
	\ifthenelse{\equal{\showsectioncaptionfig}{sssec}}{
		\def\sectionobjectnumfig {\thesubsubsection\sectioncaptiondelimiter}
	}{
	\ifthenelse{\equal{\showsectioncaptionfig}{ssssec}}{
		\ifthenelse{\equal{\showdotaftersnum}{true}}{
			\def\sectionobjectnumfig {\thesubsubsubsection}
		}{
			\def\sectionobjectnumfig {\thesubsubsubsection\sectioncaptiondelimiter}
		}
	}{
	\ifthenelse{\equal{\showsectioncaptionfig}{chap}}{
		\def\sectionobjectnumfig {\thechapter\sectioncaptiondelimiter}
	}{
		\throwbadconfig{Valor configuracion incorrecto}{\showsectioncaptionfig}{none,chap,sec,ssec,sssec,ssssec}}}}}}
	}

	% Numeración de la sección en los objetos TABLAS
	\ifthenelse{\equal{\showsectioncaptiontab}{none}}{
		\def\sectionobjectnumtab {}
	}{
	\ifthenelse{\equal{\showsectioncaptiontab}{sec}}{
		\def\sectionobjectnumtab {\thesection\sectioncaptiondelimiter}
	}{
	\ifthenelse{\equal{\showsectioncaptiontab}{ssec}}{
		\def\sectionobjectnumtab {\thesubsection\sectioncaptiondelimiter}
	}{
	\ifthenelse{\equal{\showsectioncaptiontab}{sssec}}{
		\def\sectionobjectnumtab {\thesubsubsection\sectioncaptiondelimiter}
	}{
	\ifthenelse{\equal{\showsectioncaptiontab}{ssssec}}{
		\ifthenelse{\equal{\showdotaftersnum}{true}}{
			\def\sectionobjectnumtab {\thesubsubsubsection}
		}{
			\def\sectionobjectnumtab {\thesubsubsubsection\sectioncaptiondelimiter}
		}
	}{
	\ifthenelse{\equal{\showsectioncaptiontab}{chap}}{
		\def\sectionobjectnumtab {\thechapter\sectioncaptiondelimiter}
	}{
		\throwbadconfig{Valor configuracion incorrecto}{\showsectioncaptiontab}{none,chap,sec,ssec,sssec,ssssec}}}}}}
	}

	% -----------------------------------------------------------------------------
	% Modifica numeración de objetos
	% -----------------------------------------------------------------------------

	% Código fuente, INCLUIR SECCIÓN
	\ifthenelse{\equal{\captionnumcode}{arabic}}{
		\renewcommand{\thelstlisting}{\sectionobjectnumcode\arabic{lstlisting}}
	}{
	\ifthenelse{\equal{\captionnumcode}{alph}}{
		\renewcommand{\thelstlisting}{\sectionobjectnumcode\alph{lstlisting}}
	}{
	\ifthenelse{\equal{\captionnumcode}{Alph}}{
		\renewcommand{\thelstlisting}{\sectionobjectnumcode\Alph{lstlisting}}
	}{
	\ifthenelse{\equal{\captionnumcode}{roman}}{
		\renewcommand{\thelstlisting}{\sectionobjectnumcode\roman{lstlisting}}
	}{
	\ifthenelse{\equal{\captionnumcode}{Roman}}{
		\renewcommand{\thelstlisting}{\sectionobjectnumcode\Roman{lstlisting}}
	}{
		\throwbadconfig{Tipo numero codigo fuente desconocido}{\captionnumcode}{arabic,alph,Alph,roman,Roman}}}}}
	}

	% Ecuaciones, INCLUIR SECCIÓN
	\ifthenelse{\equal{\captionnumequation}{arabic}}{
		\renewcommand{\theequation}{\sectionobjectnumeqn\arabic{equation}}
	}{
	\ifthenelse{\equal{\captionnumequation}{alph}}{
		\renewcommand{\theequation}{\sectionobjectnumeqn\alph{equation}}
	}{
	\ifthenelse{\equal{\captionnumequation}{Alph}}{
		\renewcommand{\theequation}{\sectionobjectnumeqn\Alph{equation}}
	}{
	\ifthenelse{\equal{\captionnumequation}{roman}}{
		\renewcommand{\theequation}{\sectionobjectnumeqn\roman{equation}}
	}{
	\ifthenelse{\equal{\captionnumequation}{Roman}}{
		\renewcommand{\theequation}{\sectionobjectnumeqn\Roman{equation}}
	}{
		\throwbadconfig{Tipo numero ecuacion desconocido}{\captionnumequation}{arabic,alph,Alph,roman,Roman}}}}}
	}

	% Figuras, INCLUIR SECCIÓN
	\ifthenelse{\equal{\captionnumfigure}{arabic}}{
		\renewcommand{\thefigure}{\sectionobjectnumfig\arabic{figure}}
	}{
	\ifthenelse{\equal{\captionnumfigure}{alph}}{
		\renewcommand{\thefigure}{\sectionobjectnumfig\alph{figure}}
	}{
	\ifthenelse{\equal{\captionnumfigure}{Alph}}{
		\renewcommand{\thefigure}{\sectionobjectnumfig\Alph{figure}}
	}{
	\ifthenelse{\equal{\captionnumfigure}{roman}}{
		\renewcommand{\thefigure}{\sectionobjectnumfig\roman{figure}}
	}{
	\ifthenelse{\equal{\captionnumfigure}{Roman}}{
		\renewcommand{\thefigure}{\sectionobjectnumfig\Roman{figure}}
	}{
		\throwbadconfig{Tipo numero figura desconocido}{\captionnumfigure}{arabic,alph,Alph,roman,Roman}}}}}
	}

	% Subfiguras, NO USAR SECCIONES YA QUE SON HIJAS DE FIGURA
	\ifthenelse{\equal{\captionnumsubfigure}{arabic}}{
		\renewcommand{\thesubfigure}{\arabic{subfigure}}
	}{
	\ifthenelse{\equal{\captionnumsubfigure}{alph}}{
		\renewcommand{\thesubfigure}{\alph{subfigure}}
	}{
	\ifthenelse{\equal{\captionnumsubfigure}{Alph}}{
		\renewcommand{\thesubfigure}{\Alph{subfigure}}
	}{
	\ifthenelse{\equal{\captionnumsubfigure}{roman}}{
		\renewcommand{\thesubfigure}{\roman{subfigure}}
	}{
	\ifthenelse{\equal{\captionnumsubfigure}{Roman}}{
		\renewcommand{\thesubfigure}{\Roman{subfigure}}
	}{
		\throwbadconfig{Tipo numero subfigura desconocido}{\captionnumsubfigure}{arabic,alph,Alph,roman,Roman}}}}}
	}

	% Tablas, INCLUIR SECCIÓN
	\ifthenelse{\equal{\captionnumtable}{arabic}}{
		\renewcommand{\thetable}{\sectionobjectnumtab\arabic{table}}
	}{
	\ifthenelse{\equal{\captionnumtable}{alph}}{
		\renewcommand{\thetable}{\sectionobjectnumtab\alph{table}}
	}{
	\ifthenelse{\equal{\captionnumtable}{Alph}}{
		\renewcommand{\thetable}{\sectionobjectnumtab\Alph{table}}
	}{
	\ifthenelse{\equal{\captionnumtable}{roman}}{
		\renewcommand{\thetable}{\sectionobjectnumtab\roman{table}}
	}{
	\ifthenelse{\equal{\captionnumtable}{Roman}}{
		\renewcommand{\thetable}{\sectionobjectnumtab\Roman{table}}
	}{
		\throwbadconfig{Tipo numero tabla desconocido}{\captionnumtable}{arabic,alph,Alph,roman,Roman}}}}}
	}

	% Subtablas, NO INCLUIR SECCIÓN YA QUE SON HIJAS DE LAS TABLAS
	\ifthenelse{\equal{\captionnumsubtable}{arabic}}{
		\renewcommand{\thesubtable}{\arabic{subtable}}
	}{
	\ifthenelse{\equal{\captionnumsubtable}{alph}}{
		\renewcommand{\thesubtable}{\alph{subtable}}
	}{
	\ifthenelse{\equal{\captionnumsubtable}{Alph}}{
		\renewcommand{\thesubtable}{\Alph{subtable}}
	}{
	\ifthenelse{\equal{\captionnumsubtable}{roman}}{
		\renewcommand{\thesubtable}{\roman{subtable}}
	}{
	\ifthenelse{\equal{\captionnumsubtable}{Roman}}{
		\renewcommand{\thesubtable}{\Roman{subtable}}
	}{
		\throwbadconfig{Tipo numero subtabla desconocido}{\captionnumsubtable}{arabic,alph,Alph,roman,Roman}}}}}
	}

	% -----------------------------------------------------------------------------
	% Se reestablecen números de página y secciones
	% -----------------------------------------------------------------------------

	% Se usa número de páginas en arábigo si es que se tenía activado los númeos romanos
	\ifthenelse{\equal{\predocpageromannumber}{true}}{
		\renewcommand{\thepage}{\arabic{page}}}{
	}

	% Reinicia número de página
	\ifthenelse{\equal{\predocresetpagenumber}{true}}{
		\setcounter{page}{1}}{
	}
	\setcounter{section}{0}
	\setcounter{footnote}{0}

	% -----------------------------------------------------------------------------
	% Muestra los números de línea
	% -----------------------------------------------------------------------------
	\ifthenelse{\equal{\showlinenumbers}{true}}{
		\linenumbers}{
	}

	% -----------------------------------------------------------------------------
	% Establece el estilo de las subsubsubsecciones
	% -----------------------------------------------------------------------------

}
\renewcommand{\abstractname}{\nameabstract}

\titleclass{\subsubsubsection}{straight}[\subsection]

% CONFIGURACIONES PERSONALES

% Separación entre columnas
\setlength{\columnsep}{1.5cm}

% Linea que separa las columnas
\setlength{\columnseprule}{1pt}
