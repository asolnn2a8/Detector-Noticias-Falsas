\section{Introducción}
{
% Debe mencionar las motivaciones de realizar este proyecto, definir sus objetivos, qué busca
% resolver o demostrar y una explicación general del trabajo realizado.

% Contexto de las noticias actualmente
Hoy en día, gracias al internet, los cibernautas están siendo continuamente bombardeados de información y de noticias.

% Importancia para resolver esto
Sin embargo, no siempre se puede con certeza cuando una noticia es verdadera o falsa, y más aún si la persona quién
está leyendo la noticia no tiene los conocimientos necesarios, o no ha desarrollado un pensamiento crítico para discernir 
cuando una noticia es verdadera o falsa.

% Objetivo
Es por esto que el objetivo de este trabajo es el de desarrollar una herramienta que permita \textbf{clasificar} una noticia como verdadera o falsa, en base a sus características.

Para esto, se utilizarán diversas herramientas clásicas de Aprendizaje de Máquinas, como lo es el Procesamiento de Lenguaje Natural, y Clasificación Binario.   


Es por esto que en este trabajo se explorarán métodos que permitirán verificar cuando una noticia es verdadera o falsa, utilizando las
herramientas de Aprendizaje de Máquinas. Es decir, \textbf{clasificaremos} una noticia como verdadera o falsa, en base a sus características.

% Origen de los datos
% Para esto, se obtendrán diversas noticias de US a través de 
% \href{https://www.kaggle.com/clmentbisaillon/fake-and-real-news-dataset}{kaggle}, etiquetadas por verdaderas o falsas. 

}

\section{Marco Teórico}
{
Este es un proyecto íntimamente ligado al procesamiento del lenguaje natural, y como tal, usará las herramientas principales de PLN. Esto es tokenización, lemanización y vectorización. 

El procesamiento del lenguaje natural investiga el uso de computadores para procesar o entender el lenguaje humano (natural) con el objetivo de realizar tarear útiles.  

\subsection{Herramientas de NLP}
ESCRIBIR LAS HERRAMIENTAS

}

\section{Descripción de los Datos}
{
\subsection{Descripción del dataset}
La base de datos se obtuvieron a partir de Kaggle, del \textit{dataset} llamado ``\href{https://www.kaggle.com/clmentbisaillon/fake-and-real-news-dataset}{Fake and 
real news dataset}''.

Esta base de datos consiste en dos \textit{datasets}: el de noticias falsas (17.903 datos) y el de noticias verdaderas (20.826 datos). Ambos datasets 
comparten las mismas columnas: \textit{title}, \textit{text}, \textit{subject} y \textit{date}. Estas corresponden al titular de
la noticia, al contenido de la noticia, al tipo de noticia que es y la fecha de la noticia, respectivamente.

Las columnas \textit{title} y \textit{text} son de tipo cadena; \textit{subject} es un tipo de dato categórico y \textit{date}
es un tipo de dato temporal. 

% \subsubsection{Descripción de las categorías de la columna \textit{subject}}
En el \textit{dataset} de las noticias verdaderas, existen dos etiquetas para la columna \textit{subject}, las cuales son \textbf{politicsNews} y \textbf{worldnews}. La etiqueta \textbf{politicsNews} posee 11.272 datos y es para referirse a las noticias que tengan una connotación política, mientras que la etiqueta \textbf{worldnews} posee 10.145 datos y es para referirse a las noticias que ocurren fuera de EE.UU.

Mientras que el \textit{dataset} de las noticias falsas, posee seis etiquetas:
\textbf{News}, \textbf{politics}, \textbf{Government News}, \textbf{left-news}, \textbf{US news} y \textbf{Middle-east}. A continuación se dará una breve descripción de cada categoría:
\begin{itemize}
    \item La etiqueta de \textbf{News} posee 9.050 datos y corresponde a las noticias estándar,
    \item La de \textbf{politics} posee 6.836 datos y corresponde a noticias de connotación política,
    \item La de \textbf{Government News} posee 1.568 datos y corresponde a noticias sobre el gobierno,
    \item La de \textbf{left-news} posee 4.456 datos y corresponde a noticias del espectro de la izquierda política,
    \item La de \textbf{US News} posee 784 datos y corresponde a noticias sobre EE.UU., y finalmente
    \item La de \textbf{Middle-east} posee 778 datos y corresponde a noticias del Oriente Medio.
\end{itemize}

El \textit{dataset} originalmente no posee valores nulos. Sin embargo, el \textit{dataset} de las noticias falsas posee 10 datos etiquetados como fecha, a pesar de que no lo sean.


\subsection{Tratamiento de los Datos}

Se crea una nueva columna en ambos \textit{datasets}, llamado \textit{true}, el cual determina si la noticia es verdadera o falsa. Creada la columna, se concatenan ambos \textit{datasets}. Además, se marcan los datos que no sean fechas de la columna \textit{date} por el tipo de dato llamado \textit{Not a Time}, y se eliminan los datos nulos que posea el \textit{dataset}.

Luego, se procederá a procesar la información de texto (esto es, para las columnas \textit{title} y \textit{text}) a través de NLP. Se escogerá una muestra de 2500 filas\footnote{Si se escogen más filas, la memoria para guardar la información del computador queda sobrecargada} de los datasets originales (para las noticias verdaderas y falsas) y se empieza por dejar el texto en minúsculas, luego se tokenizan, y por cada token, si es que no es una stopword, se lematiza, para finalmente unir todos los tokens lematizados en un único texto nuevamente, y esta información se reemplaza en dónde estaba la información no procesada.

Finalmente, se empieza a codificar la información a través de un vector de características. 

\subsection{Análisis Exploratorio de los Datos}

\lipsum[1]
}

\section{Metodología}
{
\subsection{Modelos}
Se escogerán tres modelos: \textit{Naïve Bayes} (NB), por su simpleza, \textit{Regresión Logística} (RL), y el \textit{Perceptrón}, ambos modelos por su capacidad de predecir sin utilizar técnicas geométricas.

\subsubsection{Preparación del modelo de NB} 
Se escogerá un prior uniforme para Naïve Bayes.

\subsubsection{Preparación del modelo de RL}
No se ajustarán hiperparámetros para este modelo.

\subsubsection{Preparación del modelo de Perceptrón}
No se ajustarán hiperparámetros para este modelo.

\subsection{Preparación del Conjunto de Entrenamiento y Test}
Luego de procesada toda la información del texto, se escoge una columna para realizar la predicción. En nuestro caso, se escogen las columnas \textit{title}, \textit{text} (ambas por separado) y una concatenación de \textit{title} $+$ \textit{text}. 

Fijada una columna, se procede a dividir de forma aleatoria un $70\%$ de los datos para el entrenamiento, y un $30\%$ de los datos para el test.

}

\section{Resultados}
{

\begin{table}
\centering
\caption{Tabla de Resumen}
\label{tab:resumen}
\begin{tabular}{rccc}
\toprule
 title &   text &  title and text \\
\midrule
 89.87 &  99.19 &           98.92 \\
 89.13 &  98.92 &           98.92 \\
 89.80 &  98.78 &           98.58 \\
\bottomrule
\end{tabular}
\end{table}


}

\section{Discusión y Conclusiones}
{
\lipsum[1]
}

\section{Futuras Mejoras}
{
\lipsum[1]
}

\section{Desarrollo}
{
% El procedimiento y los experimentos realizados, descripción de los datos, la metodología usada, las
% configuraciones (parámetros) de los algoritmos usados, etc. Explicar las decisiones que tomaron y el porqué. Si
% usaron un algoritmo para resolver el problema, explicar su elección y por qué no otro. No es necesario que sea
% una explicación rigurosa, puede ser por la naturaleza del problema, de forma intuitiva o por aspectos prácticos
\subsection{Descripción de los datos}
{
Los datos se obtuvieron de la Base de Datos \href{https://www.kaggle.com/clmentbisaillon/fake-and-real-news-dataset}{Fake and 
real news dataset}, de kaggle. 

Este consiste en dos datasets: el de noticias falsas (17903 datos) y el de noticias verdaderas (20826 datos). Ambos datasets 
comparten las mismas columnas: \textit{title}, \textit{text}, \textit{subject} y \textit{date}. Estas corresponden al titular de
la noticia, al contenido de la noticia, al tipo de noticia que es y la fecha de la noticia, respectivamente.

Las columnas \textit{title} y \textit{text} son de tipo cadena; \textit{subject} es un tipo de dato categórico y \textit{date}
es un tipo de dato temporal. 
}
}

\section{Resultados y Análisis}
{
% Mostrar, explicar y discutir los resultados obtenidos. El análisis incluye una interpretación superficial (funcionó/no funcionó), y también un análisis de si era o no lo esperado, por qué, y qué
% se podría mejorar
\lipsum[3]
}

\section{Conclusión}
{
% Las conclusiones que se obtienen del proyecto (No ponga conclusiones del estilo ”aprendí mucho”).
% Incluya las dificultades que se encontraron y propuestas de trabajo a futuro.
\lipsum[4]
}